
\noindent\rule{7in}{2.8pt}
\section{The Weak Interaction}
    
\begin{problem}{11.1}
Explain why the strong decay $\rho^0\to\pi^+\pi^-$ is observed, but the strong decay $\rho^0\to\pi^0\pi^0$ is not.
Hint : you will need to consider conservation of angular momentum, parity and the symmetry of the $\pi^0\pi^0$ wavefunction.
\end{problem}
\begin{solution}

\end{solution}

\noindent\rule{7in}{1.5pt}

%%%%%%%%%%%%%%%%%%%%%%%%%%%%%%%%%%%%%%%%%%%%%%%%%%%%%%%%%%%%%%%%%%%%%%%%%%%%%%%%%%%%%%%%%%%%%%%%%%%%%%%%%%%%%%%%%%%%%%%%%%%%%%%%%%%%%%%%

\begin{problem}{11.2}
When $\pi^−$ mesons are stopped in a deuterium target they can form a bound $(\pi^--D)$ state with zero orbital angular momentum, $l = 0$. The bound state decays by the strong interaction

\begin{align*}
    \pi^-D\to nn.
\end{align*}\\
By considering the possible spin and orbital angular momentum states of the nn system, and the required symmetry of the wavefunction, show that the pion has negative intrinsic parity.\\

Note : the deuteron has $J^P =1^+$ and the pion is a spin-0 particle.
\end{problem}
\begin{solution}

\end{solution}

\noindent\rule{7in}{1.5pt}

%%%%%%%%%%%%%%%%%%%%%%%%%%%%%%%%%%%%%%%%%%%%%%%%%%%%%%%%%%%%%%%%%%%%%%%%%%%%%%%%%%%%%%%%%%%%%%%%%%%%%%%%%%%%%%%%%%%%%%%%%%%%%%%%%%%%%%%%

\begin{problem}{11.3}
Classify the following quantities as either scalars (S), pseudoscalars (P), vectors (V) or axial-vectors (A):

\begin{enumerate}[label=(\alph*)]
    \item mechanical power, $P=\boldsymbol{F\cdot v}$;
    \item force, $\mathbf{F}$;
    \item torque, $\mathbf{G}=\boldsymbol{r\times F}$;
    \item vorticity, $\boldsymbol{\Omega}=\boldsymbol{\nabla \times v}$;
    \item magnetic flux, $\phi = \int \mathbf{B}\cdot \dif\mathbf{S}$;
    \item divergence of the electric field strength, $\boldsymbol{\nabla \cdot E}$.
\end{enumerate}
\end{problem}
\begin{solution}

\end{solution}

\noindent\rule{7in}{1.5pt}

%%%%%%%%%%%%%%%%%%%%%%%%%%%%%%%%%%%%%%%%%%%%%%%%%%%%%%%%%%%%%%%%%%%%%%%%%%%%%%%%%%%%%%%%%%%%%%%%%%%%%%%%%%%%%%%%%%%%%%%%%%%%%%%%%%%%%%%%

\begin{problem}{11.4}
In the annihilation process $e^+e^- \to qq$, the QED vector interaction leads to non-zero matrix elements only for the chiral combinations LR → LR, LR → RL, RL → RL, RL → LR. What are the corresponding allowed chiral combinations for S, P and S-P interactions?
\end{problem}
\begin{solution}

\end{solution}

\noindent\rule{7in}{1.5pt}

%%%%%%%%%%%%%%%%%%%%%%%%%%%%%%%%%%%%%%%%%%%%%%%%%%%%%%%%%%%%%%%%%%%%%%%%%%%%%%%%%%%%%%%%%%%%%%%%%%%%%%%%%%%%%%%%%%%%%%%%%%%%%%%%%%%%%%%%

\begin{problem}{11.5}
Consider the decay at rest $\tau^-\to\pi^-\nu_\tau$, where the spin of the tau is in the positive z-direction and the $\nu_\tau$ and $\pi^-$ travel in the $\pm z$-directions. Sketch the allowed spin configurations assuming that the form of the weak charged-current interaction is (i) V - A and (ii) V + A.
\end{problem}
\begin{solution}

\end{solution}

\noindent\rule{7in}{1.5pt}

%%%%%%%%%%%%%%%%%%%%%%%%%%%%%%%%%%%%%%%%%%%%%%%%%%%%%%%%%%%%%%%%%%%%%%%%%%%%%%%%%%%%%%%%%%%%%%%%%%%%%%%%%%%%%%%%%%%%%%%%%%%%%%%%%%%%%%%%

\begin{problem}{11.6}
Repeat the pion decay calculation for a pure scalar interaction and show that the predicted ratio of decay rates is

\begin{align*}
    \frac{\Gamma\left( \pi^- \to e^- \overbar{\nu}_e \right)}{\Gamma\left( \pi^- \to \mu^- \overbar{\nu}_\mu \right)} \approx 5.5 
\end{align*}\\
\end{problem}
\begin{solution}

\end{solution}

\noindent\rule{7in}{1.5pt}

%%%%%%%%%%%%%%%%%%%%%%%%%%%%%%%%%%%%%%%%%%%%%%%%%%%%%%%%%%%%%%%%%%%%%%%%%%%%%%%%%%%%%%%%%%%%%%%%%%%%%%%%%%%%%%%%%%%%%%%%%%%%%%%%%%%%%%%%

\begin{problem}{11.7}
Predict the ratio of the $K^- \to e^- \nu_e$ and $K^- \to \mu^- \nu_\mu$ weak interaction decay rates and compare your answer to the measured value of

\begin{align*}
    \frac{\Gamma\left( K^- \to e^- \overbar{\nu}_e \right)}{\Gamma\left( K^- \to \mu^- \overbar{\nu}_\mu \right)} =   \left( 2.488\pm0.012 \right)\times 10^{-5}  
\end{align*}\\
\end{problem}
\begin{solution}

\end{solution}

\noindent\rule{7in}{1.5pt}

%%%%%%%%%%%%%%%%%%%%%%%%%%%%%%%%%%%%%%%%%%%%%%%%%%%%%%%%%%%%%%%%%%%%%%%%%%%%%%%%%%%%%%%%%%%%%%%%%%%%%%%%%%%%%%%%%%%%%%%%%%%%%%%%%%%%%%%%

\begin{problem}{11.8}
Charged kaons have several weak interaction decay modes, the largest of which are

\begin{align*}
    K^+\left( u\overbar{s} \right) \to \mu^+ \nu_\mu \comma K^+\to\pi^+\pi^0 \andtxt K^+\to\pi^+\pi^+\pi-
\end{align*}\\
\begin{enumerate}[label=(\alph*)]
    \item Draw the Feynman diagrams for these three weak decays.
    \item Using the measured branching ratio 
    
    \begin{align*}
        \text{Br}\left( K^+ \to \mu^+\nu_\mu \right) = 63.55 \pm 0.11 \%
    \end{align*}\\
    \textit{estimate} the lifetime of the charged Kaon.
\end{enumerate}\vspace{0.12in}
Note: charged pions decay almost 100\% of the time by the weak interaction $\pi^+\to\mu^+\nu_\mu$ and have a lifetime of $(2.6033 \pm 0.0005) \times 10^{-8} $s.
\end{problem}
\begin{solution}

\end{solution}

\noindent\rule{7in}{1.5pt}

%%%%%%%%%%%%%%%%%%%%%%%%%%%%%%%%%%%%%%%%%%%%%%%%%%%%%%%%%%%%%%%%%%%%%%%%%%%%%%%%%%%%%%%%%%%%%%%%%%%%%%%%%%%%%%%%%%%%%%%%%%%%%%%%%%%%%%%%

\begin{problem}{11.9}
From the prediction of (11.25) and the above measured value of the charged pion lifetime, obtain a value for $f_\pi$.
\end{problem}
\begin{solution}

\end{solution}

\noindent\rule{7in}{1.5pt}

%%%%%%%%%%%%%%%%%%%%%%%%%%%%%%%%%%%%%%%%%%%%%%%%%%%%%%%%%%%%%%%%%%%%%%%%%%%%%%%%%%%%%%%%%%%%%%%%%%%%%%%%%%%%%%%%%%%%%%%%%%%%%%%%%%%%%%%%

\begin{problem}{11.10}
Calculate the partial decay width for the decay $\tau^- \to \pi^-\nu_\tau$ in the following steps.

\begin{enumerate}[label=(\alph*)]
    \item Draw the Feynman diagram and show that the corresponding matrix element is
    
    \begin{align*}
        \mathcal{M} \approx \sqrt{2}G_F f_\pi \overbar{u}(p_\nu) \gamma^\mu \frac{1}{2}\left( 1-\gamma^5 \right) u(p_\tau)g_{\mu\nu}p_\pi^\nu
    \end{align*}\\
    \item Taking the $\tau^-$ spin to be in the $z-$direction and the four-momentum of the neutrino to be 
    
    \begin{align*}
        p_\nu = p^\ast \left( 1,\sin\theta,0,\cos\theta \right)
    \end{align*}\\
    show that the leptonic current is 

    \begin{align*}
        j^\mu = \sqrt{2m_\tau p^\ast } \left( -\sin\halftheta,-\cos\halftheta,-i\cos\halftheta,\sin\halftheta \right)
    \end{align*}\\
    Note that, for this configuration, the spinor for the $\tau^-$ can be taken to be $u_1$ for a particle at rest.

    \item Write down the four-momentum of the $\pi^-$ and show that
    
    \begin{align*}
        \big| \mathcal{M} \big|^2 = 4G_F^2 f_\pi^2 m_\tau^3 p^\ast \sin^2 \halftheta
    \end{align*}\\
    \item Hence show that 
    
    \begin{align*}
        \Gamma\left( \tau^- \to \pi^- \nu_\tau \right) = \frac{G_F^2 f_\pi^2}{16\pi} m_\tau^3 \left( \frac{m_\tau^2 - m_\pi^2}{m_\tau^2} \right)^2
    \end{align*}\\
    \item Using the value of $f_\pi$ obtained in the previous problem, find a numerical value for $\Gamma\left( \tau^- \to \pi^- \nu_\tau \right)$
    \item Given that the lifetime of the $\tau$ lepton is measured to be $\tau_\tau=\num{2.906e-13}$s, find an approximate value for the $\tau^- \to \pi^- \nu_\tau$ branching ratio.
\end{enumerate}
\end{problem}
\begin{solution}

\end{solution}

\noindent\rule{7in}{1.5pt}

%%%%%%%%%%%%%%%%%%%%%%%%%%%%%%%%%%%%%%%%%%%%%%%%%%%%%%%%%%%%%%%%%%%%%%%%%%%%%%%%%%%%%%%%%%%%%%%%%%%%%%%%%%%%%%%%%%%%%%%%%%%%%%%%%%%%%%%%
