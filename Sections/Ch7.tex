
\noindent\rule{7in}{2.8pt}
\section{Electron-Proton Elastic Scattering}
    
\begin{problem}{7.1}
The derivation of (7.8) used the algebraic relation

\begin{align*}
    \left(\gamma+1\right)^2 \left(1-\kappa^2\right)^2 = 4,
\end{align*}\\
where 

\begin{align*}
    \kappa = \frac{\beta\gamma}{\gamma+1} \andtxt \left(1-\beta^2\right)\gamma^2 = 1
\end{align*}\\
Show that this holds.
\end{problem}
\begin{solution}
Using the given relation of $\kappa$ in terms of $\beta,\gamma$,

\begin{align*}
    (\gamma+1)^2 (1-\kappa^2)^2 &= (\gamma+1)^2 \left[ 1- \frac{\beta^2\gamma^2}{(\gamma+1)^2} \right]^2 \\[0.12in]
                                &= \left[ (\gamma+1) - \frac{\beta^2\gamma^2}{\gamma+1} \right]^2 \impliedby \beta^2\gamma^2 = \gamma^2 - 1 \\[0.12in]
                                &= \left[ (\gamma+1) - \frac{\gamma^2-1}{\gamma+1} \right]^2 = \left[ (\gamma+1) - (\gamma-1) \right]^2 = 4 \qed
\end{align*}
\end{solution}

\noindent\rule{7in}{1.5pt}

%%%%%%%%%%%%%%%%%%%%%%%%%%%%%%%%%%%%%%%%%%%%%%%%%%%%%%%%%%%%%%%%%%%%%%%%%%%%%%%%%%%%%%%%%%%%%%%%%%%%%%%%%%%%%%%%%%%%%%%%%%%%%%%%%%%%%%%%

\begin{problem}{7.2}
    By considering momentum and energy conservation in $e^-p$ elastic scattering from a proton at rest, find an expression for the fractional energy loss of the scattered electron $\left( E_1 - E_3 \right)/E_1$ in terms of the scattering angle and the parameter

    \begin{align*}
        \kappa = \frac{p}{E_1 +m_e} \equiv \frac{\beta\gamma}{\gamma+1}
    \end{align*}
\end{problem}
\begin{solution}
The author stated in the \href{https://www.hep.phy.cam.ac.uk/~thomson/MPP/ModernParticlePhysics_Errata.pdf}{\texttt{errata}} that this problem should be ignored. 
\end{solution}

\noindent\rule{7in}{1.5pt}

%%%%%%%%%%%%%%%%%%%%%%%%%%%%%%%%%%%%%%%%%%%%%%%%%%%%%%%%%%%%%%%%%%%%%%%%%%%%%%%%%%%%%%%%%%%%%%%%%%%%%%%%%%%%%%%%%%%%%%%%%%%%%%%%%%%%%%%%

\begin{problem}{7.3}
    In an $e^-p$ scattering experiment, the incident electron has energy $E_1 = 529.5 \MeV$ and the scattered electrons are detected at an angle of $\theta = 75^\circ$ relative to the incoming beam.
    \begin{enumerate}[label=(\alph*)]
        \item At this angle, almost all of the scattered electrons are measured to have an energy of $E_3 \approx 373 \MeV$. What can be concluded from this observation?
        \item Find the corresponding value of $Q^2$.
    \end{enumerate}
\end{problem}
\begin{solution}
\begin{enumerate}[label=(\alph*)]
    \item Let us assume that these observed electrons experienced a relativistic elastic scattering with the proton, then under such hypothesis the measured electron energy after the scattering process $E_3$ shall be :
    
    \begin{align*}
        E_3 = \frac{E_1 m_p}{m_p + E_1 (1-\cos\theta)} \simeq \frac{530\MeV \cdot 940 \MeV}{940 \MeV + 530 \MeV \cdot (1-0.25)} \simeq 372 \MeV
    \end{align*}\\
    which well corresponds with most of the observed energy of the scattered electrons. This implies that these electrons actually underwent an elastic collision with the proton, and the proton in concern remains intact.

    \item Using the observed values, 
    
    \begin{align*}
        Q^2 = \frac{2m_p E_1^2(1-\cos\theta)}{m_p + E_1 (1-\cos\theta)} \simeq \frac{2\cdot 940 \MeV \cdot 530^2 \MeV^2 \cdot (1-0.25)}{940 \MeV + 530 \MeV \cdot (1-0.25)} \simeq \boxed{0.32 \GeV^2}
    \end{align*}
\end{enumerate}
\end{solution}

\noindent\rule{7in}{1.5pt}

%%%%%%%%%%%%%%%%%%%%%%%%%%%%%%%%%%%%%%%%%%%%%%%%%%%%%%%%%%%%%%%%%%%%%%%%%%%%%%%%%%%%%%%%%%%%%%%%%%%%%%%%%%%%%%%%%%%%%%%%%%%%%%%%%%%%%%%%

\begin{problem}{7.4} \customlabel{P7.4}{7.4}
    For a spherically symmetric charge distribution $\rho(r)$, where

    \begin{align*}
        \int \rho(r) \dif^3 \mathbf{r} = 1,
    \end{align*}\\
    show that the form factor can be expressed as

    \begin{align*}
        F(\mathbf{q}^2) = \frac{4\pi}{q} \int^\infty_0 r \sin(qr) \rho(r) \dif r \simeq 1 -\frac{1}{6}q^2\expval{R^2} + \cdots
    \end{align*}\\
    where $\expval{R^2}$  is the mean square charge radius. Hence show that,

    \begin{align*}
        \expval{R^2} = -6 \left[ \frac{\dif F(\mathbf{q}^2)}{\dif q^2} \right]_{q^2=0}
    \end{align*}
\end{problem}
\begin{solution}
Starting from the definition of the form factor, and letting $\theta$ be the angle between $\mathbf{q}$ and $\mathbf{r}$ while $\phi$ as the azimuthal angle, 

\begin{align*}
    F(q^2) &= \int \rho(r) \exp\left[i\mathbf{q}\cdot\mathbf{r}\right] \dif^3 \mathbf{r} \\[0.12in]
           &= -\int \rho(r) \exp\left[ iqr\cos\theta \right]  r^2 \dif r \dif \cos\theta \dif \phi \\[0.12in]
           &=  2\pi \int_0^\infty r^2 \rho(r) \int_{-1}^{+1} \exp\left[ iqr\cos\theta \right] \dif \cos\theta    \dif r \\[0.12in]
           &= 2\pi \int_0^\infty r^2 \rho(r) \frac{2}{qr} \sin(qr) \dif r = \frac{4\pi}{q} \int^\infty_0 r \sin(qr) \rho(r) \dif r 
\end{align*}\\
Using the Taylor expansion of $\sin(qr)$, one could approximate the above expression of the form factor as 

\begin{align*}
    F(q^2) &=  \frac{4\pi}{q} \int^\infty_0 r \sin(qr) \rho(r) \dif r = \int_0^\infty 4\pi r^2 \rho(r) \dif r - \frac{1}{3!} q^2 \int_0^\infty r^5 \rho(r) \dif r + \mathcal{O}(q^4) \\[0.12in]
    &= \int \rho(r) \dif \mathbf{r} - \frac{1}{6} q^2 \int_0^\infty r^5 \rho(r) \dif r + \mathcal{O}(q^4)  \\[0.12in]
    &\simeq 1 - \frac{1}{6}q^2 \expval{R^2} \qed
\end{align*} 
\end{solution}

\noindent\rule{7in}{1.5pt}

%%%%%%%%%%%%%%%%%%%%%%%%%%%%%%%%%%%%%%%%%%%%%%%%%%%%%%%%%%%%%%%%%%%%%%%%%%%%%%%%%%%%%%%%%%%%%%%%%%%%%%%%%%%%%%%%%%%%%%%%%%%%%%%%%%%%%%%%

\begin{problem}{7.5}
    Using the answer to the previous question and the data in Figure 7.8a, estimate the root-mean-squared charge
    radius of the proton.
\end{problem}
\begin{solution}
From the data in Figure 7.8a, the slope of $G_{\text{E}}(q^2)$ can be obtained by drawing a tangent at $q^2=0$ which has an intercept in the $Q^2$-axis as $0.32 \GeV^2$, hence

\begin{align*}
    \frac{\dif G_{\text{E}}(q^2)}{\dif q^2} \bigg|_{\scriptsize q^2=0} \simeq - \frac{1}{0.32 \GeV^2} \simeq -3.125 \GeV^{-2}
\end{align*}\\
which could be interpreted as the root-mean-squared charge radius using the results from Problem (\ref{P7.4}) as,

\begin{align*}
    \expval{R^2} = -6 \cdot \frac{\dif G_{\text{E}}(q^2)}{\dif q^2} \bigg|_{\scriptsize q^2=0} \simeq 18.75 \GeV^{-2} \sim\boxed{ 0.72 \text{fm}^2}
\end{align*}
\end{solution}

\noindent\rule{7in}{1.5pt}

%%%%%%%%%%%%%%%%%%%%%%%%%%%%%%%%%%%%%%%%%%%%%%%%%%%%%%%%%%%%%%%%%%%%%%%%%%%%%%%%%%%%%%%%%%%%%%%%%%%%%%%%%%%%%%%%%%%%%%%%%%%%%%%%%%%%%%%%

\begin{problem}{7.6}
    From the slope and intercept of the right plot of Figure 7.7, obtain values for $G_{\text{M}}(0.292\text{GeV}^2 )$ and $G_{\text{E}}(0.292\text{GeV}^2)$
\end{problem}
\begin{solution}

\end{solution}

\noindent\rule{7in}{1.5pt}

%%%%%%%%%%%%%%%%%%%%%%%%%%%%%%%%%%%%%%%%%%%%%%%%%%%%%%%%%%%%%%%%%%%%%%%%%%%%%%%%%%%%%%%%%%%%%%%%%%%%%%%%%%%%%%%%%%%%%%%%%%%%%%%%%%%%%%%%

\begin{problem}{7.7}
    Use the data of Figure 7.7 to estimate $G_E (Q^2 )$ at $Q^2 = 0.500 \GeV^2 $.
\end{problem}
\begin{solution}

\end{solution}

\noindent\rule{7in}{1.5pt}

%%%%%%%%%%%%%%%%%%%%%%%%%%%%%%%%%%%%%%%%%%%%%%%%%%%%%%%%%%%%%%%%%%%%%%%%%%%%%%%%%%%%%%%%%%%%%%%%%%%%%%%%%%%%%%%%%%%%%%%%%%%%%%%%%%%%%%%%

\begin{problem}{7.8}
    The experimental data of Figure 7.8 can be described by the form factor

    \begin{align*}
        G(Q^2)= \frac{G(0)}{\left( 1+Q^2/Q_0^2 \right)^2}
    \end{align*}\\
    with $Q_0 = 0.71 \GeV$. Taking $Q_2 \approx \mathbf{q}^2$, show that this implies that proton has an exponential charge distri-bution of the form

    \begin{align*}
        \rho(\mathbf{r}) = \rho_0 e^{-\frac{r}{a}}
    \end{align*}\\
    and find the value of a.
\end{problem}
\begin{solution}

\end{solution}

\noindent\rule{7in}{1.5pt}

%%%%%%%%%%%%%%%%%%%%%%%%%%%%%%%%%%%%%%%%%%%%%%%%%%%%%%%%%%%%%%%%%%%%%%%%%%%%%%%%%%%%%%%%%%%%%%%%%%%%%%%%%%%%%%%%%%%%%%%%%%%%%%%%%%%%%%%%
