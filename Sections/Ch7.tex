
\noindent\rule{7in}{2.8pt}
\section{Electron-Proton Elastic Scattering}
    
\begin{problem}{7.1}
The derivation of (7.8) used the algebraic relation

\begin{align*}
    \left(\gamma+1\right)^2 \left(1-\kappa^2\right)^2 = 4,
\end{align*}\\
where 

\begin{align*}
    \kappa = \frac{\beta\gamma}{\gamma+1} \andtxt \left(1-\beta^2\right)\gamma^2 = 1
\end{align*}\\
Show that this holds.
\end{problem}
\begin{solution}

\end{solution}

\noindent\rule{7in}{1.5pt}

%%%%%%%%%%%%%%%%%%%%%%%%%%%%%%%%%%%%%%%%%%%%%%%%%%%%%%%%%%%%%%%%%%%%%%%%%%%%%%%%%%%%%%%%%%%%%%%%%%%%%%%%%%%%%%%%%%%%%%%%%%%%%%%%%%%%%%%%

\begin{problem}{7.2}
    By considering momentum and energy conservation in $e^-p$ elastic scattering from a proton at rest, find an expression for the fractional energy loss of the scattered electron $\left( E_1 - E_3 \right)/E_1$ in terms of the scattering angle and the parameter

    \begin{align*}
        \kappa = \frac{p}{E_1 +m_e} \equiv \frac{\beta\gamma}{\gamma+1}
    \end{align*}
\end{problem}
\begin{solution}

\end{solution}

\noindent\rule{7in}{1.5pt}

%%%%%%%%%%%%%%%%%%%%%%%%%%%%%%%%%%%%%%%%%%%%%%%%%%%%%%%%%%%%%%%%%%%%%%%%%%%%%%%%%%%%%%%%%%%%%%%%%%%%%%%%%%%%%%%%%%%%%%%%%%%%%%%%%%%%%%%%

\begin{problem}{7.3}
    In an $e^-p$ scattering experiment, the incident electron has energy $E_1 = 529.5 \MeV$ and the scattered electrons are detected at an angle of $\theta = 75^\circ$ relative to the incoming beam.
    \begin{enumerate}[label=(\alph*)]
        \item At this angle, almost all of the scattered electrons are measured to have an energy of $E_3 \approx 373 \MeV$. What can be concluded from this observation?
        \item Find the corresponding value of $Q^2$.
    \end{enumerate}
\end{problem}
\begin{solution}

\end{solution}

\noindent\rule{7in}{1.5pt}

%%%%%%%%%%%%%%%%%%%%%%%%%%%%%%%%%%%%%%%%%%%%%%%%%%%%%%%%%%%%%%%%%%%%%%%%%%%%%%%%%%%%%%%%%%%%%%%%%%%%%%%%%%%%%%%%%%%%%%%%%%%%%%%%%%%%%%%%

\begin{problem}{7.4}
    For a spherically symmetric charge distribution $\rho(r)$, where

    \begin{align*}
        \int \rho(r) \dif^3 \mathbf{r} = 1,
    \end{align*}\\
    show that the form factor can be expressed as

    \begin{align*}
        F(\mathbf{q}^2) = \frac{4\pi}{q} \int^\infty_0 r \sin(qr) \rho(r) \dif r \simeq 1 -\frac{1}{6}q^2\expval{R^2} + \cdots
    \end{align*}\\
    where $\expval{R^2}$  is the mean square charge radius. Hence show that,

    \begin{align*}
        \expval{R^2} = -6 \left[ \frac{\dif F(\mathbf{q}^2)}{\dif q^2} \right]_{q^2=0}
    \end{align*}
\end{problem}
\begin{solution}

\end{solution}

\noindent\rule{7in}{1.5pt}

%%%%%%%%%%%%%%%%%%%%%%%%%%%%%%%%%%%%%%%%%%%%%%%%%%%%%%%%%%%%%%%%%%%%%%%%%%%%%%%%%%%%%%%%%%%%%%%%%%%%%%%%%%%%%%%%%%%%%%%%%%%%%%%%%%%%%%%%

\begin{problem}{7.5}
    Using the answer to the previous question and the data in Figure 7.8a, estimate the root-mean-squared charge
    radius of the proton.
\end{problem}
\begin{solution}

\end{solution}

\noindent\rule{7in}{1.5pt}

%%%%%%%%%%%%%%%%%%%%%%%%%%%%%%%%%%%%%%%%%%%%%%%%%%%%%%%%%%%%%%%%%%%%%%%%%%%%%%%%%%%%%%%%%%%%%%%%%%%%%%%%%%%%%%%%%%%%%%%%%%%%%%%%%%%%%%%%

\begin{problem}{7.6}
    From the slope and intercept of the right plot of Figure 7.7, obtain values for $G_M(0.292\text{GeV}^2 )$ and
    $G_E(0.292\text{GeV}^2)$
\end{problem}
\begin{solution}

\end{solution}

\noindent\rule{7in}{1.5pt}

%%%%%%%%%%%%%%%%%%%%%%%%%%%%%%%%%%%%%%%%%%%%%%%%%%%%%%%%%%%%%%%%%%%%%%%%%%%%%%%%%%%%%%%%%%%%%%%%%%%%%%%%%%%%%%%%%%%%%%%%%%%%%%%%%%%%%%%%

\begin{problem}{7.7}
    Use the data of Figure 7.7 to estimate $G_E (Q^2 )$ at $Q^2 = 0.500 \GeV^2 $.
\end{problem}
\begin{solution}

\end{solution}

\noindent\rule{7in}{1.5pt}

%%%%%%%%%%%%%%%%%%%%%%%%%%%%%%%%%%%%%%%%%%%%%%%%%%%%%%%%%%%%%%%%%%%%%%%%%%%%%%%%%%%%%%%%%%%%%%%%%%%%%%%%%%%%%%%%%%%%%%%%%%%%%%%%%%%%%%%%

\begin{problem}{7.8}
    The experimental data of Figure 7.8 can be described by the form factor

    \begin{align*}
        G(Q^2)= \frac{G(0)}{\left( 1+Q^2/Q_0^2 \right)^2}
    \end{align*}\\
    with $Q_0 = 0.71 \GeV$. Taking $Q_2 \approx \mathbf{q}^2$, show that this implies that proton has an exponential charge distri-bution of the form

    \begin{align*}
        \rho(\mathbf{r}) = \rho_0 e^{-\frac{r}{a}}
    \end{align*}\\
    and find the value of a.
\end{problem}
\begin{solution}

\end{solution}

\noindent\rule{7in}{1.5pt}

%%%%%%%%%%%%%%%%%%%%%%%%%%%%%%%%%%%%%%%%%%%%%%%%%%%%%%%%%%%%%%%%%%%%%%%%%%%%%%%%%%%%%%%%%%%%%%%%%%%%%%%%%%%%%%%%%%%%%%%%%%%%%%%%%%%%%%%%
