
\noindent\rule{7in}{2.8pt}
\section{The Dirac Equation}
    
\begin{problem}{4.1}
Show that
\begin{align*}
    \left[ \hat{\boldp}^2,\hat{\mathbf{r}}\times \hat{\boldp}\right]=0,
\end{align*}
and hence the Hamiltonian of the free-particle Schrödinger equation commutes with the angular momentum operator.
\end{problem}
\begin{solution}
One could expand the given commutator as, 

\begin{align*}
    \left[ \hat{\boldp}^2 , \hat{\mathbf{r}} \times \hat{\boldp} \right] &=  \left[ \hat{\boldp}_a\hat{\boldp}_a,\epsilon_{abc}r_c \hat{\boldp}_b \hat{\mathbf{c}} \right]\\[0.15in]
        &= \epsilon_{abc}r_c \left[ \hat{\boldp}_a\hat{\boldp}_a,\hat{\boldp}_b \hat{\mathbf{c}} \right]\\[0.15in]
        &= \epsilon_{abc}r_c \left\{ \hat{\boldp}_a \left[ \hat{\boldp}_a,\hat{\boldp}_b \hat{\mathbf{c}} \right] + \left[ \hat{\boldp}_a,\hat{\boldp}_b \hat{\mathbf{c}} \right] \hat{\boldp}_a  \right\} \\[0.15in]
        &= \epsilon_{abc}r_c \left\{ \hat{\boldp}_a \left[ \hat{\boldp}_a,\hat{\boldp}_b \hat{\mathbf{c}} \right] + \left[ \hat{\boldp}_a,\hat{\boldp}_b \hat{\mathbf{c}} \right] \hat{\boldp}_a  \right\} \quad \impliedby  \left[ \hat{\boldp}_a,\hat{\boldp}_b \hat{\mathbf{c}} \right] = \delta_{ab} \hat{\mathbf{c}} - i \delta_{ac} \hat{\boldp}_b
\end{align*}
\end{solution}

\noindent\rule{7in}{1.5pt}

%%%%%%%%%%%%%%%%%%%%%%%%%%%%%%%%%%%%%%%%%%%%%%%%%%%%%%%%%%%%%%%%%%%%%%%%%%%%%%%%%%%%%%%%%%%%%%%%%%%%%%%%%%%%%%%%%%%%%%%%%%%%%%%%%%%%%%%%

\begin{problem}{4.2}
Show that $u_1$ and $u_2$ are orthogonal, i.e. $u_1^\dagger u_2 =0$.
\end{problem}
\begin{solution}
Let us first denote 

\begin{align*}
    u_\uparrow = \begin{pmatrix}
        1 \\
        0
    \end{pmatrix} \quad \text{and} \quad 
    u_\downarrow = \begin{pmatrix}
        0 \\
        1
    \end{pmatrix} .
\end{align*}\\
One should note that $u_\uparrow^{\dagger}u_\downarrow = 0$. $u_1,u_2$ could also be expressed in terms of 

\begin{align*}
    u_1 = \begin{pmatrix}[1.25]
        u_\uparrow \\
        \frac{\boldsymbol{\sigma}\cdot\boldp}{E+m} u_\uparrow
    \end{pmatrix} \quad \text{and} \quad 
    u_2 = \begin{pmatrix}[1.25]
        u_\downarrow \\
        \frac{\boldsymbol{\sigma}\cdot\boldp}{E+m} u_\downarrow
    \end{pmatrix}.
\end{align*}\\
Now $u_1^\dagger u_2$ could be written as, 

\begin{align*}
    u_1^\dagger u_2 &=  \begin{pmatrix}[1.25]
        u_\uparrow \\
        \frac{\boldsymbol{\sigma}\cdot\boldp}{E+m} u_\uparrow
    \end{pmatrix}^\dagger
    \begin{pmatrix}[1.25]
        u_\downarrow \\
        \frac{\boldsymbol{\sigma}\cdot\boldp}{E+m} u_\downarrow
    \end{pmatrix} \\[0.15in]
    &=  \begin{pmatrix}[1.2]
        u_\uparrow^\dagger & \frac{1}{E+m} \left(\left(\boldsymbol{\sigma}\cdot\boldp\right) u_\uparrow\right)^\dagger
    \end{pmatrix}
    \begin{pmatrix}[1.25]
        u_\downarrow \\
        \frac{\boldsymbol{\sigma}\cdot\boldp}{E+m} u_\downarrow \\
    \end{pmatrix} \\[0.15in]
    &=  u_\uparrow^\dagger u_\downarrow + \frac{1}{\left(E+m\right)^2} \left(\left(\boldsymbol{\sigma}\cdot\boldp\right) u_\uparrow\right)^\dagger \left(\left(\boldsymbol{\sigma}\cdot\boldp\right) u_\downarrow\right) \\[0.15in]
    &=  u_\uparrow^\dagger u_\downarrow + \frac{1}{\left(E+m\right)^2}  u_\uparrow^\dagger \left(\boldsymbol{\sigma}\cdot\boldp\right)^\dagger \left(\boldsymbol{\sigma}\cdot\boldp\right) u_\downarrow  
\end{align*}
One could use the fact that,

\begin{align*}
    \left(\boldsymbol{\sigma}\cdot\boldp\right)^\dagger \left(\boldsymbol{\sigma}\cdot\boldp\right) =  \left(\boldsymbol{\sigma}\cdot\boldp\right)^2 = \begin{pmatrix}[1.5]
        p_x^2 + p_y^2 + p_z^2 & 0 \\
        0 & p_x^2 + p_y^2 + p_z^2 \\
    \end{pmatrix} = \left(E^2-m^2\right)I_2
\end{align*}\\
Thus it could be tidied up as, 

\begin{align*}
    u_1^\dagger u_2 &=  u_\uparrow^\dagger u_\downarrow + \frac{1}{\left(E+m\right)^2}  u_\uparrow^\dagger \left(\boldsymbol{\sigma}\cdot\boldp\right)^\dagger \left(\boldsymbol{\sigma}\cdot\boldp\right) u_\downarrow  \\[0.15in]
                    &= \left[ 1 + \frac{E^2-m^2}{\left(E+m\right)^2}  \right]  u_\uparrow^\dagger u_\downarrow = 0 \qed
\end{align*}
\end{solution}

\noindent\rule{7in}{1.5pt}

%%%%%%%%%%%%%%%%%%%%%%%%%%%%%%%%%%%%%%%%%%%%%%%%%%%%%%%%%%%%%%%%%%%%%%%%%%%%%%%%%%%%%%%%%%%%%%%%%%%%%%%%%%%%%%%%%%%%%%%%%%%%%%%%%%%%%%%%

\begin{problem}{4.3}
Verify the statement that the Einstein energy-momentum relationship is recovered if any of the four Dirac spinors of (4.48)
are subtitutes into the Dirac equation written in terms of momentum, $\left( \gamma^\mu p_\mu-m \right)u=0$.
\end{problem}
\begin{solution}
Let us choose $u_1$ to plug in the Dirac equation. Then it could be expressed as,

\begin{align*}
    \left(\slashed{p}-m\right)u_1 = 0 &\implies \begin{pmatrix}[1.2]
        \left(E-m\right)I_2 & -\sigp \\
        \sigp & -\left(E+m\right) I_2 
    \end{pmatrix}
    \begin{pmatrix}[1.2]
        E+m \\
        0 \\
        p_z \\
        p_x + ip_y 
    \end{pmatrix} = 0  \\[0.15in]
    &\implies \begin{pmatrix}[1.2]
        E^2-m^2 \\
        0
    \end{pmatrix} + \sigp\begin{pmatrix}[1.2]
        E+m-p_z \\
        -p_x -ip_y
    \end{pmatrix} - \left(E+m\right)\begin{pmatrix}[1.2]
        p_z \\
        p_x+ip_y
    \end{pmatrix} = 0 \\[0.15in]
    \text{[first row]} &\implies \left(E^2-m^2\right) + \left(E+m\right)p_z - \left(p_x^2+p_y^2+p_z^2\right) - \left(E+m\right) p_z = 0 \\[0.15in] 
    &\implies E^2 = p_x^2 +p_y^2 + p_z^2 +m^2  \qed
\end{align*}
\end{solution}

\noindent\rule{7in}{1.5pt}

%%%%%%%%%%%%%%%%%%%%%%%%%%%%%%%%%%%%%%%%%%%%%%%%%%%%%%%%%%%%%%%%%%%%%%%%%%%%%%%%%%%%%%%%%%%%%%%%%%%%%%%%%%%%%%%%%%%%%%%%%%%%%%%%%%%%%%%%

\begin{problem}{4.4}
For a particle with four-momentum $p^\mu = (E,\boldp)$, the general solution to the free-particle Dirac equation can be written
\begin{align*}
    \psi(p) = \left[ au_1(p)+bu_2(p) \right] e^{i\left( \boldp\cdot\boldx - Et \right)}
\end{align*}
Using the explicit forms for $u_1$ and $u_2$, show that the four-vector current $j^\mu=(\rho,\mathbf{j})$ is given by
\begin{align*}
    j^\mu = 2 p^\mu
\end{align*}
Furthermore, show that the resulting probability density and probability current are consistent with a particle
moving with velocity $\beta = p /E$.
\end{problem}
\begin{solution}

\end{solution}

\noindent\rule{7in}{1.5pt}

%%%%%%%%%%%%%%%%%%%%%%%%%%%%%%%%%%%%%%%%%%%%%%%%%%%%%%%%%%%%%%%%%%%%%%%%%%%%%%%%%%%%%%%%%%%%%%%%%%%%%%%%%%%%%%%%%%%%%%%%%%%%%%%%%%%%%%%%

\begin{problem}{4.5}
Writing the four-component spinor $u$ in terms of two two-component vectors
\begin{align*}
    u = \begin{pmatrix}
        u_A \\
        u_B
    \end{pmatrix},
\end{align*}
show that in the non-relativistic limit, where $\beta \cong v/c \ll 1$, the components of $u_B$ are smaller than those of $u_A$ by a factor $v/c$.
\end{problem}
\begin{solution}

\end{solution}

\noindent\rule{7in}{1.5pt}

%%%%%%%%%%%%%%%%%%%%%%%%%%%%%%%%%%%%%%%%%%%%%%%%%%%%%%%%%%%%%%%%%%%%%%%%%%%%%%%%%%%%%%%%%%%%%%%%%%%%%%%%%%%%%%%%%%%%%%%%%%%%%%%%%%%%%%%%

\begin{problem}{4.6}
By considering the three cases $\mu=\nu=0$, $\mu=\nu\neq 0$ and $\mu \neq \nu$ show that 
\begin{align*}
    \gamma^\mu \gamma^\nu + \gamma^\nu \gamma^\mu = 2 g^{\mu\nu}.
\end{align*}
\end{problem}
\begin{solution}
For brevity, the gamma matrices will be presented in terms of direct products as :

\begin{align*}
    \gamma^0 &= \begin{pmatrix}[1.1]
        1 & 0 \\
        0 & -1 
    \end{pmatrix} \otimes I_2 \equiv \beta  \\[0.15in]
    \gamma^i &\equiv \beta \alpha_i \\[0.15in]
             &= \left[\begin{pmatrix}[1.1]
                1 & 0 \\
                0 & -1 
            \end{pmatrix} \otimes I_2 \right] \left[\begin{pmatrix}[1.1]
                0 & 1 \\
                1 & 0 
            \end{pmatrix} \otimes \sigma_i \right] \\[0.15in]
             &= \left[\begin{pmatrix}[1.1]
                0 & 1 \\
                -1 & 0 
            \end{pmatrix} \otimes \sigma_i \right]
\end{align*}
Then considering the three cases,

\begin{enumerate}[label=(\alph*)]
    \item  $\mu=\nu=0$
    \begin{align*}
        \gamma^\mu \gamma^\nu + \gamma^\nu \gamma^\mu &= 2 \gamma^0\gamma^0 \\[0.15in]
        &= 2 \left[\begin{pmatrix}[1.1]
            1 & 0 \\
            0 & -1 
        \end{pmatrix} \otimes I_2 \right] \left[\begin{pmatrix}[1.1]
            1 & 0 \\
            0 & -1 
        \end{pmatrix} \otimes I_2 \right] \\[0.15in]
        &= 2 \begin{pmatrix}[1.1]
            1 & 0 \\
            0 & 1 
        \end{pmatrix} \otimes I_2  = 2I_4 = 2g^{00} I_4
    \end{align*}
    \item $\mu\neq0,\nu\neq0$ 
    \begin{align*}
        \gamma^\mu \gamma^\nu + \gamma^\nu \gamma^\mu &= \gamma^i \gamma^j + \gamma^j \gamma^i \\[0.15in]
                                                      &= \begin{pmatrix}[1.1]
                                                        0 & 1 \\
                                                        -1 & 0 
                                                    \end{pmatrix}^2  \otimes \sigma_i\sigma_j + \begin{pmatrix}[1.1]
                                                        0 & 1 \\
                                                        -1 & 0 
                                                    \end{pmatrix}^2  \otimes \sigma_j\sigma_i \\[0.15in]
                                                      &= - I_2 \otimes \left( \sigma_j\sigma_i + \sigma_i\sigma_j \right) \\[0.15in]
                                                      &= - I_2 \otimes \left\{ \sigma_i,\sigma_j \right\} \\[0.15in]
                                                      &= - 2 I_2 \otimes \delta_{ij} I_2 = -2 \delta_{ij}I_4  = 2g^{ij} I_4                                                      
    \end{align*}
    \item $\mu\neq\nu$
    
    Due to the result from (b), one could easily realize that when both indices are not 0, $\left\{\gamma^i,\gamma^j\right\}=0$ which is $2g^{ij}I_4$.
    Consider one of the indices to be 0, then :

    \begin{align*}
        \gamma^\mu \gamma^\nu + \gamma^\nu \gamma^\mu &= \gamma^0 \gamma^i + \gamma^i \gamma^0 \\[0.15in]
                                                      &= \left[ 
                                                        \begin{pmatrix}[1.1]
                                                            1 & 0 \\
                                                            0 & -1 
                                                        \end{pmatrix}
                                                        \begin{pmatrix}[1.1]
                                                            0 & 1 \\
                                                            -1 & 0 
                                                        \end{pmatrix} + 
                                                        \begin{pmatrix}[1.1]
                                                            0 & 1 \\
                                                            -1 & 0 
                                                        \end{pmatrix}
                                                        \begin{pmatrix}[1.1]
                                                            1 & 0 \\
                                                            0 & -1 
                                                        \end{pmatrix}
                                                      \right] \otimes \sigma_i =0 = 2g^{0i} I_4 \qed
    \end{align*}
\end{enumerate}
\end{solution}

\noindent\rule{7in}{1.5pt}

%%%%%%%%%%%%%%%%%%%%%%%%%%%%%%%%%%%%%%%%%%%%%%%%%%%%%%%%%%%%%%%%%%%%%%%%%%%%%%%%%%%%%%%%%%%%%%%%%%%%%%%%%%%%%%%%%%%%%%%%%%%%%%%%%%%%%%%%

\begin{problem}{4.7}
By operating on the Dirac equation,
\begin{align*}
    \left( i\gamma^\mu \partial_\mu - m \right) \psi = 0
\end{align*}
with $\gamma^\nu\partial_\nu$ prove that the components of $\psi$ satisfy the Klein-Gordon equation,
\begin{align*}
    \left( \partial^\mu \partial_\mu +m^2 \right) \psi = 0.
\end{align*}
\end{problem}
\begin{solution}
Straightforwardly following the instructions given by the problem, 

\begin{align*}
    \left( i\gamma^\mu \partial_\mu - m \right) \psi = 0  &\implies \gamma^\nu \partial_\nu \left( i\gamma^\mu \partial_\mu - m \right) \psi = 0 \\[0.15in]
     &\implies \left[  i \gamma^\nu \partial_\nu \left( \gamma^\mu \partial_\mu  \right) - m \gamma^\nu \partial_\nu \right] \psi = 0 \\[0.15in]
     &\implies \left[  i \gamma^\nu \left( \partial_\nu  \gamma^\mu \right) \partial_\mu   +  i \gamma^\nu  \gamma^\mu \partial_\nu   \partial_\mu    - m \gamma^\nu \partial_\nu \right] \psi = 0 \\[0.15in]
     &\implies \left(  \frac{i}{2} \left\{ \gamma^\nu , \gamma^\mu \right\} \partial_\nu   \partial_\mu    - m \gamma^\nu \partial_\nu \right) \psi = 0 \\[0.15in]
     &\implies \left( i g^{\nu\mu} \partial_\nu   \partial_\mu    - m \gamma^\nu \partial_\nu \right) \psi = 0 
\end{align*}\\
For the latter term, one could utilize the Dirac equation :

\begin{align*}
    \left( i\gamma^\mu \partial_\mu - m \right) \psi = 0  &\implies \slashed{\partial}\psi = -im \psi 
\end{align*}\\
Thus the above could be tidied up as, 

\begin{align*}
    \left( i g^{\nu\mu} \partial_\nu   \partial_\mu    - m \gamma^\nu \partial_\nu \right) \psi = 0 &\implies \left( i\partial^\mu\partial_\mu - m \slashed{\partial}  \right) \psi = 0 \\[0.15in]
    &\implies \left( \partial^\mu\partial_\mu +m^2 \right) \psi = 0  \qed
\end{align*}
\end{solution}

\noindent\rule{7in}{1.5pt}

%%%%%%%%%%%%%%%%%%%%%%%%%%%%%%%%%%%%%%%%%%%%%%%%%%%%%%%%%%%%%%%%%%%%%%%%%%%%%%%%%%%%%%%%%%%%%%%%%%%%%%%%%%%%%%%%%%%%%%%%%%%%%%%%%%%%%%%%

\begin{problem}{4.8}
Show that
\begin{align*}
    \left(\gamma^\mu\right)^\dagger = \gamma^0 \gamma^\mu \gamma^0.
\end{align*}
\end{problem}
\begin{solution}
Let us seperate the cases with indices being 0 and else. 
\begin{enumerate}[label=(\alph*)]
    \item $\mu=0$
        \begin{align*}
            \gamma^{0\dagger} &= \gamma^0  \\[0.15in]
                              &= I_4 \gamma^0  = \gamma^0\gamma^0\gamma^0 
        \end{align*}
    \item $\mu=k\neq 0$
        \begin{align*}
            \gamma^{k\dagger} &= -\gamma^k  \\[0.15in]
                              &= -I_4 \gamma^k  = -\gamma^0\gamma^0\gamma^k  = \gamma^0\gamma^k\gamma^0 \qed
        \end{align*}
\end{enumerate}
\end{solution}

\noindent\rule{7in}{1.5pt}

%%%%%%%%%%%%%%%%%%%%%%%%%%%%%%%%%%%%%%%%%%%%%%%%%%%%%%%%%%%%%%%%%%%%%%%%%%%%%%%%%%%%%%%%%%%%%%%%%%%%%%%%%%%%%%%%%%%%%%%%%%%%%%%%%%%%%%%%

\begin{problem}{4.9}
Starting from
\begin{align*}
    \left( \gamma^\mu p_\mu - m  \right) u = 0,
\end{align*}
show that the corresponding equation for the adjoint spinor is
\begin{align*}
    \overbar{u} \left( \gamma^\mu p_\mu -m \right)=0.
\end{align*} 
Hence, without using the explicit form for the $u$ spinors, show that the normalisation condition $u^\dagger u = 2E$ leads to
\begin{align*}
    \overbar{u}u = 2m,
\end{align*}
and that
\begin{align*}
   \overbar{u} \gamma^\mu u = 2p^\mu.
\end{align*}
\end{problem}
\begin{solution}
Let us first derive the corresponding Dirac equation for the adjoint spinor. 

\begin{align*}
    \left( \gamma^\mu p_\mu - m  \right) u = 0 &\implies  u^\dagger \left( \gamma^\mu p_\mu - m   \right)^\dagger = 0 \\[0.15in]
    &\implies \overbar{u} \gamma^0 \left( \gamma^{\mu\dagger} p_\mu - m   \right)  = 0  \quad \text{from} \quad \overbar{u} = u^\dagger \gamma^0 \iff u^\dagger = \overbar{u}\gamma^0 \\[0.15in]
    &\implies \overbar{u} \left( \gamma^0 \gamma^{\mu\dagger} \gamma^0 p_\mu - m \gamma^0 \gamma^0 \right) = 0 \\[0.15in]
    &\implies \overbar{u} \left(   \gamma^\mu p_\mu - m \right) = 0  \quad \text{from} \quad \gamma^0\gamma^\mu\gamma^0 = \gamma^{\mu\dagger} 
\end{align*}\\
In order to obtain the other relations, let us start from evaluating $\overbar{u}\gamma^\mu u$ first.

\begin{align*}
    \overbar{u}\gamma^\mu u &= \frac{1}{m} \overbar{u} \gamma^\mu \slashed{p} u \quad \impliedby \slashed{p}u = mu  \\[0.15in]
                            &= \frac{1}{m} \overbar{u} \gamma^\mu \gamma^\nu p_\nu u = \frac{1}{m} \overbar{u} \left[ 2g^{\mu\nu}- \gamma^\nu \gamma^\mu \right] p_\nu u \\[0.15in]
                            &= \frac{1}{m} \left[ 2\overbar{u}u p^\mu - \overbar{u} \gamma^\nu \gamma^\mu  p_\nu u \right]  \\[0.15in]
                            &= \frac{1}{m} \left[ 2\overbar{u}u p^\mu - \overbar{u} \slashed{p} \gamma^\mu u \right] \quad \impliedby \overbar{u}\slashed{p} = m \overbar{u} \\[0.15in]
                            &= \frac{1}{m} \left[ 2\overbar{u}u p^\mu - m \overbar{u} \gamma^\mu u \right]   \\[0.15in]
                            &\iff \overbar{u}\gamma^\mu u = \frac{1}{m}\overbar{u}p^\mu u
\end{align*}\\
Under such relation letting $\mu=0$ gives 

\begin{align*}
    \overbar{u}\gamma^0 u = \frac{1}{m}\overbar{u}p^0 u &\implies u^\dagger \gamma^0 \gamma^0 u = \frac{E}{m} \overbar{u} u \\[0.15in]
                                                        &\implies u^\dagger u  =  \frac{E}{m} \overbar{u} u \\[0.15in]
                                                        &\implies 2E =  \frac{E}{m} \overbar{u} u \\[0.15in]
                                                        &\implies \overbar{u} u = 2m 
\end{align*}\\
Now plugging in such relation back into $\overbar{u}\gamma^\mu u $ gives,

\begin{align*}
    \overbar{u}\gamma^\mu u  = \frac{1}{m}\overbar{u} u p^\mu = 2p^\mu. \qed
\end{align*}
\end{solution}

\noindent\rule{7in}{1.5pt}

%%%%%%%%%%%%%%%%%%%%%%%%%%%%%%%%%%%%%%%%%%%%%%%%%%%%%%%%%%%%%%%%%%%%%%%%%%%%%%%%%%%%%%%%%%%%%%%%%%%%%%%%%%%%%%%%%%%%%%%%%%%%%%%%%%%%%%%%

\begin{problem}{4.10}
Demonstrate that the two relations of Equation (4.45) are consistent by showing that
\begin{align*}
    \left( \boldsymbol{\sigma} \cdot \boldp \right)^2 = \boldp^2.
\end{align*}
\end{problem}
\begin{solution}
One could easily show that

\begin{align*}
    \left(\boldsymbol{\sigma}\cdot \boldp \right)^2 &= \sigma^i\sigma^j p_i p_j \\[0.15in]
                                                    &= \frac{1}{2} \left( \sigma^i\sigma^j  +  \sigma^j\sigma^i \right) p_i p_j \impliedby \left\{ \sigma^i, \sigma^j\right\} = 2\delta^{ij} \\[0.15in]
                                                    &= \delta^{ij} p_i p_j = \boldp^2. \qed
\end{align*}\\
One could notice that not only for $\boldp$ but for any cartesian vector the above should hold. This relation leads to the equivalence of, 

\begin{align*}
    u_A = \frac{\left(\boldsymbol{\sigma}\cdot \boldp\right)}{E-m} u_B &\implies \left(\boldsymbol{\sigma}\cdot \boldp\right) u_A = \frac{\left(\boldsymbol{\sigma}\cdot \boldp\right)^2}{E-m} u_B \\[0.15in]
                                                                       &\implies \left(\boldsymbol{\sigma}\cdot \boldp\right) u_A = \frac{ \boldp^2}{E-m} u_B \\[0.15in]
                                                                       &\implies \frac{\left(\boldsymbol{\sigma}\cdot \boldp\right)}{E+m} u_A = u_B 
\end{align*}\\
which is the desired result.
\end{solution}

\noindent\rule{7in}{1.5pt}

%%%%%%%%%%%%%%%%%%%%%%%%%%%%%%%%%%%%%%%%%%%%%%%%%%%%%%%%%%%%%%%%%%%%%%%%%%%%%%%%%%%%%%%%%%%%%%%%%%%%%%%%%%%%%%%%%%%%%%%%%%%%%%%%%%%%%%%%

\begin{problem}{4.11}
Consider the $e^+e^-\to\gamma\to e^+e^-$ annihilation process in the centre-of-mass frame where the energy of the photon is $2E$.
Discuss energy and charge conservation for the two cases where:
\begin{enumerate}[label=(\alph*)]
    \item the negative energy solutions of the Dirac equation are interpreted as negative energy particles propagating backwards in time
    \item the negative energy solutions of the Dirac equation are interpreted as positive energy antiparticles propa- gating forwards in time
\end{enumerate}
\end{problem}
\begin{solution}
    %\begin{multicols}{2}
    %    \begin{enumerate}[label=(\alph*)]
    %        \item the negative energy solutions of the Dirac equation are interpreted as negative energy particles propagating backwards in time
    %        \item the negative energy solutions of the Dirac equation are interpreted as positive energy antiparticles propa- gating forwards in time
    %    \end{enumerate}
    %\end{multicols}
    \begin{multicols}{2}
    \begin{enumerate}[label=(\alph*)]
        \item the negative energy solutions of the Dirac equation are interpreted as negative energy particles propagating backwards in time
        \begin{center}
            \scalebox{1.25}{
                \begin{tikzpicture}
                    \begin{feynman}
                        
                        \vertex (g1);
                        \vertex at ($(g1) + (-3.5em,3.5em) $) (e1) {\(e^-\)};
                        \vertex at ($(g1) + (-3.5em,-3.5em) $) (e2) {\(e^-\)};
    
                        \vertex at ($(g1) + (3em,0)$) (g2) ;
                        \vertex at ($(g2) + (3.5em,3.5em) $) (m1) {\(e^-\)};
                        \vertex at ($(g2) + (3.5em,-3.5em) $) (m2) {\(e^-\)};
                        \diagram* {
                            {[edges=fermion]
                              (e1) -- (g1) -- (e2),
                              (m2) -- (g2) -- (m1)
                            },
                            (g1) -- [boson, edge label =\(\gamma^\ast\)] (g2)
                        };
    
                    \end{feynman}
                \end{tikzpicture}
            }
        \end{center} \columnbreak
        \item the negative energy solutions of the Dirac equation are interpreted as positive energy antiparticles propa- gating forwards in time
        \begin{center}
            \scalebox{1.25}{
                \begin{tikzpicture}
                    \begin{feynman}
                        
                        \vertex (g1);
                        \vertex at ($(g1) + (-3.5em,3.5em) $) (e1) {\(e^-\)};
                        \vertex at ($(g1) + (-3.5em,-3.5em) $) (e2) {\(e^+\)};
    
                        \vertex at ($(g1) + (3em,0)$) (g2) ;
                        \vertex at ($(g2) + (3.5em,3.5em) $) (m1) {\(e^-\)};
                        \vertex at ($(g2) + (3.5em,-3.5em) $) (m2) {\(e^+\)};
                        \diagram* {
                            {[edges=fermion]
                              (e1) -- (g1),  (e2) -- (g1),
                              (g2) -- (m2) , (g2) -- (m1)
                            },
                            (g1) -- [boson, edge label =\(\gamma^\ast\)] (g2)
                        };
    
                    \end{feynman}
                \end{tikzpicture}
            }
        \end{center}
    \end{enumerate}
\end{multicols}
\end{solution}

\noindent\rule{7in}{1.5pt}

%%%%%%%%%%%%%%%%%%%%%%%%%%%%%%%%%%%%%%%%%%%%%%%%%%%%%%%%%%%%%%%%%%%%%%%%%%%%%%%%%%%%%%%%%%%%%%%%%%%%%%%%%%%%%%%%%%%%%%%%%%%%%%%%%%%%%%%%

\begin{problem}{4.12}
Verify that the helicity operator
\begin{align*}
    \hat{h} = \frac{\hat{\boldsymbol{\Sigma}}\cdot\hat{\boldp}}{2p} = \frac{1}{2p}
    \begin{pmatrix}
        \boldmath{\sigma}\cdot\hat{\boldp} & 0 \\
        0 & \boldmath{\sigma}\cdot\hat{\boldp}
    \end{pmatrix}
\end{align*}
commutes with the Dirac Hamiltonian,
\begin{align*}
    \hat{H}_D = \boldsymbol{\alpha}\cdot\hat{\boldp} + \beta m
\end{align*}
\end{problem}
\begin{solution}
The commutation between $\hat{h}$ and $\hat{H}_D$ could be written as,

\begin{align*}
    [ \hat{H}_D , \hat{h} ] &= \frac{1}{2p} \left[\boldsymbol{\alpha}\cdot\hat{\boldp} + \beta m,  \hat{\boldsymbol{\Sigma}}\cdot\hat{\boldp} \right] \\[0.15in]
                            &= \frac{1}{2p}   \left[ \boldsymbol{\alpha}\cdot\hat{\boldp} , \hat{\boldsymbol{\Sigma}}\cdot\hat{\boldp}   \right] + \frac{m}{2p} \left[\beta,\hat{\boldsymbol{\Sigma}}\cdot\hat{\boldp} \right] 
\end{align*}\\
Let us first evaluate the second commutator : 

\begin{align*}
    \left[\beta,\hat{\boldsymbol{\Sigma}}\cdot\hat{\boldp} \right] &= \left[\beta,\Sigma_i \hatp_i\right] \\
    &= \cancelto{0}{\left[\beta,\Sigma_i\right]\hatp_i} + \Sigma_i \cancelto{0}{\left[ \beta,\hatp_i \right]}
\end{align*}\\
where the second term is eliminated due to the fact that matrix operation and differentiation could be commuted (one could check oneself by operating on an arbitrary spinor). The first commutator then could be again reduced into,

\begin{align*}
    \left[ \boldsymbol{\alpha}\cdot\hat{\boldp} , \hat{\boldsymbol{\Sigma}}\cdot\hat{\boldp}   \right] &= \left[ \alpha_i\hat{p}_i , \Sigma_j \hat{p}_j \right] \\[0.1in]
    &= \alpha_i \cancelto{0}{\left[ \hatp_i,\Sigma_j \right]}\hatp_j + \alpha_i \Sigma_j \cancelto{0}{\left[ \hatp_i,\hatp_j \right]} + \left[\alpha_i,\Sigma_j\right]\hatp_j\hatp_i + \Sigma_j \cancelto{0}{\left[ \alpha_i,\hatp_j \right]} \hatp_i \\[0.1in]
    &=  2i \epsilon_{ijk}\alpha_k \hatp_j \hatp_i    = 2i\epsilon_{ijk}\alpha_k \left(\delta_{ji} - \hatp_i\hatp_j\right) = 2i \epsilon_{ijk}\alpha_k \hatp_i \hatp_j \\[0.1in]
    &= 0 \qed 
\end{align*}
\end{solution}

\noindent\rule{7in}{1.5pt}

%%%%%%%%%%%%%%%%%%%%%%%%%%%%%%%%%%%%%%%%%%%%%%%%%%%%%%%%%%%%%%%%%%%%%%%%%%%%%%%%%%%%%%%%%%%%%%%%%%%%%%%%%%%%%%%%%%%%%%%%%%%%%%%%%%%%%%%%

\begin{problem}{4.13}
Show that
\begin{align*}
     \mathsf{P} u_\uparrow \left( \theta,\phi \right) = u_\downarrow \left( \pi-\theta,\pi+\phi \right)
\end{align*}
and comment on the result.
\end{problem}
\begin{solution}
One could straightforwardly write down,

\begin{align*}
    \mathsf{P}u_\uparrow\left(\theta,\phi\right) = \gamma^0 u_\uparrow\left(\theta,\phi\right) &= N \gamma^0 \begin{pmatrix}[1.5] 
        \cos \frac{\theta}{2} \\
        e^{i\phi} \sin \frac{\theta}{2} \\
        \frac{p}{E+m} \cos \frac{\theta}{2} \\
        \frac{p}{E+m} e^{i\phi} \sin \frac{\theta}{2} \\
    \end{pmatrix} = N  \begin{pmatrix}[1.5] 
        \cos \frac{\theta}{2} \\
        e^{i\phi} \sin \frac{\theta}{2} \\
        -\frac{p}{E+m} \cos \frac{\theta}{2} \\
        -\frac{p}{E+m} e^{i\phi} \sin \frac{\theta}{2} \\
    \end{pmatrix} \\[0.15in]
    &= N  \begin{pmatrix}[1.5] 
        - \sin \left(\frac{\pi}{2} - \frac{\theta}{2}\right) \\
        e^{i(\pi+\phi)} \cos \left(\frac{\pi}{2} - \frac{\theta}{2}\right) \\
         \frac{p}{E+m} \sin \left(\frac{\pi}{2} - \frac{\theta}{2}\right) \\
        -\frac{p}{E+m} e^{i(\pi+\phi)} \cos  \left(\frac{\pi}{2} - \frac{\theta}{2}\right) \\
    \end{pmatrix} \\[0.15in]
    &= u_\downarrow \left( \pi-\theta,\pi+\phi\right) \qed
\end{align*}\\
One could notice that in terms of spherical angles, the result above shows that the parity operator actually flips the momentum direction to $-\boldp$, which will result in the opposite helicity.
\end{solution}

\noindent\rule{7in}{1.5pt}

%%%%%%%%%%%%%%%%%%%%%%%%%%%%%%%%%%%%%%%%%%%%%%%%%%%%%%%%%%%%%%%%%%%%%%%%%%%%%%%%%%%%%%%%%%%%%%%%%%%%%%%%%%%%%%%%%%%%%%%%%%%%%%%%%%%%%%%%

\begin{problem}{4.14}
Under the combined operation of parity and charge conjugation $(\mathsf{CP})$ spinors transform as
\begin{align*}
    \psi \to \psi^C = \mathsf{CP} \psi = i \gamma^2\gamma^0 \psi^\ast 
\end{align*}
Show that up to an overall complex phase factor
\begin{align*}
    \mathsf{CP} u_\uparrow \left( \theta,\phi \right) = v_\downarrow  \left( \pi-\theta,\pi+\phi \right)
\end{align*}
\end{problem}
\begin{solution}

\end{solution}

\noindent\rule{7in}{1.5pt}

%%%%%%%%%%%%%%%%%%%%%%%%%%%%%%%%%%%%%%%%%%%%%%%%%%%%%%%%%%%%%%%%%%%%%%%%%%%%%%%%%%%%%%%%%%%%%%%%%%%%%%%%%%%%%%%%%%%%%%%%%%%%%%%%%%%%%%%%

\begin{problem}{4.15}
Starting from the Dirac equation, derive the identity
\begin{align*}
    \overbar{u}  (p') \gamma^\mu u ( p ) = \frac{1}{2m} \overbar{u}(p') \left(p+p'\right) u(p) + \frac{i}{m} \overbar{u}\left(p'\right) \Sigma^{\mu\nu} q_\nu u(p)
\end{align*}
where $q = p'-p$ and $\Sigma^{\mu\nu}=\frac{i}{4}\left[ \gamma^\mu,\gamma^\nu \right]$
\end{problem}
\begin{solution}
Using the Dirac equation for $u(p)$, one could write down $\overbar{u}(p')\gamma^\mu u(p)$ as,

\begin{align}
    \overbar{u}(p')\gamma^\mu u(p) &= \frac{1}{m}  \overbar{u}(p')\gamma^\mu \slashed{p} u(p)  \label{4.15.1} \\[0.15in]
                                   &= \frac{1}{m}  \overbar{u}(p')\gamma^\mu \gamma^\nu p_\nu u(p) \nonumber \\[0.15in]
                                   &= \frac{1}{m}  \overbar{u}(p')\left( 2g^{\mu\nu} - \gamma^\nu \gamma^\mu \right) p_\nu u(p) \nonumber \\[0.15in]
                                   &= \frac{2}{m}  \overbar{u}(p') p^\mu  u(p) - \frac{1}{m}  \overbar{u}(p') \slashed{p} \gamma^\mu   u(p)    \label{4.15.2}
\end{align}\\
One could do the same using the Dirac equation for the adjoint case,

\begin{align}
    \overbar{u}(p')\gamma^\mu u(p) &= \frac{1}{m}  \overbar{u}(p')  \slashed{p'} \gamma^\mu u(p)  \label{4.15.3} \\[0.15in]
                                   &= \frac{1}{m}  \overbar{u}(p')\gamma^\nu \gamma^\mu p'_\nu u(p) \nonumber \\[0.15in]
                                   &= \frac{1}{m}  \overbar{u}(p')\left( 2g^{\nu\mu} - \gamma^\mu \gamma^\nu \right) p'_\nu u(p) \nonumber \\[0.15in]
                                   &= \frac{2}{m}  \overbar{u}(p') p'^\mu  u(p) - \frac{1}{m}  \overbar{u}(p') \gamma^\mu   \slashed{p'} u(p)   \label{4.15.4}
\end{align}\\
Using the above relations one could again express  $\overbar{u}(p')\gamma^\mu u(p)$ as,

\begin{align}
    \overbar{u}(p')\gamma^\mu u(p) &= \frac{1}{2} \left\{ \frac{1}{m}  \overbar{u}(p')\gamma^\mu \slashed{p} u(p)  + \frac{1}{m}  \overbar{u}(p')  \slashed{p'} \gamma^\mu u(p) \right\} \nonumber \\[0.15in]
                                   &= \frac{1}{m}   \overbar{u}(p') (p+p')^\mu  u(p)  - \frac{1}{2m} \overbar{u}(p') \left\{ \slashed{p}\gamma^\mu + \gamma^\mu \slashed{p'} \right\}  u(p) \label{4.15.5}
\end{align}\\
Before moving on, one could again use another relation that could be derived from (\ref{4.15.1}) to (\ref{4.15.4}) as :

\begin{align}
    (\ref{4.15.1}) = (\ref{4.15.2}) &\implies \frac{1}{m}  \overbar{u}(p')\gamma^\mu \slashed{p} u(p) = \frac{2}{m}  \overbar{u}(p') p^\mu  u(p) - \frac{1}{m}  \overbar{u}(p') \slashed{p} \gamma^\mu   u(p) \nonumber \\[0.15in]
                                    &\implies 2 \overbar{u}(p') p^\mu  u(p) = \overbar{u}(p') \left\{ \gamma^\mu \slashed{p} + \slashed{p}\gamma^\mu \right\} u(p)  \label{4.15.6} \\[0.15in]
    (\ref{4.15.3}) = (\ref{4.15.4}) &\implies  \overbar{u}(p')  \slashed{p'} \gamma^\mu u(p)  = \frac{2}{m}  \overbar{u}(p') p'^\mu  u(p) - \frac{1}{m}  \overbar{u}(p') \gamma^\mu   \slashed{p'} u(p)    \nonumber \\[0.15in]
                                    &\implies 2 \overbar{u}(p') p'^\mu  u(p) = \overbar{u}(p') \left\{ \gamma^\mu \slashed{p'} + \slashed{p'}\gamma^\mu \right\} u(p)   \label{4.15.7}
\end{align}\\
Adding up both (\ref{4.15.6}) and (\ref{4.15.7}) gives, 

\begin{align}
    \overbar{u}(p') (p+p')^\mu  u(p) = \frac{1}{2} \overbar{u}(p') \left\{ \gamma^\mu \slashed{p'} + \slashed{p'}\gamma^\mu + \gamma^\mu \slashed{p} + \slashed{p}\gamma^\mu \right\} u(p) \label{4.15.8}
\end{align}\\
Plugging (\ref{4.15.8}) into (\ref{4.15.5}) but splitting the first term into half gives, 

\begin{align}
    \overbar{u}(p')\gamma^\mu u(p) &=   \frac{1}{2m}   \overbar{u}(p') (p+p')^\mu  u(p) + \frac{1}{2m}   \overbar{u}(p') (p+p')^\mu  u(p)  - \frac{1}{2m} \overbar{u}(p') \left\{ \slashed{p}\gamma^\mu + \gamma^\mu \slashed{p'} \right\}  u(p)  \nonumber \\[0.15in]
                                   &=   \frac{1}{2m}   \overbar{u}(p') (p+p')^\mu  u(p) + \frac{1}{4m} \overbar{u}(p') \left\{ \gamma^\mu \slashed{p'} + \slashed{p'}\gamma^\mu + \gamma^\mu \slashed{p} + \slashed{p}\gamma^\mu \right\} u(p) - \frac{1}{2m} \overbar{u}(p') \left\{ \slashed{p}\gamma^\mu + \gamma^\mu \slashed{p'} \right\}  u(p)  \nonumber \\[0.15in]
                                   &=   \frac{1}{2m}   \overbar{u}(p') (p+p')^\mu  u(p) - \frac{1}{4m} \overbar{u}(p') \left\{  \gamma^\mu \slashed{p'} - \slashed{p'}\gamma^\mu - \gamma^\mu \slashed{p} + \slashed{p}\gamma^\mu \right\} u(p)  \nonumber \\[0.15in]
                                   &=   \frac{1}{2m}   \overbar{u}(p') (p+p')^\mu  u(p) - \frac{1}{4m} \overbar{u}(p') \left\{ \gamma^\mu (\slashed{p'}-\slashed{p}) - (\slashed{p'}-\slashed{p}) \gamma^\mu \right\} u(p)  \nonumber \\[0.15in]
                                   &=   \frac{1}{2m}   \overbar{u}(p') (p+p')^\mu  u(p) - \frac{1}{4m} \overbar{u}(p') \left[  \gamma^\mu , \slashed{q} \right] u(p)  \nonumber \\[0.15in]
                                   &=   \frac{1}{2m}   \overbar{u}(p') (p+p')^\mu  u(p) - \frac{1}{4m} \overbar{u}(p') \left[  \gamma^\mu , \gamma^\nu \right] q_\nu u(p)  \nonumber \\[0.15in]
                                   &=   \frac{1}{2m}   \overbar{u}(p') (p+p')^\mu  u(p) + \frac{i}{m} \overbar{u}\left(p'\right) \Sigma^{\mu\nu} q_\nu u(p) \qed \nonumber
\end{align}
\end{solution}

\noindent\rule{7in}{1.5pt}

%%%%%%%%%%%%%%%%%%%%%%%%%%%%%%%%%%%%%%%%%%%%%%%%%%%%%%%%%%%%%%%%%%%%%%%%%%%%%%%%%%%%%%%%%%%%%%%%%%%%%%%%%%%%%%%%%%%%%%%%%%%%%%%%%%%%%%%%

