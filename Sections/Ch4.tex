
\noindent\rule{7in}{2.8pt}
\section{The Dirac Equation}
    
\begin{problem}{4.1}
Show that
\begin{align*}
    \left[ \hat{\boldp}^2,\hat{\mathbf{r}}\times \hat{\boldp}\right]=0,
\end{align*}
and hence the Hamiltonian of the free-particle Schrödinger equation commutes with the angular momentum operator.
\end{problem}
\begin{solution}
One could expand the given commutator as, 

\begin{align*}
    \left[ \hat{\boldp}^2 , \hat{\mathbf{r}} \times \hat{\boldp} \right] &=  \left[ \hat{\boldp}_a\hat{\boldp}_a,\epsilon_{abc}r_c \hat{\boldp}_b \hat{\mathbf{c}} \right]\\[0.15in]
        &= \epsilon_{abc}r_c \left[ \hat{\boldp}_a\hat{\boldp}_a,\hat{\boldp}_b \hat{\mathbf{c}} \right]\\[0.15in]
        &= \epsilon_{abc}r_c \left\{ \hat{\boldp}_a \left[ \hat{\boldp}_a,\hat{\boldp}_b \hat{\mathbf{c}} \right] + \left[ \hat{\boldp}_a,\hat{\boldp}_b \hat{\mathbf{c}} \right] \hat{\boldp}_a  \right\} \\[0.15in]
        &= \epsilon_{abc}r_c \left\{ \hat{\boldp}_a \left[ \hat{\boldp}_a,\hat{\boldp}_b \hat{\mathbf{c}} \right] + \left[ \hat{\boldp}_a,\hat{\boldp}_b \hat{\mathbf{c}} \right] \hat{\boldp}_a  \right\} \quad \impliedby  \left[ \hat{\boldp}_a,\hat{\boldp}_b \hat{\mathbf{c}} \right] = \delta_{ab} \hat{\mathbf{c}} - i \delta_{ac} \hat{\boldp}_b
\end{align*}
\end{solution}

\noindent\rule{7in}{1.5pt}

%%%%%%%%%%%%%%%%%%%%%%%%%%%%%%%%%%%%%%%%%%%%%%%%%%%%%%%%%%%%%%%%%%%%%%%%%%%%%%%%%%%%%%%%%%%%%%%%%%%%%%%%%%%%%%%%%%%%%%%%%%%%%%%%%%%%%%%%

\begin{problem}{4.2}
Show that $u_1$ and $u_2$ are orthogonal, i.e. $u_1^\dagger u_2 =0$.
\end{problem}
\begin{solution}
Let us first denote 

\begin{align*}
    u_\uparrow = \begin{pmatrix}
        1 \\
        0
    \end{pmatrix} \quad \text{and} \quad 
    u_\downarrow = \begin{pmatrix}
        0 \\
        1
    \end{pmatrix} .
\end{align*}\\
One should note that $u_\uparrow^{\dagger}u_\downarrow = 0$. $u_1,u_2$ could also be expressed in terms of 

\begin{align*}
    u_1 = \begin{pmatrix}[1.25]
        u_\uparrow \\
        \frac{\boldsymbol{\sigma}\cdot\boldp}{E+m} u_\uparrow
    \end{pmatrix} \quad \text{and} \quad 
    u_2 = \begin{pmatrix}[1.25]
        u_\downarrow \\
        \frac{\boldsymbol{\sigma}\cdot\boldp}{E+m} u_\downarrow
    \end{pmatrix}.
\end{align*}\\
Now $u_1^\dagger u_2$ could be written as, 

\begin{align*}
    u_1^\dagger u_2 &=  \begin{pmatrix}[1.25]
        u_\uparrow \\
        \frac{\boldsymbol{\sigma}\cdot\boldp}{E+m} u_\uparrow
    \end{pmatrix}^\dagger
    \begin{pmatrix}[1.25]
        u_\downarrow \\
        \frac{\boldsymbol{\sigma}\cdot\boldp}{E+m} u_\downarrow
    \end{pmatrix} \\[0.15in]
    &=  \begin{pmatrix}[1.2]
        u_\uparrow^\dagger & \frac{1}{E+m} \left(\left(\boldsymbol{\sigma}\cdot\boldp\right) u_\uparrow\right)^\dagger
    \end{pmatrix}
    \begin{pmatrix}[1.25]
        u_\downarrow \\
        \frac{\boldsymbol{\sigma}\cdot\boldp}{E+m} u_\downarrow \\
    \end{pmatrix} \\[0.15in]
    &=  u_\uparrow^\dagger u_\downarrow + \frac{1}{\left(E+m\right)^2} \left(\left(\boldsymbol{\sigma}\cdot\boldp\right) u_\uparrow\right)^\dagger \left(\left(\boldsymbol{\sigma}\cdot\boldp\right) u_\downarrow\right) \\[0.15in]
    &=  u_\uparrow^\dagger u_\downarrow + \frac{1}{\left(E+m\right)^2}  u_\uparrow^\dagger \left(\boldsymbol{\sigma}\cdot\boldp\right)^\dagger \left(\boldsymbol{\sigma}\cdot\boldp\right) u_\downarrow  
\end{align*}
One could use the fact that,

\begin{align*}
    \left(\boldsymbol{\sigma}\cdot\boldp\right)^\dagger \left(\boldsymbol{\sigma}\cdot\boldp\right) =  \left(\boldsymbol{\sigma}\cdot\boldp\right)^2 = \begin{pmatrix}[1.5]
        p_x^2 + p_y^2 + p_z^2 & 0 \\
        0 & p_x^2 + p_y^2 + p_z^2 \\
    \end{pmatrix} = \left(E^2-m^2\right)I_2
\end{align*}\\
Thus it could be tidied up as, 

\begin{align*}
    u_1^\dagger u_2 &=  u_\uparrow^\dagger u_\downarrow + \frac{1}{\left(E+m\right)^2}  u_\uparrow^\dagger \left(\boldsymbol{\sigma}\cdot\boldp\right)^\dagger \left(\boldsymbol{\sigma}\cdot\boldp\right) u_\downarrow  \\[0.15in]
                    &= \left[ 1 + \frac{E^2-m^2}{\left(E+m\right)^2}  \right]  u_\uparrow^\dagger u_\downarrow = 0 \qed
\end{align*}
\end{solution}

\noindent\rule{7in}{1.5pt}

%%%%%%%%%%%%%%%%%%%%%%%%%%%%%%%%%%%%%%%%%%%%%%%%%%%%%%%%%%%%%%%%%%%%%%%%%%%%%%%%%%%%%%%%%%%%%%%%%%%%%%%%%%%%%%%%%%%%%%%%%%%%%%%%%%%%%%%%

\begin{problem}{4.3}
Verify the statement that the Einstein energy-momentum relationship is recovered if any of the four Dirac spinors of (4.48)
are subtitutes into the Dirac equation written in terms of momentum, $\left( \gamma^\mu p_\mu-m \right)u=0$.
\end{problem}
\begin{solution}
Let us choose $u_1$ to plug in the Dirac equation. Then it could be expressed as,

\begin{align*}
    \left(\slashed{p}-m\right)u_1 = 0 &\implies \begin{pmatrix}[1.2]
        \left(E-m\right)I_2 & -\sigp \\
        \sigp & -\left(E+m\right) I_2 
    \end{pmatrix}
    \begin{pmatrix}[1.2]
        E+m \\
        0 \\
        p_z \\
        p_x + ip_y 
    \end{pmatrix} = 0  \\[0.15in]
    &\implies \begin{pmatrix}[1.2]
        E^2-m^2 \\
        0
    \end{pmatrix} + \sigp\begin{pmatrix}[1.2]
        E+m-p_z \\
        -p_x -ip_y
    \end{pmatrix} - \left(E+m\right)\begin{pmatrix}[1.2]
        p_z \\
        p_x+ip_y
    \end{pmatrix} = 0 \\[0.15in]
    \text{[first row]} &\implies \left(E^2-m^2\right) + \left(E+m\right)p_z - \left(p_x^2+p_y^2+p_z^2\right) - \left(E+m\right) p_z = 0 \\[0.15in] 
    &\implies E^2 = p_x^2 +p_y^2 + p_z^2 +m^2  \qed
\end{align*}
\end{solution}

\noindent\rule{7in}{1.5pt}

%%%%%%%%%%%%%%%%%%%%%%%%%%%%%%%%%%%%%%%%%%%%%%%%%%%%%%%%%%%%%%%%%%%%%%%%%%%%%%%%%%%%%%%%%%%%%%%%%%%%%%%%%%%%%%%%%%%%%%%%%%%%%%%%%%%%%%%%

\begin{problem}{4.4}
For a particle with four-momentum $p^\mu = (E,\boldp)$, the general solution to the free-particle Dirac equation can be written
\begin{align*}
    \psi(p) = \left[ au_1(p)+bu_2(p) \right] e^{i\left( \boldp\cdot\boldx - Et \right)}
\end{align*}
Using the explicit forms for $u_1$ and $u_2$, show that the four-vector current $j^\mu=(\rho,\mathbf{j})$ is given by
\begin{align*}
    j^\mu = 2 p^\mu
\end{align*}
Furthermore, show that the resulting probability density and probability current are consistent with a particle
moving with velocity $\beta = p /E$.
\end{problem}
\begin{solution}

\end{solution}

\noindent\rule{7in}{1.5pt}

%%%%%%%%%%%%%%%%%%%%%%%%%%%%%%%%%%%%%%%%%%%%%%%%%%%%%%%%%%%%%%%%%%%%%%%%%%%%%%%%%%%%%%%%%%%%%%%%%%%%%%%%%%%%%%%%%%%%%%%%%%%%%%%%%%%%%%%%

\begin{problem}{4.5}
Writing the four-component spinor $u$ in terms of two two-component vectors
\begin{align*}
    u = \begin{pmatrix}
        u_A \\
        u_B
    \end{pmatrix},
\end{align*}
show that in the non-relativistic limit, where $\beta \cong v/c \ll 1$, the components of $u_B$ are smaller than those of $u_A$ by a factor $v/c$.
\end{problem}
\begin{solution}

\end{solution}

\noindent\rule{7in}{1.5pt}

%%%%%%%%%%%%%%%%%%%%%%%%%%%%%%%%%%%%%%%%%%%%%%%%%%%%%%%%%%%%%%%%%%%%%%%%%%%%%%%%%%%%%%%%%%%%%%%%%%%%%%%%%%%%%%%%%%%%%%%%%%%%%%%%%%%%%%%%

\begin{problem}{4.6}
By considering the three cases $\mu=\nu=0$, $\mu=\nu\neq 0$ and $\mu \neq \nu$ show that 
\begin{align*}
    \gamma^\mu \gamma^\nu + \gamma^\nu \gamma^\mu = 2 g^{\mu\nu}.
\end{align*}
\end{problem}
\begin{solution}

\end{solution}

\noindent\rule{7in}{1.5pt}

%%%%%%%%%%%%%%%%%%%%%%%%%%%%%%%%%%%%%%%%%%%%%%%%%%%%%%%%%%%%%%%%%%%%%%%%%%%%%%%%%%%%%%%%%%%%%%%%%%%%%%%%%%%%%%%%%%%%%%%%%%%%%%%%%%%%%%%%

\begin{problem}{4.7}
By operating on the Dirac equation,
\begin{align*}
    \left( i\gamma^\mu \partial_\mu - m \right) \psi = 0
\end{align*}
with $\gamma^\nu\partial_\nu$ prove that the components of $\psi$ satisfy the Klein-Gordon equation,
\begin{align*}
    \left( \partial^\mu \partial_\mu +m^2 \right) \psi = 0.
\end{align*}
\end{problem}
\begin{solution}

\end{solution}

\noindent\rule{7in}{1.5pt}

%%%%%%%%%%%%%%%%%%%%%%%%%%%%%%%%%%%%%%%%%%%%%%%%%%%%%%%%%%%%%%%%%%%%%%%%%%%%%%%%%%%%%%%%%%%%%%%%%%%%%%%%%%%%%%%%%%%%%%%%%%%%%%%%%%%%%%%%

\begin{problem}{4.8}
Show that
\begin{align*}
    \left(\gamma^\mu\right)^\dagger = \gamma^0 \gamma^\mu \gamma^0.
\end{align*}
\end{problem}
\begin{solution}

\end{solution}

\noindent\rule{7in}{1.5pt}

%%%%%%%%%%%%%%%%%%%%%%%%%%%%%%%%%%%%%%%%%%%%%%%%%%%%%%%%%%%%%%%%%%%%%%%%%%%%%%%%%%%%%%%%%%%%%%%%%%%%%%%%%%%%%%%%%%%%%%%%%%%%%%%%%%%%%%%%

\begin{problem}{4.9}
Starting from
\begin{align*}
    \left( \gamma^\mu p_\mu - m  \right) u = 0,
\end{align*}
show that the corresponding equation for the adjoint spinor is
\begin{align*}
    \overbar{u} \left( \gamma^\mu p_\mu -m \right)=0.
\end{align*} 
Hence, without using the explicit form for the $u$ spinors, show that the normalisation condition $u^\dagger u = 2E$ leads to1
\begin{align*}
    u\overbar{u} = 2m,
\end{align*}
and that
\begin{align*}
   \overbar{u} \gamma^\mu u = 2p^\mu.
\end{align*}
\end{problem}
\begin{solution}

\end{solution}

\noindent\rule{7in}{1.5pt}

%%%%%%%%%%%%%%%%%%%%%%%%%%%%%%%%%%%%%%%%%%%%%%%%%%%%%%%%%%%%%%%%%%%%%%%%%%%%%%%%%%%%%%%%%%%%%%%%%%%%%%%%%%%%%%%%%%%%%%%%%%%%%%%%%%%%%%%%

\begin{problem}{4.10}
Demonstrate that the two relations of Equation (4.45) are consistnet by showing that
\begin{align*}
    \left( \boldsymbol{\sigma} \cdot \boldp \right)^2 = \boldp^2.
\end{align*}
\end{problem}
\begin{solution}

\end{solution}

\noindent\rule{7in}{1.5pt}

%%%%%%%%%%%%%%%%%%%%%%%%%%%%%%%%%%%%%%%%%%%%%%%%%%%%%%%%%%%%%%%%%%%%%%%%%%%%%%%%%%%%%%%%%%%%%%%%%%%%%%%%%%%%%%%%%%%%%%%%%%%%%%%%%%%%%%%%

\begin{problem}{4.11}
Consider the $e^+e^-\to\gamma\to e^+e^-$ annihilation process in the centre-of-mass frame where the energy of the photon is $2E$.
Discuss energy and charge conservation for the two cases where:
\begin{enumerate}[label=(\alph*)]
    \item the negative energy solutions of the Dirac equation are interpreted as negative energy particles propagating backwards in time
    \item the negative energy solutions of the Dirac equation are interpreted as positive energy antiparticles propa- gating forwards in time
\end{enumerate}
\end{problem}
\begin{solution}

\end{solution}

\noindent\rule{7in}{1.5pt}

%%%%%%%%%%%%%%%%%%%%%%%%%%%%%%%%%%%%%%%%%%%%%%%%%%%%%%%%%%%%%%%%%%%%%%%%%%%%%%%%%%%%%%%%%%%%%%%%%%%%%%%%%%%%%%%%%%%%%%%%%%%%%%%%%%%%%%%%

\begin{problem}{4.12}
Verify that the helicity operator
\begin{align*}
    \hat{h} = \frac{\hat{\boldsymbol{\Sigma}}\cdot\hat{\boldp}}{2p} = \frac{1}{2p}
    \begin{pmatrix}
        \boldmath{\sigma}\cdot\hat{\boldp} & 0 \\
        0 & \boldmath{\sigma}\cdot\hat{\boldp}
    \end{pmatrix}
\end{align*}
commutes with the Dirac Hamiltonian,
\begin{align*}
    \hat{H}_D = \boldsymbol{\alpha}\cdot\hat{\boldp} + \beta m
\end{align*}
\end{problem}
\begin{solution}

\end{solution}

\noindent\rule{7in}{1.5pt}

%%%%%%%%%%%%%%%%%%%%%%%%%%%%%%%%%%%%%%%%%%%%%%%%%%%%%%%%%%%%%%%%%%%%%%%%%%%%%%%%%%%%%%%%%%%%%%%%%%%%%%%%%%%%%%%%%%%%%%%%%%%%%%%%%%%%%%%%

\begin{problem}{4.13}
Show that
\begin{align*}
    \hat{p}_{u_\uparrow} \left( \theta,\phi \right) = u_\downarrow \left( \pi-\theta,\pi+\phi \right)
\end{align*}
and comment on the result.
\end{problem}
\begin{solution}

\end{solution}

\noindent\rule{7in}{1.5pt}

%%%%%%%%%%%%%%%%%%%%%%%%%%%%%%%%%%%%%%%%%%%%%%%%%%%%%%%%%%%%%%%%%%%%%%%%%%%%%%%%%%%%%%%%%%%%%%%%%%%%%%%%%%%%%%%%%%%%%%%%%%%%%%%%%%%%%%%%

\begin{problem}{4.14}
Under the combined operation of parity and charge conjugation $(\hat{C}\hat{P})$ spinors transform as
\begin{align*}
    \psi \to \psi^C = \hat{C}\hat{P} \psi = i \gamma^2\gamma^0 \psi^\ast 
\end{align*}
Show that up to an overall complex phase factor
\begin{align*}
    \hat{C}\hat{p}_{u_\uparrow} \left( \theta,\phi \right) = v_\downarrow  \left( \pi-\theta,\pi+\phi \right)
\end{align*}
\end{problem}
\begin{solution}

\end{solution}

\noindent\rule{7in}{1.5pt}

%%%%%%%%%%%%%%%%%%%%%%%%%%%%%%%%%%%%%%%%%%%%%%%%%%%%%%%%%%%%%%%%%%%%%%%%%%%%%%%%%%%%%%%%%%%%%%%%%%%%%%%%%%%%%%%%%%%%%%%%%%%%%%%%%%%%%%%%

\begin{problem}{4.15}
Starting from the Dirac equation, derive the identity
\begin{align*}
    \overbar{u}  (p') \gamma^\mu u ( p ) = \frac{1}{2m} \overbar{u}(p') \left(p+p'\right) u(p) + \frac{i}{m} \overbar{u}\left(p'\right) \Sigma^{\mu\nu} q_\nu u(p)
\end{align*}
where $q = p'-p$ and $\Sigma^{\mu\nu}=\frac{i}{4}\left[ \gamma^\mu,\gamma^\nu \right]$
\end{problem}
\begin{solution}

\end{solution}

\noindent\rule{7in}{1.5pt}

%%%%%%%%%%%%%%%%%%%%%%%%%%%%%%%%%%%%%%%%%%%%%%%%%%%%%%%%%%%%%%%%%%%%%%%%%%%%%%%%%%%%%%%%%%%%%%%%%%%%%%%%%%%%%%%%%%%%%%%%%%%%%%%%%%%%%%%%

