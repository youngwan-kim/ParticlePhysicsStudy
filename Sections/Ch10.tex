
\noindent\rule{7in}{2.8pt}
\section{Quantum Chromodynamics}
    
\begin{problem}{10.1}
By considering the symmetry of the wavefunction, explain why the existence of the $\Omega^-(sss)$ $L = 0$ baryon provides evidence for a degree of freedom in addition to space $\times$ spin $\times$ flavour.
\end{problem}
\begin{solution}

\end{solution}

\noindent\rule{7in}{1.5pt}

%%%%%%%%%%%%%%%%%%%%%%%%%%%%%%%%%%%%%%%%%%%%%%%%%%%%%%%%%%%%%%%%%%%%%%%%%%%%%%%%%%%%%%%%%%%%%%%%%%%%%%%%%%%%%%%%%%%%%%%%%%%%%%%%%%%%%%%%

\begin{problem}{10.2}
From the expression for the running of $\als$ with $N_f = 3$, determine the value of $q^2$ at which $\als$ appears to become infinite. Comment on this result.
\end{problem}
\begin{solution}

\end{solution}

\noindent\rule{7in}{1.5pt}

%%%%%%%%%%%%%%%%%%%%%%%%%%%%%%%%%%%%%%%%%%%%%%%%%%%%%%%%%%%%%%%%%%%%%%%%%%%%%%%%%%%%%%%%%%%%%%%%%%%%%%%%%%%%%%%%%%%%%%%%%%%%%%%%%%%%%%%%

\begin{problem}{10.3}
Find the overall “colour factor” for $qq \to qq$ if QCD corresponded to a SU(2) colour symmetry.
\end{problem}
\begin{solution}

\end{solution}

\noindent\rule{7in}{1.5pt}

%%%%%%%%%%%%%%%%%%%%%%%%%%%%%%%%%%%%%%%%%%%%%%%%%%%%%%%%%%%%%%%%%%%%%%%%%%%%%%%%%%%%%%%%%%%%%%%%%%%%%%%%%%%%%%%%%%%%%%%%%%%%%%%%%%%%%%%%

\begin{problem}{10.4}
Calculate the non-relativistic QCD potential between quarks $q_1$ and $q_2$ in a $q_1q_2q_3$ baryon with colour wavefunction

\begin{align*}
    \psi = \frac{1}{\sqrt{6}} \left(  rgb-grb+gbr-bgr  +brg - rbg\right)
\end{align*}\\
\end{problem}
\begin{solution}

\end{solution}

\noindent\rule{7in}{1.5pt}

%%%%%%%%%%%%%%%%%%%%%%%%%%%%%%%%%%%%%%%%%%%%%%%%%%%%%%%%%%%%%%%%%%%%%%%%%%%%%%%%%%%%%%%%%%%%%%%%%%%%%%%%%%%%%%%%%%%%%%%%%%%%%%%%%%%%%%%%

\begin{problem}{10.5}
Draw the lowest-order QCD Feynman diagrams for the process $p\overbar{p}\to$ two-jets + X, where X represents the remnants of the colliding hadrons.
\end{problem}
\begin{solution}

\end{solution}

\noindent\rule{7in}{1.5pt}

%%%%%%%%%%%%%%%%%%%%%%%%%%%%%%%%%%%%%%%%%%%%%%%%%%%%%%%%%%%%%%%%%%%%%%%%%%%%%%%%%%%%%%%%%%%%%%%%%%%%%%%%%%%%%%%%%%%%%%%%%%%%%%%%%%%%%%%%

\begin{problem}{10.6}
The observed events in the process $pp\to$ two-jets at the LHC can be described in terms of the jet $\pt$ and the jet rapidities $y_3$ and $y_4$.

\begin{enumerate}[label=(\alph*)]
    \item Assuming that the jets are massless, $E^2 = \pt^2 + p_z^2 $, show that the four-momenta of the final-state jets can be written as
    
        \begin{align*}
            p_3 &= \left( \pt \cosh y_3,+ \pt \sin \phi ,+ \pt \cos \phi , \pt \sinh y_3  \right) \\
            p_4 &= \left( \pt \cosh y_4,- \pt \sin \phi ,- \pt \cos \phi , \pt \sinh y_4  \right) 
        \end{align*}\\
    \item By writing the four-momenta of the colliding partons in a pp collision as 
    
        \begin{align*}
            p_1 = \frac{\sqrt{s}}{2} \left( x_1,0,0,x_1 \right) \andtxt p_2 = \frac{\sqrt{s}}{2} \left( x_2,0,0,-x_1 \right)
        \end{align*}\\
        show that conservation of energy and momentum implies

        \begin{align*}
            x_1 = \frac{\pt}{\sqrt{s}}\left( e^{+y_3} + e^{+y_4} \right) \andtxt x_2 = \frac{\pt}{\sqrt{s}}\left( e^{-y_3} + e^{-y_4} \right)
        \end{align*}\\
    \item Hence show that
    
        \begin{align*}
            Q^2 = \pt^2 \left( 1 + e^{y_4-y_3} \right).
        \end{align*}\\
\end{enumerate}
\end{problem}
\begin{solution}

\end{solution}

\noindent\rule{7in}{1.5pt}

%%%%%%%%%%%%%%%%%%%%%%%%%%%%%%%%%%%%%%%%%%%%%%%%%%%%%%%%%%%%%%%%%%%%%%%%%%%%%%%%%%%%%%%%%%%%%%%%%%%%%%%%%%%%%%%%%%%%%%%%%%%%%%%%%%%%%%%%

\begin{problem}{10.7}
Using the results of the previous question show that the Jacobian

    \begin{align*}
        \frac{\partial \left( y_3,y_4,\pt^2 \right)}{\partial \left( x_1, x_2, q^2 \right)} = \frac{1}{x_1x_2}.
    \end{align*}\\
\end{problem}
\begin{solution}
Using the expressions derived in the previous question, one could simply calculate the Jacobian as,

    \begin{align*}
        \frac{\partial \left( y_3,y_4,\pt^2 \right)}{\partial \left( x_1, x_2, q^2 \right)} = 
        \begin{vmatrix}
            \pder{y_3}{x_1} & \pder{y_3}{x_2} & \pder{y_3}{q^2}\\ 
            \pder{y_4}{x_1} & \pder{y_4}{x_2} & \pder{y_4}{q^2}\\ 
            \pder{\pt^2}{x_1} & \pder{\pt^2}{x_2} & \pder{\pt^2}{q^2}
       \end{vmatrix}
    \end{align*}
\end{solution}

\noindent\rule{7in}{1.5pt}

%%%%%%%%%%%%%%%%%%%%%%%%%%%%%%%%%%%%%%%%%%%%%%%%%%%%%%%%%%%%%%%%%%%%%%%%%%%%%%%%%%%%%%%%%%%%%%%%%%%%%%%%%%%%%%%%%%%%%%%%%%%%%%%%%%%%%%%%

\begin{problem}{10.8}
The total cross section for the Drell-Yan process $p\overbar{p}\to \mu^+\mu^-X$ was shown to be

\begin{align*}
    \sigma_\text{DY} = \frac{4\pi \alpha^2}{81s} \int_0^1 \int_0^1 \frac{1}{x_1x_2} \left[ 4u(x_1)u(x_2) + 4 \overbar{u}(x_1)\overbar{u}(x_2) + d(x_1) d(x_2) + \overbar{d}(x_1)\overbar{d}(x_2)  \right] \dif x_1 \dif x_2
\end{align*}\\
\begin{enumerate}[label=(\alph*)]
    \item Express this cross section in terms of the valence quark PDFs and a single PDF for the sea contribution, where $S(x)=\overbar{u}(x) = \overbar{d}(x)$.
    \item Obtain the corresponding expression for $pp\to\mu^+\mu^-X$.
    \item Sketch the region in the $ x_1-x_2$ plane corresponding $S_{q\overbar{q}}>s/4$. Comment on the expected ratio of the Drell-Yan cross sections in $pp$ and $p\overbar{p}$ collisions (at the same centre-of-mass energy) for the two cases: (i) $\hat{s} \ll s$ and (ii) $\hat{s} > s/4$, where $\hat{s}$ is the centre-of-mass energy of the colliding partons.
\end{enumerate}
\end{problem}
\begin{solution}

\end{solution}

\noindent\rule{7in}{1.5pt}

%%%%%%%%%%%%%%%%%%%%%%%%%%%%%%%%%%%%%%%%%%%%%%%%%%%%%%%%%%%%%%%%%%%%%%%%%%%%%%%%%%%%%%%%%%%%%%%%%%%%%%%%%%%%%%%%%%%%%%%%%%%%%%%%%%%%%%%%

\begin{problem}{10.9}
Drell-Yan production of $\mu^+\mu^-$ pairs with an invariant mass $Q^2$ has been studied in $\pi^\pm$ interactions with carbon (which has equal numbers of protons and neutrons). Explain why the ratio

\begin{align*}
    \frac{\sigma\left( \pi^+ \text{C} \to \mu^+\mu^-X \right)}{\sigma\left( \pi^- \text{C} \to \mu^+\mu^-X \right)}
\end{align*}\\
tends to unity for small $Q^2$ and tends to $\frac{1}{4}$ as $Q^2$ approaches $s$.
\end{problem}
\begin{solution}

\end{solution}

\noindent\rule{7in}{1.5pt}

%%%%%%%%%%%%%%%%%%%%%%%%%%%%%%%%%%%%%%%%%%%%%%%%%%%%%%%%%%%%%%%%%%%%%%%%%%%%%%%%%%%%%%%%%%%%%%%%%%%%%%%%%%%%%%%%%%%%%%%%%%%%%%%%%%%%%%%%
