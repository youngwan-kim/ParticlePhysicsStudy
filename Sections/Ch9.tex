
\noindent\rule{7in}{2.8pt}
\section{Symmetries and the Quark Model}
    
\begin{problem}{9.1}
By writing down the general term in the binomial expansion of

\begin{align*}
    \left( 1+ i \frac{1}{n} \boldsymbol{\alpha} \cdot \hat{\boldsymbol{G}}\right)^n,
\end{align*}\\
show that

\begin{align*}
    \hat{U}(\boldsymbol{\alpha}) = \lim_{n\to\infty}   \left( 1+ i \frac{1}{n} \boldsymbol{\alpha} \cdot \hat{\boldsymbol{G}}\right)^n = \exp \left( i   \boldsymbol{\alpha} \cdot \hat{\boldsymbol{G}} \right).
\end{align*}\\
\end{problem}
\begin{solution}

\end{solution}

\noindent\rule{7in}{1.5pt}

%%%%%%%%%%%%%%%%%%%%%%%%%%%%%%%%%%%%%%%%%%%%%%%%%%%%%%%%%%%%%%%%%%%%%%%%%%%%%%%%%%%%%%%%%%%%%%%%%%%%%%%%%%%%%%%%%%%%%%%%%%%%%%%%%%%%%%%%

\begin{problem}{9.2}
    For an infinitesimal rotation about the z-axis through an angle $\epsilon$ show that

    \begin{align*}
        \hat{U} = 1-i\epsilon \hat{J}_z,
    \end{align*}\\
    where $\hat{J}_z$ is the angular momentum operator $\hat{J}_z = x \hatp_y - y\hatp_x $. 
\end{problem}
\begin{solution}

\end{solution}

\noindent\rule{7in}{1.5pt}

%%%%%%%%%%%%%%%%%%%%%%%%%%%%%%%%%%%%%%%%%%%%%%%%%%%%%%%%%%%%%%%%%%%%%%%%%%%%%%%%%%%%%%%%%%%%%%%%%%%%%%%%%%%%%%%%%%%%%%%%%%%%%%%%%%%%%%%%

\begin{problem}{9.3}
By considering the isospin states, show that the rates for the following strong interaction decays occur in the ratios

\begin{align*}
    \Gamma \left( \Delta^- \to \pi^- n \right) \spc : \spc   &\Gamma \left( \Delta^0 \to \pi^- p \right) \spc  : \spc   \Gamma \left( \Delta^0 \to \pi^0 n \right) \spc : \spc   \Gamma \left( \Delta^+ \to \pi^+ n \right) \spc : \\
    &\Gamma \left( \Delta^+ \to \pi^0 p \right) \spc : \spc    \Gamma \left( \Delta^{++} \to \pi^+ p \right) \spc =  \spc 3:1:2:1:2:3
\end{align*}\\
\end{problem}
\begin{solution}

\end{solution}

\noindent\rule{7in}{1.5pt}

%%%%%%%%%%%%%%%%%%%%%%%%%%%%%%%%%%%%%%%%%%%%%%%%%%%%%%%%%%%%%%%%%%%%%%%%%%%%%%%%%%%%%%%%%%%%%%%%%%%%%%%%%%%%%%%%%%%%%%%%%%%%%%%%%%%%%%%%

\begin{problem}{9.4}
If quarks and antiquarks were spin-zero particles, what would be the multiplicity of the $L = 0$ multiplet(s). Remember that the overall wavefunction for bosons must be symmetric under particle exchange.
\end{problem}
\begin{solution}

\end{solution}

\noindent\rule{7in}{1.5pt}

%%%%%%%%%%%%%%%%%%%%%%%%%%%%%%%%%%%%%%%%%%%%%%%%%%%%%%%%%%%%%%%%%%%%%%%%%%%%%%%%%%%%%%%%%%%%%%%%%%%%%%%%%%%%%%%%%%%%%%%%%%%%%%%%%%%%%%%%

\begin{problem}{9.5}
The neutral vector mesons can decay leptonically through a virtual photon, for example by $V(q\overbar{q}) \to \gamma \to e^+e^-$. The matrix element for this decay is proportional to $\braket{\psi|\hat{Q}_q|\psi}$ where $\psi$ is the meson flavour wavefunction and $\hat{Q}_q$ is an operator that is proportional to the quark charge. Neglecting the relatively small differences in phase space, show that

\begin{align*}
    \Gamma \left( \rho^0 \to e^+e^- \right) \spc  : \spc   \Gamma \left( \omega \to  e^+e^- \right) \spc : \spc   \Gamma \left( \phi^- \to e^+e^- \right) \spc \approx \spc 9:1:2
\end{align*}\\
\end{problem}
\begin{solution}

\end{solution}

\noindent\rule{7in}{1.5pt}

%%%%%%%%%%%%%%%%%%%%%%%%%%%%%%%%%%%%%%%%%%%%%%%%%%%%%%%%%%%%%%%%%%%%%%%%%%%%%%%%%%%%%%%%%%%%%%%%%%%%%%%%%%%%%%%%%%%%%%%%%%%%%%%%%%%%%%%%

\begin{problem}{9.6}
Using the meson mass formulae of (9.37) and (9.38), obtain predictions for the masses of the $\pi^\pm , \pi^0 , \eta , \eta' , \rho^0 , \rho^\pm , \omega$ and $\phi$. Compare the values obtained to the experimental values listed in Table 9.1.
\end{problem}
\begin{solution}

\end{solution}

\noindent\rule{7in}{1.5pt}

%%%%%%%%%%%%%%%%%%%%%%%%%%%%%%%%%%%%%%%%%%%%%%%%%%%%%%%%%%%%%%%%%%%%%%%%%%%%%%%%%%%%%%%%%%%%%%%%%%%%%%%%%%%%%%%%%%%%%%%%%%%%%%%%%%%%%%%%

\begin{problem}{9.7}
Compare the experimentally measured values of the masses of the $J_P = \frac{3}{2}^+ $ baryons, given in Table 9.2, with the predictions of (9.41). You will need to consider the combined spin of any two quarks in a spin-3/2 baryon state.
\end{problem}
\begin{solution}

\end{solution}

\noindent\rule{7in}{1.5pt}

%%%%%%%%%%%%%%%%%%%%%%%%%%%%%%%%%%%%%%%%%%%%%%%%%%%%%%%%%%%%%%%%%%%%%%%%%%%%%%%%%%%%%%%%%%%%%%%%%%%%%%%%%%%%%%%%%%%%%%%%%%%%%%%%%%%%%%%%

\begin{problem}{9.8}
Starting from the wavefunction for the $\Sigma^-$ baryon :

\begin{enumerate}[label=(\alph*)]
    \item obtain the wavefunction for the $\Sigma^0$ and therefore find the wavefunction for the $\Lambda$
    \item using (9.41), obtain predictions for the masses of the $\Sigma^0$ and the $\Lambda$ baryons and compare these to the measured values.
\end{enumerate}
\end{problem}
\begin{solution}

\end{solution}

\noindent\rule{7in}{1.5pt}

%%%%%%%%%%%%%%%%%%%%%%%%%%%%%%%%%%%%%%%%%%%%%%%%%%%%%%%%%%%%%%%%%%%%%%%%%%%%%%%%%%%%%%%%%%%%%%%%%%%%%%%%%%%%%%%%%%%%%%%%%%%%%%%%%%%%%%%%

\begin{problem}{9.9}
Show that the quark model predictions for the magnetic moments of the $\Sigma^+,\Sigma^-$ and $\Omega^-$ baryons are

\begin{align*}
    \mu(\Sigma^+) = \frac{1}{3} \left( 4\mu_u - \mu_s \right) \spc , \spc \mu(\Sigma^-) = \frac{1}{3} \left( 4\mu_d - \mu_s \right) \andtxt \mu(\Omega^-) = 3\mu_s .
\end{align*}\\
What values of the quark constituent masses are required to give the best agreement with the measured values of

\begin{align*}
    \mu(\Sigma^+) = \left( 2.46 \pm 0.01 \right) \mu_N  \spc , \spc \mu(\Sigma^-) = \left( -1.16 \pm 0.03 \right) \mu_N \andtxt \mu(\Omega^-) = \left( -2.02 \pm 0.06 \right)  \mu_N .
\end{align*}
\end{problem}
\begin{solution}

\end{solution}

\noindent\rule{7in}{1.5pt}

%%%%%%%%%%%%%%%%%%%%%%%%%%%%%%%%%%%%%%%%%%%%%%%%%%%%%%%%%%%%%%%%%%%%%%%%%%%%%%%%%%%%%%%%%%%%%%%%%%%%%%%%%%%%%%%%%%%%%%%%%%%%%%%%%%%%%%%%

\begin{problem}{9.10}
If the colour did not exist, baryon wavefunctions would be constructed from

\begin{align*}
    \psi = \phi_\text{flavour} \chi_\text{spin} \eta_\text{space} .
\end{align*}\\
Taking $L = 0$ and using the flavour and spin wavefunctions derived in the text:

\begin{enumerate}[label=(\alph*)]
    \item show that it is still possible to construct a wavefunction for a spin-up proton for which $\phi_\text{flavour} \chi_\text{spin}$ is totally antisymmetric;
    \item predict the baryon multiplet structure for this model;
    \item for this colourless model, show that $\mu_p$ is negative and that the ratio of the neutron and proton magnetic moments would be
    
    \begin{align*}
        \frac{\mu_n}{\mu_p} = -2 .
    \end{align*}\\
\end{enumerate}
\end{problem}
\begin{solution}

\end{solution}

\noindent\rule{7in}{1.5pt}

%%%%%%%%%%%%%%%%%%%%%%%%%%%%%%%%%%%%%%%%%%%%%%%%%%%%%%%%%%%%%%%%%%%%%%%%%%%%%%%%%%%%%%%%%%%%%%%%%%%%%%%%%%%%%%%%%%%%%%%%%%%%%%%%%%%%%%%%
