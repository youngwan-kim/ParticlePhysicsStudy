
\noindent\rule{7in}{2.8pt}
\section{Symmetries and the Quark Model}
    
\begin{problem}{9.1}
By writing down the general term in the binomial expansion of

\begin{align*}
    \left( 1+ i \frac{1}{n} \boldsymbol{\alpha} \cdot \hat{\boldsymbol{G}}\right)^n,
\end{align*}\\
show that

\begin{align*}
    \hat{U}(\boldsymbol{\alpha}) = \lim_{n\to\infty}   \left( 1+ i \frac{1}{n} \boldsymbol{\alpha} \cdot \hat{\boldsymbol{G}}\right)^n = \exp \left( i   \boldsymbol{\alpha} \cdot \hat{\boldsymbol{G}} \right).
\end{align*}\\
\end{problem}
\begin{solution}
From the general term of the binomial expansion,

    \begin{align*}
        \left( 1+ i \frac{1}{n} \boldsymbol{\alpha} \cdot \hat{\boldsymbol{G}}\right)^n &= \sum_{k=0}^n \frac{n!}{k!\left( n-k \right)!} \left(  i \frac{1}{n} \boldsymbol{\alpha} \cdot \hat{\boldsymbol{G}}\right)^k \\[0.12in]
        &= \sum_{k=0}^n \frac{\left( n-k+1 \right) \cdots\left( n-1 \right) n}{k! } \frac{1}{n^k} \left(  i   \boldsymbol{\alpha} \cdot \hat{\boldsymbol{G}}\right)^k \\[0.12in]
        &=  \sum_{k=0}^n  \frac{n-k+1}{n} \frac{n-k+2}{n} \cdots \frac{n-1}{n} \frac{n}{n} \frac{1}{k!} \left(  i   \boldsymbol{\alpha} \cdot \hat{\boldsymbol{G}}\right)^k \\[0.12in]
        &= \sum_{k=0}^n \left[ \prod_{j=1}^k\left(  1-\frac{k-j}{n}  \right)\right]\frac{1}{k!} \left(  i   \boldsymbol{\alpha} \cdot \hat{\boldsymbol{G}}\right)^k
    \end{align*}\\
    Now taking the limit of $n\to\infty$ gives,

    \begin{align*}
        \hat{U}(\boldsymbol{\alpha}) &= \lim_{n\to\infty}   \left( 1+ i \frac{1}{n} \boldsymbol{\alpha} \cdot \hat{\boldsymbol{G}}\right)^n \\[0.12in]
        &= \lim_{n\to\infty} \sum_{k=0}^n \Bigg[ \cancelto{1}{\prod_{j=1}^k\left(  1-\frac{k-j}{n}  \right)}\Bigg]\frac{1}{k!} \left(  i   \boldsymbol{\alpha} \cdot \hat{\boldsymbol{G}}\right)^k \\[0.12in]
        &= \sum_{k=0}^n \frac{1}{k!} \left(  i   \boldsymbol{\alpha} \cdot \hat{\boldsymbol{G}}\right)^k = \exp \left( i   \boldsymbol{\alpha} \cdot \hat{\boldsymbol{G}} \right) \qed
    \end{align*}
\end{solution}

\noindent\rule{7in}{1.5pt}

%%%%%%%%%%%%%%%%%%%%%%%%%%%%%%%%%%%%%%%%%%%%%%%%%%%%%%%%%%%%%%%%%%%%%%%%%%%%%%%%%%%%%%%%%%%%%%%%%%%%%%%%%%%%%%%%%%%%%%%%%%%%%%%%%%%%%%%%

\begin{problem}{9.2}
    For an infinitesimal rotation about the z-axis through an angle $\epsilon$ show that

    \begin{align*}
        \hat{U} = 1-i\epsilon \hat{J}_z,
    \end{align*}\\
    where $\hat{J}_z$ is the angular momentum operator $\hat{J}_z = x \hatp_y - y\hatp_x $. 
\end{problem}
\begin{solution}
One could write down the new coordinates $(x',y')$ after an infinitesimal counter clockwise z-axis rotation of $\epsilon$ as 

\begin{align*}
    x' &= x \cos \epsilon + y \sin \epsilon = x + y \epsilon + \mathcal{O}(\epsilon^2) \\[0.1in]
    y' &= - x\sin \epsilon + y \cos \theta = -x \epsilon + y + \mathcal{O}(\epsilon^2) 
\end{align*}\\
which transforms the wavefunction as, 

\begin{align*}
    \psi(x,y,z) \to \psi'(x',y',z') &= \psi(x+y\epsilon,-x\epsilon+y,z) + \mathcal{O}(\epsilon^2) \\[0.135in]
    &= \psi(x,y,z) + y\epsilon \pder{\psi}{x} - x\epsilon \pder{\psi}{y} + \mathcal{O}(\epsilon^2) \\[0.12in]
    &\simeq \left[  1 - \epsilon \left( x \pder{}{y} - y \pder{}{x} \right) \right] \psi(x,y,z) = \left[ 1-i\epsilon \left( x \hatp_y - y \hatp_x \right) \right] \psi  \qed 
\end{align*}
\end{solution}

\noindent\rule{7in}{1.5pt}

%%%%%%%%%%%%%%%%%%%%%%%%%%%%%%%%%%%%%%%%%%%%%%%%%%%%%%%%%%%%%%%%%%%%%%%%%%%%%%%%%%%%%%%%%%%%%%%%%%%%%%%%%%%%%%%%%%%%%%%%%%%%%%%%%%%%%%%%

\begin{problem}{9.3}
By considering the isospin states, show that the rates for the following strong interaction decays occur in the ratios

\begin{align*}
    \Gamma \left( \Delta^- \to \pi^- n \right) \spc : \spc   &\Gamma \left( \Delta^0 \to \pi^- p \right) \spc  : \spc   \Gamma \left( \Delta^0 \to \pi^0 n \right) \spc : \spc   \Gamma \left( \Delta^+ \to \pi^+ n \right) \spc : \\
    &\Gamma \left( \Delta^+ \to \pi^0 p \right) \spc : \spc    \Gamma \left( \Delta^{++} \to \pi^+ p \right) \spc =  \spc 3:1:2:1:2:3
\end{align*}\\
\end{problem}
\begin{solution}
One should first note that branching ratio of certain decay modes depend on their cross-section which is proportional to the squared matrix element. Considering isospin states, the matrix element $\mathcal{M}$ can be expressed in terms of initial and final isospin states as, 

\begin{align*}
    \mathcal{M} = \braket{\psi_i|A_{if}|\psi_f}
\end{align*}\\
where $A_{if}$ is the isospin operator. Upon the fact that for any total isospin $I$, the strong force acts the same, it is good to specify $\mathcal{M}_I$ for each total isospin $I$ which is defined as 

\begin{align*}
    \mathcal{M}_I = \braket{\psi^I_f|A_I|\psi^I_i}
\end{align*}\\
where $A_I$ corresponds to the isospin operator for total isospin $I$. The first step is to identify the isospin states. This could be achieved by realizing that the isospin states can be composed using Clebsch-Gordan coefficients, which is summarized in Figure \ref{P9.3.f1}
\begin{figure}[tbh!] 
    \begin{center}
    \setlength{\unitlength}{1mm}
    \begin{picture}(140,65)(0,10)
    \scriptsize

    %
    % j1 x j2
    %
    \put(4,26){\line(1,0){36}}
    \put(4,26){\line(0,1){14}}
    \put(20,40){\line(-1,0){16}}
    \put(20,40){\line(0,1){8}}
    \put(40,48){\line(-1,0){20}}
    \put(40,48){\line(0,-1){22}}
    \multiput(20,40)(2,0){10}{\line(1,0){1}}
    \multiput(20,40)(0,-2){7}{\line(0,-1){1}}
    \put(23,46){\makebox(0,0){$J$}}
    \put(23,42){\makebox(0,0){$M$}}
    \put(29,46){\makebox(0,0){$J$}}
    \put(29,42){\makebox(0,0){$M$}}
    \put(35,46){\makebox(0,0){$\cdots$}}
    \put(35,42){\makebox(0,0){$\cdots$}}
    \put(8,38){\makebox(0,0){$m_1$}}
    \put(16,38){\makebox(0,0){$m_2$}}
    \put(8,34){\makebox(0,0){$m_1$}}
    \put(16,34){\makebox(0,0){$m_2$}}
    \put(8,31){\makebox(0,0){$\vdots$}}
    \put(16,31){\makebox(0,0){$\vdots$}}
    \put(12,44){\makebox(0,0){\normalsize $j_1 \, \times \, j_2$}}
    \put(30,33){\makebox(0,0){Coefficients}}
    
    
    
    
    %
    % 1/2 x 1/2
    %
    \put(60,58){\line(1,0){24}}
    \put(60,58){\line(0,1){4}}
    \put(76,62){\line(-1,0){16}}
    \put(76,62){\line(0,1){8}}
    \put(84,70){\line(-1,0){8}}
    \put(84,70){\line(0,-1){12}}
    \multiput(76,62)(2,0){4}{\line(1,0){1}}
    \multiput(76,62)(0,-2){2}{\line(0,-1){1}}
    \put(68,50){\line(1,0){32}}
    \put(68,50){\line(0,1){8}}
    \put(100,66){\line(-1,0){16}}
    \put(100,66){\line(0,-1){16}}
    \multiput(84,58)(2,0){8}{\line(1,0){1}}
    \multiput(84,58)(0,-2){4}{\line(0,-1){1}}
    \put(84,46){\line(1,0){24}}
    \put(84,46){\line(0,1){4}}
    \put(108,58){\line(-1,0){8}}
    \put(108,58){\line(0,-1){12}}
    \multiput(100,50)(2,0){4}{\line(1,0){1}}
    \multiput(100,50)(0,-2){2}{\line(0,-1){1}}
    \put(66,66){\makebox(0,0){\normalsize 1/2$\, \times \,$1/2}}
    \put(80,68){\makebox(0,0){1}}
    \put(80,64){\makebox(0,0){1}}
    \put(88,64){\makebox(0,0){1}}
    \put(96,64){\makebox(0,0){0}}
    \put(64,60){\makebox(0,0){1/2}}
    \put(72,60){\makebox(0,0){1/2}}
    \put(80,60){\makebox(0,0){1}}
    \put(88,60){\makebox(0,0){0}}
    \put(96,60){\makebox(0,0){0}}
    \put(72,56){\makebox(0,0){1/2}}
    \put(80,56){\makebox(0,0){-1/2}}
    \put(88,56){\makebox(0,0){1/2}}
    \put(96,56){\makebox(0,0){1/2}}
    \put(104,56){\makebox(0,0){1}}
    \put(72,52){\makebox(0,0){-1/2}}
    \put(80,52){\makebox(0,0){1/2}}
    \put(88,52){\makebox(0,0){1/2}}
    \put(96,52){\makebox(0,0){-1/2}}
    \put(104,52){\makebox(0,0){-1}}
    \put(88,48){\makebox(0,0){-1/2}}
    \put(96,48){\makebox(0,0){-1/2}}
    \put(104,48){\makebox(0,0){1}}
    
        
        
        %
    % 1 x 1/2
    %
    \put(76,32){\line(-1,0){16}}
    \put(76,32){\line(0,1){8}}
    \put(60,28){\line(1,0){24}}
    \put(60,28){\line(0,1){4}}
    \put(84,40){\line(-1,0){8}}
    \put(84,40){\line(0,-1){12}}
    \multiput(76,32)(2,0){4}{\line(1,0){1}}
    \multiput(76,32)(0,-2){2}{\line(0,-1){1}}
    \put(68,20){\line(1,0){32}}
    \put(68,20){\line(0,1){8}}
    \put(100,36){\line(-1,0){16}}
    \put(100,36){\line(0,-1){16}}
    \multiput(84,28)(2,0){8}{\line(1,0){1}}
    \multiput(84,28)(0,-2){4}{\line(0,-1){1}}
    \put(84,12){\line(1,0){32}}
    \put(84,12){\line(0,1){8}}
    \put(116,28){\line(-1,0){16}}
    \put(116,28){\line(0,-1){16}}
    \multiput(100,20)(2,0){8}{\line(1,0){1}}
    \multiput(100,20)(0,-2){4}{\line(0,-1){1}}
    \put(100,8){\line(1,0){24}}
    \put(100,8){\line(0,1){4}}
    \put(124,20){\line(-1,0){8}}
    \put(124,20){\line(0,-1){12}}
    \multiput(116,12)(2,0){4}{\line(1,0){1}}
    \multiput(116,12)(0,-2){2}{\line(0,-1){1}}
    \put(68,36){\makebox(0,0){\normalsize 1$\, \times \,$1/2}}
    \put(80,38){\makebox(0,0){3/2}}
    \put(80,34){\makebox(0,0){3/2}}
    \put(88,34){\makebox(0,0){3/2}}
    \put(96,34){\makebox(0,0){1/2}}
    \put(64,30){\makebox(0,0){1}}
    \put(72,30){\makebox(0,0){1/2}}
    \put(80,30){\makebox(0,0){1}}
    \put(88,30){\makebox(0,0){1/2}}
    \put(96,30){\makebox(0,0){1/2}}
    \put(72,26){\makebox(0,0){1}}
    \put(80,26){\makebox(0,0){-1/2}}
    \put(88,26){\makebox(0,0){1/3}}
    \put(96,26){\makebox(0,0){2/3}}
    \put(104,26){\makebox(0,0){3/2}}
    \put(112,26){\makebox(0,0){1/2}}
    \put(72,22){\makebox(0,0){0}}
    \put(80,22){\makebox(0,0){1/2}}
    \put(88,22){\makebox(0,0){2/3}}
    \put(96,22){\makebox(0,0){-1/3}}
    \put(104,22){\makebox(0,0){-1/2}}
    \put(112,22){\makebox(0,0){-1/2}}
    \put(88,18){\makebox(0,0){0}}
    \put(96,18){\makebox(0,0){-1/2}}
    \put(104,18){\makebox(0,0){2/3}}
    \put(112,18){\makebox(0,0){1/3}}
    \put(120,18){\makebox(0,0){3/2}}
    \put(88,14){\makebox(0,0){-1}}
    \put(96,14){\makebox(0,0){1/2}}
    

    \end{picture}
    \end{center}
    \caption{Clebsch-Gordan coefficients where square roots are understood on each coefficient, that is, $-1/2$ meaning $\protect -\sqrt{1/2}$.}
    \label{P9.3.f1}
\end{figure}\\
The initial state which are solely $\Delta$ baryons, can be represented by the following isospin states where each states are displayed in the form of $\ket{I\spc I_3}$ : 

\begin{align*}
    \psi_{\Delta^{++}} =  \Big| \frac{3}{2} \spc \frac{3}{2} \Big\rangle \comma \psi_{\Delta^{+}} =  \Big| \frac{3}{2} \spc \frac{1}{2} \Big\rangle \comma \psi_{\Delta^{0}} =  \Big| \frac{3}{2} \spc -\frac{1}{2} \Big\rangle \comma \psi_{\Delta^{-}} =  \Big| \frac{3}{2} \spc -\frac{3}{2} \Big\rangle 
\end{align*}\\
The isospin states for the $\pi$ mesons are simply derived using Figure \ref{P9.3.f1},

\begin{align*}
    \psi_{\pi^+} &= \psi_{uu} =  \Big| \frac{1}{2} \spc \frac{1}{2} \Big\rangle \otimes  \Big| \frac{1}{2} \spc \frac{1}{2} \Big\rangle = \ket{1\spc 1} \\[0.12in]
    \psi_{\pi^-} &= \psi_{dd} =  \Big| \frac{1}{2} \spc -\frac{1}{2} \Big\rangle \otimes  \Big| \frac{1}{2} \spc -\frac{1}{2} \Big\rangle = \ket{1\spc -1} \\[0.12in]
    \psi_{\pi^0} &= \psi_{\frac{1}{\sqrt{2}}\left( ud+du \right)} =  \frac{1}{\sqrt{2}}\left[ \Big| \frac{1}{2} \spc \frac{1}{2} \Big\rangle \otimes  \Big| \frac{1}{2} \spc -\frac{1}{2} \Big\rangle +   \Big| \frac{1}{2} \spc- \frac{1}{2} \Big\rangle \otimes  \Big| \frac{1}{2} \spc  \frac{1}{2} \Big\rangle  \right]= \ket{1\spc 0} 
\end{align*}\\
And for the decayed final state, again using the Clebsch-Gordan coefficients from Figure \ref{P9.3.f1}, the isospin states can be obtained :

\begin{align*}
    \psi_{\pi^+,p} &= \ket{1\spc 1} \otimes \Big| \frac{1}{2} \spc \frac{1}{2} \Big\rangle = \Big| \frac{3}{2} \spc \frac{3}{2} \Big\rangle \\[0.12in]
    \psi_{\pi^+,n} &= \ket{1\spc 1} \otimes \Big| \frac{1}{2} \spc -\frac{1}{2} \Big\rangle = \sqrt{\frac{1}{3}} \Big| \frac{3}{2} \spc \frac{1}{2} \Big\rangle + \sqrt{\frac{2}{3}} \Big| \frac{1}{2} \spc \frac{1}{2} \Big\rangle \\[0.12in]
    \psi_{\pi^0,p} &= \ket{1\spc 0} \otimes \Big| \frac{1}{2} \spc \frac{1}{2} \Big\rangle = \sqrt{\frac{2}{3}} \Big| \frac{3}{2} \spc \frac{1}{2} \Big\rangle - \sqrt{\frac{1}{3}} \Big| \frac{1}{2} \spc \frac{1}{2} \Big\rangle \\[0.12in]
    \psi_{\pi^0,n} &= \ket{1\spc 0} \otimes \Big| \frac{1}{2} \spc -\frac{1}{2} \Big\rangle = \sqrt{\frac{2}{3}} \Big| \frac{3}{2} \spc -\frac{1}{2} \Big\rangle + \sqrt{\frac{1}{3}} \Big| \frac{1}{2} \spc -\frac{1}{2} \Big\rangle \\[0.12in]
    \psi_{\pi^-,p} &= \ket{1\spc -1} \otimes \Big| \frac{1}{2} \spc \frac{1}{2} \Big\rangle = \sqrt{\frac{1}{3}} \Big| \frac{3}{2} -\spc \frac{1}{2} \Big\rangle + \sqrt{\frac{2}{3}} \Big| \frac{1}{2} \spc -\frac{1}{2} \Big\rangle \\[0.12in]
    \psi_{\pi^-,n} &= \ket{1\spc -1} \otimes \Big| \frac{1}{2} \spc -\frac{1}{2} \Big\rangle =   \Big| \frac{3}{2} \spc -\frac{3}{2} \Big\rangle 
\end{align*}\\
Now using the above isospin expression of each initial and final states, one could estimate the respective cross section and their ratios. As the initial states are all represented with pure $I=3/2$ states, one would only have to compare the corresponding coefficients which is to only consider $\mathcal{M}_{3/2}$. For instance, consider the $\Delta^- \to \pi^- n $ decay mode, 

\begin{align*}
    \mathcal{M}_{3/2} \left( \Delta^- \to \pi^- n \right) \sim \braket{\psi_{\Delta^-}  | \psi_{\pi^-,n}^{3/2}} =   \Big\langle \frac{3}{2} -\spc \frac{3}{2}  \Big| \frac{3}{2} -\spc \frac{3}{2} \Big\rangle  
\end{align*}\\
and considering that $\Gamma \sim \mathcal{M}^2$, one could say that 

\begin{align*}
    \Gamma\left( \Delta^- \to \pi^- n \right) \sim \mathcal{M}_{3/2} \left( \Delta^- \to \pi^- n \right)^2 \sim 1
\end{align*}\\
up to some constant factor. Doing the same for all of the other processes, one could obtain 

\begin{align*}
    \Gamma\left( \Delta^0 \to \pi^- p \right) &\sim \mathcal{M}_{3/2} \left( \Delta^0 \to \pi^- p \right)^2 = \left[ \sqrt{\frac{1}{3}}   \Big\langle \frac{3}{2} \spc -\frac{1}{2}  \Big| \frac{3}{2}  \spc- \frac{1}{2} \Big\rangle   \right]^2 \sim \frac{1}{3} \\[0.12in]
    \Gamma\left( \Delta^0 \to \pi^0 n \right) &\sim \mathcal{M}_{3/2} \left( \Delta^0 \to \pi^0 n  \right)^2 = \left[ \sqrt{\frac{2}{3}}   \Big\langle \frac{3}{2}  \spc -\frac{1}{2}  \Big| \frac{3}{2}  \spc- \frac{1}{2} \Big\rangle   \right]^2 \sim \frac{2}{3} \\[0.12in]
    \Gamma\left( \Delta^+ \to \pi^+ n \right) &\sim \mathcal{M}_{3/2} \left( \Delta^+ \to \pi^+ n  \right)^2 = \left[ \sqrt{\frac{1}{3}}   \Big\langle \frac{3}{2}  \spc \frac{1}{2}  \Big| \frac{3}{2}  \spc \frac{1}{2} \Big\rangle   \right]^2 \sim \frac{1}{3} \\[0.12in]
    \Gamma\left( \Delta^+ \to \pi^0 p \right) &\sim \mathcal{M}_{3/2} \left( \Delta^+ \to \pi^0 p  \right)^2 = \left[ \sqrt{\frac{2}{3}}   \Big\langle \frac{3}{2}  \spc \frac{1}{2}  \Big| \frac{3}{2}  \spc \frac{1}{2} \Big\rangle   \right]^2 \sim \frac{2}{3} \\[0.12in]
    \Gamma\left( \Delta^{++} \to \pi^+ p \right) &\sim \mathcal{M}_{3/2} \left( \Delta^{++} \to \pi^+ p  \right)^2 = \left[     \Big\langle \frac{3}{2}  \spc \frac{3}{2}  \Big| \frac{3}{2}  \spc \frac{3}{2} \Big\rangle   \right]^2 \sim 1  
\end{align*}\\[0.12in]
by multiplying all with a constant factor of 3, one would finally obtain the ratio of $\boxed{3:1:2:1:2:3}$ for the given branching ratios. 
\end{solution}

\noindent\rule{7in}{1.5pt}

%%%%%%%%%%%%%%%%%%%%%%%%%%%%%%%%%%%%%%%%%%%%%%%%%%%%%%%%%%%%%%%%%%%%%%%%%%%%%%%%%%%%%%%%%%%%%%%%%%%%%%%%%%%%%%%%%%%%%%%%%%%%%%%%%%%%%%%%

\begin{problem}{9.4}
If quarks and antiquarks were spin-zero particles, what would be the multiplicity of the $L = 0$ multiplet(s). Remember that the overall wavefunction for bosons must be symmetric under particle exchange.
\end{problem}
\begin{solution}
As mentioned in the textbook, it is known that the colour wavefunction $\xi_\text{colour}$ in the  overall wavefunction $\psi=\phi_\text{flavour}\chi_\text{spin}\xi_\text{colour}\eta_\text{space}$ is necessarily totally antisymmetric, plus only mesonic and baryonic multiplets are available due to colour confinement.  
    \begin{enumerate}[label=(\roman*)]
        \item Meson ($q\overbar{q}$)
        
            As quarks and antiquarks are distinguishable, there is no restriction on the exchange symmetry of the overall wavefunction. Thus the multiplicity will simply be \boxed{9} in this case.
        \item Baryon ($qqq$)
        
        If (anti)quarks were spin-zero particles, the spin-statistics theorem states that the overall wavefunction $\psi=\phi_\text{flavour}\chi_\text{spin}\xi_\text{colour}\eta_\text{space}$ of the multiplets should be symmetric under the interchange of any two of the quarks. Also, as $L=0$ is considered the quarks are descirbed by $l=0$ s-waves, thus the exchange symmetry factor is given as $(-1)^l = 1$ for any two quarks which implies that $\eta_\text{space}$ is symmetric. As $\xi_\text{colour}\eta_\text{space}$ is antisymmetric, $\phi_\text{flavour}\chi_\text{spin}$ must be antisymmetric too in order to make $\psi$ totally symmetric. For the spin wavefunction, any multiplet with arbitrary multiplicity will end up being a singlet as it would be a combination of spin zero particles. This boils down to the requirement that if $\phi_\text{flavour}$ is totally antisymmetric, the overall wavefunction will be symmetric. As there is only one totally antisymmetric flavour state, in this case the multiplicity will be \boxed{1}.
    \end{enumerate}

\end{solution}

\noindent\rule{7in}{1.5pt}

%%%%%%%%%%%%%%%%%%%%%%%%%%%%%%%%%%%%%%%%%%%%%%%%%%%%%%%%%%%%%%%%%%%%%%%%%%%%%%%%%%%%%%%%%%%%%%%%%%%%%%%%%%%%%%%%%%%%%%%%%%%%%%%%%%%%%%%%

\begin{problem}{9.5}
The neutral vector mesons can decay leptonically through a virtual photon, for example by $V(q\overbar{q}) \to \gamma \to e^+e^-$. The matrix element for this decay is proportional to $\braket{\psi|\hat{Q}_q|\psi}$ where $\psi$ is the meson flavour wavefunction and $\hat{Q}_q$ is an operator that is proportional to the quark charge. Neglecting the relatively small differences in phase space, show that

\begin{align*}
    \Gamma \left( \rho^0 \to e^+e^- \right) \spc  : \spc   \Gamma \left( \omega \to  e^+e^- \right) \spc : \spc   \Gamma \left( \phi^- \to e^+e^- \right) \spc \approx \spc 9:1:2
\end{align*}\\
\end{problem}
\begin{solution}

\end{solution}

\noindent\rule{7in}{1.5pt}

%%%%%%%%%%%%%%%%%%%%%%%%%%%%%%%%%%%%%%%%%%%%%%%%%%%%%%%%%%%%%%%%%%%%%%%%%%%%%%%%%%%%%%%%%%%%%%%%%%%%%%%%%%%%%%%%%%%%%%%%%%%%%%%%%%%%%%%%

\begin{problem}{9.6}
Using the meson mass formulae of (9.37) and (9.38), obtain predictions for the masses of the $\pi^\pm , \pi^0 , \eta , \eta' , \rho^0 , \rho^\pm , \omega$ and $\phi$. Compare the values obtained to the experimental values listed in Table 9.1.
\end{problem}
\begin{solution}

\end{solution}

\noindent\rule{7in}{1.5pt}

%%%%%%%%%%%%%%%%%%%%%%%%%%%%%%%%%%%%%%%%%%%%%%%%%%%%%%%%%%%%%%%%%%%%%%%%%%%%%%%%%%%%%%%%%%%%%%%%%%%%%%%%%%%%%%%%%%%%%%%%%%%%%%%%%%%%%%%%

\begin{problem}{9.7}
Compare the experimentally measured values of the masses of the $J_P = \frac{3}{2}^+ $ baryons, given in Table 9.2, with the predictions of (9.41). You will need to consider the combined spin of any two quarks in a spin-3/2 baryon state.
\end{problem}
\begin{solution}

\end{solution}

\noindent\rule{7in}{1.5pt}

%%%%%%%%%%%%%%%%%%%%%%%%%%%%%%%%%%%%%%%%%%%%%%%%%%%%%%%%%%%%%%%%%%%%%%%%%%%%%%%%%%%%%%%%%%%%%%%%%%%%%%%%%%%%%%%%%%%%%%%%%%%%%%%%%%%%%%%%

\begin{problem}{9.8}
Starting from the wavefunction for the $\Sigma^-$ baryon :

\begin{enumerate}[label=(\alph*)]
    \item obtain the wavefunction for the $\Sigma^0$ and therefore find the wavefunction for the $\Lambda$
    \item using (9.41), obtain predictions for the masses of the $\Sigma^0$ and the $\Lambda$ baryons and compare these to the measured values.
\end{enumerate}
\end{problem}
\begin{solution}

\end{solution}

\noindent\rule{7in}{1.5pt}

%%%%%%%%%%%%%%%%%%%%%%%%%%%%%%%%%%%%%%%%%%%%%%%%%%%%%%%%%%%%%%%%%%%%%%%%%%%%%%%%%%%%%%%%%%%%%%%%%%%%%%%%%%%%%%%%%%%%%%%%%%%%%%%%%%%%%%%%

\begin{problem}{9.9}
Show that the quark model predictions for the magnetic moments of the $\Sigma^+,\Sigma^-$ and $\Omega^-$ baryons are

\begin{align*}
    \mu(\Sigma^+) = \frac{1}{3} \left( 4\mu_u - \mu_s \right) \spc , \spc \mu(\Sigma^-) = \frac{1}{3} \left( 4\mu_d - \mu_s \right) \andtxt \mu(\Omega^-) = 3\mu_s .
\end{align*}\\
What values of the quark constituent masses are required to give the best agreement with the measured values of

\begin{align*}
    \mu(\Sigma^+) = \left( 2.46 \pm 0.01 \right) \mu_N  \spc , \spc \mu(\Sigma^-) = \left( -1.16 \pm 0.03 \right) \mu_N \andtxt \mu(\Omega^-) = \left( -2.02 \pm 0.06 \right)  \mu_N .
\end{align*}
\end{problem}
\begin{solution}

\end{solution}

\noindent\rule{7in}{1.5pt}

%%%%%%%%%%%%%%%%%%%%%%%%%%%%%%%%%%%%%%%%%%%%%%%%%%%%%%%%%%%%%%%%%%%%%%%%%%%%%%%%%%%%%%%%%%%%%%%%%%%%%%%%%%%%%%%%%%%%%%%%%%%%%%%%%%%%%%%%

\begin{problem}{9.10}
If the colour did not exist, baryon wavefunctions would be constructed from

\begin{align*}
    \psi = \phi_\text{flavour} \chi_\text{spin} \eta_\text{space} .
\end{align*}\\
Taking $L = 0$ and using the flavour and spin wavefunctions derived in the text:

\begin{enumerate}[label=(\alph*)]
    \item show that it is still possible to construct a wavefunction for a spin-up proton for which $\phi_\text{flavour} \chi_\text{spin}$ is totally antisymmetric;
    \item predict the baryon multiplet structure for this model;
    \item for this colourless model, show that $\mu_p$ is negative and that the ratio of the neutron and proton magnetic moments would be
    
    \begin{align*}
        \frac{\mu_n}{\mu_p} = -2 .
    \end{align*}\\
\end{enumerate}
\end{problem}
\begin{solution}

\end{solution}

\noindent\rule{7in}{1.5pt}

%%%%%%%%%%%%%%%%%%%%%%%%%%%%%%%%%%%%%%%%%%%%%%%%%%%%%%%%%%%%%%%%%%%%%%%%%%%%%%%%%%%%%%%%%%%%%%%%%%%%%%%%%%%%%%%%%%%%%%%%%%%%%%%%%%%%%%%%
