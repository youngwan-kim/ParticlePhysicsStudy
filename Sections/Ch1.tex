\section{Introduction}

\begin{problem}{1.1}
Feynman diagrams are constructed out of the Standard Model vertices shown in Figure 1.4.
Only the weak charged-current interaction can change the flavour of the particle at the interaction vertex.
Explaining your reasoning, state whether each of the sixteen diagrams below represents a valid Standard Model vertex.
\end{problem}
\begin{solution}
    
\end{solution} 
\noindent\rule{7in}{1.5pt}
    
%%%%%%%%%%%%%%%%%%%%%%%%%%%%%%%%%%%%%%%%%%%%%%%%%%%%%%%%%%%%%%%%%%%%%%%%%
    

\begin{problem}{1.2}
Draw the Feynman diagram for $\tau^-\to\pi^-\nu_\tau$. (The $\pi^-$ is the lightest $d\bar{u}$ meson)
\end{problem}
\begin{solution} 
    \\[0.15in]
    \begin{center}
        \scalebox{1.5}{
        \begin{tikzpicture}
            \begin{feynman}
              \vertex (a1) {\(\tau^-\)};
              \vertex[right=2.5cm of a1] (a2) ;
              \vertex[right=5cm of a1] (a3) {\(\nu_\tau\)};
          
              \vertex[above=of a3] (c1) {\(\overline u\)};
              \vertex[above=2.5em of c1] (c3) {\(d\)};
              \vertex at ($(c1)!0.5!(c3) - (1cm, 0)$) (c2);
          
              \diagram* {
                {[edges=fermion]
                  (a1) -- (a2) -- (a3),
                },

                (c1) -- [fermion, out=180, in=-45] (c2) -- [fermion, out=45, in=180] (c3),
                (a2) -- [boson, bend left, edge label=\(W^{-}\)] (c2),
              };
          
              \draw [decoration={brace}, decorate] (c3.north east) -- (c1.south east)
                    node [pos=0.5, right] {\(\pi^{-}\)};
            \end{feynman}
          \end{tikzpicture}
        }
    \end{center}
\end{solution} 
\noindent\rule{7in}{1.5pt}

%%%%%%%%%%%%%%%%%%%%%%%%%%%%%%%%%%%%%%%%%%%%%%%%%%%%%%%%%%%%%%%%%%%%%%%%%%%%%%%%%%%%%%%%%%%%%%%%%%%%%%%%%%%%%%%%%%%%%%%%%%%%%%%%%%%%%%%%

\begin{problem}{1.3}
Explain why it is not possible to construct a valid Feynman diagram using the Standard Model vertices for the following processes :
\begin{enumerate}[label=(\alph*)]
    \item $\mu^- \to e^+e^-e^+$
    \item $\nu_\tau + p \to \mu^- + n$
    \item $\nu_\tau + p \to \tau^+ + n$
    \item $\pi^+(u\overbar{d})+\pi^-(d\overbar{u}) \to n(udd) + \pi^0(u\overbar{u})$
\end{enumerate}
\end{problem}
\begin{solution}
\begin{enumerate}[label=(\alph*)]
    \item $\mu^- \to e^+e^-e^+$ : Charge is not conserved, as well as lepton numbers.
    \item $\nu_\tau + p \to \mu^- + n$
    \item $\nu_\tau + p \to \tau^+ + n$
    \item $\pi^+(u\overbar{d})+\pi^-(d\overbar{u}) \to n(udd) + \pi^0(u\overbar{u})$
\end{enumerate}
\end{solution}

\noindent\rule{7in}{1.5pt}

%%%%%%%%%%%%%%%%%%%%%%%%%%%%%%%%%%%%%%%%%%%%%%%%%%%%%%%%%%%%%%%%%%%%%%%%%%%%%%%%%%%%%%%%%%%%%%%%%%%%%%%%%%%%%%%%%%%%%%%%%%%%%%%%%%%%%%%%

\begin{problem}{1.4}
Draw the Feynman diagram for the decays:
\begin{enumerate}[label=(\alph*)]
    \item $\Delta^+(udd)\to n(udd)\pi^+(u\overbar{d})$
    \item $\Sigma^0(uds)\to\Lambda(uds)\gamma$
    \item $pi^+(u\overbar{d})\to mu^+ \nu_\mu$
\end{enumerate}
\end{problem}
\begin{solution}
\begin{enumerate}[label=(\alph*)]
    \item $\Delta^+(udd)\to n(udd)\pi^+(u\overbar{d})$
    \item $\Sigma^0(uds)\to\Lambda(uds)\gamma$
    \item $pi^+(u\overbar{d})\to mu^+ \nu_\mu$
\end{enumerate}
\end{solution}

\noindent\rule{7in}{1.5pt}

%%%%%%%%%%%%%%%%%%%%%%%%%%%%%%%%%%%%%%%%%%%%%%%%%%%%%%%%%%%%%%%%%%%%%%%%%%%%%%%%%%%%%%%%%%%%%%%%%%%%%%%%%%%%%%%%%%%%%%%%%%%%%%%%%%%%%%%%

\begin{problem}{1.5}
Treating the $\pi^0$ as a $uu$ bound state, draw the Feynman diagrams for:
\begin{enumerate}[label=(\alph*)]
    \item $\pi^0\to\gamma\gamma$
    \item $\pi^0\to\gamma e^+e^-$
    \item $\pi^0\to e^+e^-e^+e^-$
    \item $\pi^0\to e^+e^-$
\end{enumerate}
\end{problem}
\begin{solution}

\end{solution}

\noindent\rule{7in}{1.5pt}

%%%%%%%%%%%%%%%%%%%%%%%%%%%%%%%%%%%%%%%%%%%%%%%%%%%%%%%%%%%%%%%%%%%%%%%%%%%%%%%%%%%%%%%%%%%%%%%%%%%%%%%%%%%%%%%%%%%%%%%%%%%%%%%%%%%%%%%%

\begin{problem}{1.6}
Particle interactions fall into two main categories, scattering processes and annihilation processes, as indicated by the Feynman diagrams below.
\begin{center}
    \scalebox{1.5}{
    \begin{tikzpicture}
        \begin{feynman}
            \vertex (a1);
            \vertex[below=4em of a1] (a3);
            \vertex[right=1.5em of a3] (a2t);
            \vertex[above=2em of a2t] (a2);
    
            \vertex[right=2em of a2] (b2);
            \vertex[right=5em of a1] (b1);
            \vertex[right=5em of a3] (b3);
    
            \diagram* {
                {[edges=fermion]
                  (a1) -- (a2) -- (a3),
                  (b3) -- (b2) -- (b1)
                },
                (a2) -- [boson] (b2)
            };
        \end{feynman}
    \end{tikzpicture} \hspace{0.15in}
    \begin{tikzpicture}
        \begin{feynman}
            \vertex (a1);
            \vertex[right=5em of a1] (a3);
            \vertex[right=2.5em of a1] (a2t);
            \vertex[below=1em of a2t] (a2);
    
            \vertex[below=4em of a1] (b1);
            \vertex[right=5em of b1] (b3);
            \vertex[right=2.5em of b1] (b2t);
            \vertex[above=1em of b2t] (b2);

            \diagram* {
                {[edges=fermion]
                  (a1) -- (a2) -- (a3),
                  (b1) -- (b2) -- (b3)
                },
                (a2) -- [boson] (b2)
            };
            
        \end{feynman}
    \end{tikzpicture}
    }
\end{center}
Draw the lowest-order Feynman diagrams for the scattering and/or annihilation processes:
\begin{enumerate}[label=(\alph*)]
    \item $e^-e^- \to e^-e^-$
    \item $e^+e^- \to \mu^+\mu^-$
    \item $e^+e^- \to e^+e^-$
    \item $e^-\nu_e \to e^-\nu_e$
    \item $e^-\overbar{\nu_e} \to e^-\overbar{\nu_e}$
\end{enumerate}
\end{problem}
\begin{solution}
\begin{enumerate}[label=(\alph*)]
    \item $e^-e^- \to e^-e^-$

    \item $e^+e^- \to \mu^+\mu^-$
    \item $e^+e^- \to e^+e^-$
    \item $e^-\nu_e \to e^-\nu_e$
    \item $e^-\overbar{\nu_e} \to e^-\overbar{\nu_e}$
\end{enumerate}
\end{solution}

\noindent\rule{7in}{1.5pt}

%%%%%%%%%%%%%%%%%%%%%%%%%%%%%%%%%%%%%%%%%%%%%%%%%%%%%%%%%%%%%%%%%%%%%%%%%%%%%%%%%%%%%%%%%%%%%%%%%%%%%%%%%%%%%%%%%%%%%%%%%%%%%%%%%%%%%%%%

\begin{problem}{1.7}
High-energy muons traversing matter lose energy according to
\begin{align*}
    -\frac{1}{\rho} \frac{\dif E}{\dif x} \approx a + bE 
\end{align*}
where $a$ is due to ionisation energy loss and $b$ is due to the bremsstrahlung and $e^+e^-$ pair-production processes. 
For standard rock, taken to have $A = 22, Z = 11$ and $\rho = 2.65 \unit{\gram\centi\metre^{-3}}$ , the parameters $a$ and $b$ depend only weakly on 
the muon energy and have values $a \approx 2.5 \MeV\unit{\gram^{-1}\centi\metre^2}$ and $b \approx \num{3.5e-6} \unit{\gram^{-1}\centi\metre^2}$.
\begin{enumerate}[label=(\alph*)]
    \item At what muon energy are the ionisation and bremsstrahlung/pair production processes equally important?
    \item Approximately how far does a 100 GeV cosmic-ray muon propagate in rock?
\end{enumerate}
\end{problem}
\begin{solution}

\end{solution}

\noindent\rule{7in}{1.5pt}

%%%%%%%%%%%%%%%%%%%%%%%%%%%%%%%%%%%%%%%%%%%%%%%%%%%%%%%%%%%%%%%%%%%%%%%%%%%%%%%%%%%%%%%%%%%%%%%%%%%%%%%%%%%%%%%%%%%%%%%%%%%%%%%%%%%%%%%%

\begin{problem}{1.8}
Tugsten has a radiation length of $X_0=0.35$ cm and a critical energy of $E_c-7.97$ MeV. Roughly what thickness of tungsten is required to fully contain a 500 GeV electromagnetic shower from an electron?
\end{problem}

\begin{solution}

\end{solution} 
\noindent\rule{7in}{1.5pt}

%%%%%%%%%%%%%%%%%%%%%%%%%%%%%%%%%%%%%%%%%%%%%%%%%%%%%%%%%%%%%%%%%%%%%%%%%%%%%%%%%%%%%%%%%%%%%%%%%%%%%%%%%%%%%%%%%%%%%%%%%%%%%%%%%%%%%%%%

\begin{problem}{1.9}
The CPLEAR detector consisted of: tracking detectors in a magnetic field of 0.44 T; and electromagnetic calorimeter;
and Čerenkov detectors with a radiator of refractive index $n=1.25$ used to distinguish $\pi^\pm$ from $K^\pm$.

A charged particle travelling perpendicular to the direction of the magnetic field leaves a track with a
measured radius of curvature of $R=4$m. If it is observed to give a Čerenkov signal, is it 
possible to distinguish between the particle being a pion or kaon? Take $m_\pi \approx 140 \MeV /\text{c}^2$ and $m_K \approx 494 \MeV /\text{c}^2$ 
\end{problem}
\begin{solution}

\end{solution}

\noindent\rule{7in}{1.5pt}

%%%%%%%%%%%%%%%%%%%%%%%%%%%%%%%%%%%%%%%%%%%%%%%%%%%%%%%%%%%%%%%%%%%%%%%%%%%%%%%%%%%%%%%%%%%%%%%%%%%%%%%%%%%%%%%%%%%%%%%%%%%%%%%%%%%%%%%%

\begin{problem}{1.10}
In a fixed-target pp experiment, what proton energy would be required to achieve the same centre-of-mass energy as the LHC, which will ultimately operate at 14 TeV.
\end{problem}
\begin{solution}
    
\end{solution} 
\noindent\rule{7in}{1.5pt}
    
%%%%%%%%%%%%%%%%%%%%%%%%%%%%%%%%%%%%%%%%%%%%%%%%%%%%%%%%%%%%%%%%%%%%%%%%%%%%%%%%%%%%%%%%%%%%%%%%%%%%%%%%%%%%%%%%%%%%%%%%%%%%%%%%%%%%%%%%

\begin{problem}{1.11}
At the LEP $e^+e^-$ collider, which had a circumference of 27 km, the electron and positron beam currents were both 1.0 mA. Each beam consisted of four equally spaced bunches of electrons/positrons. The bunches had an effec- tive area of $\num{1.8e4}\unit{\micro\metre^2}$.
Calculate the instantaneous luminosity on the assumption that the beams collided head-on.
\end{problem}
\begin{solution}

\end{solution}