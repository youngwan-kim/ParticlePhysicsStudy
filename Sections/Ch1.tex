\section{Introduction}

\begin{problem}{1.1}
Feynman diagrams are constructed out of the Standard Model vertices shown in Figure 1.4.
Only the weak charged-current interaction can change the flavour of the particle at the interaction vertex.
Explaining your reasoning, state whether each of the sixteen diagrams below represents a valid Standard Model vertex.
\end{problem}
\begin{solution}
\begin{enumerate}[label=(\alph*)]
    \item Valid. 
    \item Invalid, due to the fact that $\nu_e$ has no electric charge. 
    \item Valid.
    \item Valid.
    \item Invalid. Flavor shouldn't change. 
    \item Valid.
    \item Invalid. Flavor shouldn't change. 
    \item Invalid. Flavor shouldn't change. 
    \item Invalid, leptons do not carry color charge.
    \item Valid. 
    \item Valid.
    \item Invalid.
    \item Invalid.
    \item Valid.
    \item Valid.
    \item Invalid. No such 4-point vertex in QED.
\end{enumerate}
\end{solution} 
\noindent\rule{7in}{1.5pt}
    
%%%%%%%%%%%%%%%%%%%%%%%%%%%%%%%%%%%%%%%%%%%%%%%%%%%%%%%%%%%%%%%%%%%%%%%%%
    

\begin{problem}{1.2}
Draw the Feynman diagram for $\tau^-\to\pi^-\nu_\tau$. (The $\pi^-$ is the lightest $d\bar{u}$ meson)
\end{problem}
\begin{solution} 
    \\[0.15in]
    \begin{center}
        \scalebox{1.25}{
        \begin{tikzpicture}
            \begin{feynman}
              \vertex (a1) {\(\tau^-\)};
              \vertex[right=2.5cm of a1] (a2) ;
              \vertex[right=5cm of a1] (a3) {\(\nu_\tau\)};
          
              \vertex[above=of a3] (c1) {\(\overline u\)};
              \vertex[above=2.5em of c1] (c3) {\(d\)};
              \vertex at ($(c1)!0.5!(c3) - (1cm, 0)$) (c2);
          
              \diagram* {
                {[edges=fermion]
                  (a1) -- (a2) -- (a3),
                },

                (c1) -- [fermion, out=180, in=-45] (c2) -- [fermion, out=45, in=180] (c3),
                (a2) -- [boson, bend left, edge label=\(W^{-}\)] (c2),
              };
          
              \draw [decoration={brace}, decorate] (c3.north east) -- (c1.south east)
                    node [pos=0.5, right] {\(\pi^{-}\)};
            \end{feynman}
          \end{tikzpicture}
        }
    \end{center}
\end{solution} 
\noindent\rule{7in}{1.5pt}

%%%%%%%%%%%%%%%%%%%%%%%%%%%%%%%%%%%%%%%%%%%%%%%%%%%%%%%%%%%%%%%%%%%%%%%%%%%%%%%%%%%%%%%%%%%%%%%%%%%%%%%%%%%%%%%%%%%%%%%%%%%%%%%%%%%%%%%%

\begin{problem}{1.3}
Explain why it is not possible to construct a valid Feynman diagram using the Standard Model vertices for the following processes :
\begin{enumerate}[label=(\alph*)]
    \item $\mu^- \to e^+e^-e^+$
    \item $\nu_\tau + p \to \mu^- + n$
    \item $\nu_\tau + p \to \tau^+ + n$
    \item $\pi^+(u\overbar{d})+\pi^-(d\overbar{u}) \to n(udd) + \pi^0(u\overbar{u})$
\end{enumerate}
\end{problem}
\begin{solution}
\begin{enumerate}[label=(\alph*)]
    \item $\mu^- \to e^+e^-e^+$ : Charge is not conserved, as well as lepton numbers.
    \item $\nu_\tau + p \to \mu^- + n$ : Charge is not conserved, as well as baryon numbers.
    \item $\nu_\tau + p \to \tau^+ + n$ : Both baryon and lepton number is not conserved.
    \item $\pi^+(u\overbar{d})+\pi^-(d\overbar{u}) \to n(udd) + \pi^0(u\overbar{u})$ : Baryon number is not conserved.
\end{enumerate}
\end{solution}

\noindent\rule{7in}{1.5pt}

%%%%%%%%%%%%%%%%%%%%%%%%%%%%%%%%%%%%%%%%%%%%%%%%%%%%%%%%%%%%%%%%%%%%%%%%%%%%%%%%%%%%%%%%%%%%%%%%%%%%%%%%%%%%%%%%%%%%%%%%%%%%%%%%%%%%%%%%

\begin{problem}{1.4}
Draw the Feynman diagram for the decays:
\begin{enumerate}[label=(\alph*)]
    \item $\Delta^+(uud)\to n(udd)\pi^+(u\overbar{d})$
    \item $\Sigma^0(uds)\to\Lambda(uds)\gamma$
    \item $\pi^+(u\overbar{d})\to \mu^+ \nu_\mu$
\end{enumerate}
\end{problem}
\begin{solution}
\begin{enumerate}[label=(\alph*)]
    \item $\Delta^+(uud)\to n(udd)\pi^+(u\overbar{d})$
    \begin{center}
        \scalebox{1.25}{
            \begin{tikzpicture}
                \begin{feynman}
                    \vertex (a1) {\(u\)};
                    \vertex[right=2.5cm of a1] (a2t);
                    \vertex[left=1.cm of a2t] (a2l);
                    \vertex[right=1.cm of a2t] (a2r);
                    \vertex[right=5cm of a1] (a3) {\(d\)};

                    \vertex[below=0.75em of a1] (b1) {\(u\)};
                    \vertex[below=0.75em of a3] (b3) {\(u\)};

                    \vertex[below=0.75em of b1] (d1) {\(d\)};
                    \vertex[below=0.75em of b3] (d3) {\(d\)};

                    \vertex[above=of a2r] (c3t);
                    \vertex[left=0.25em of c3t] (c3){\( \overline d \)};
                    %\vertex[right=1em of c1t] (c1){\(\overline d\)};
                    \vertex[above=of a2l] (c1t) ;
                    \vertex[right=0.15em of c1t] (c1) {\(u\)};

                    \vertex[above=.75em of c1t] (c1tu);
                    \vertex[above=.75em of c3t] (c3tu);

                    \diagram* {
                      {[edges=fermion]
                        (a1) -- (a2l) -- (c1), (a2r) -- (a3),
                        (b1) -- (b3),
                        (d1) -- (d3),
                        (a2r) -- (c3), 
                      },

                      (a2l) -- [gluon] (a2r),
                      
                    };

                    \draw [decoration={brace}, decorate] (c1tu) -- (c3tu)
                          node [pos=0.5, above] {\(\pi^{+}\)};
                    \draw [decoration={brace}, decorate] (d1.south west) -- (a1.north west)
                          node [pos=0.5, left] {\(\Delta^{+}\)};
                    \draw [decoration={brace}, decorate] (a3.north east) -- (d3.south east) 
                          node [pos=0.5, right] {\(n\)};
                \end{feynman}
            \end{tikzpicture}
        }
    \end{center}
    \item $\Sigma^0(uds)\to\Lambda(uds)\gamma$ 
    \begin{center}
        \scalebox{1.25}{
            \begin{tikzpicture}
                \begin{feynman}
                    \vertex (a1) {\(u\)};
                    \vertex[right=2.5cm of a1] (a2t);
                    \vertex[left=1.cm of a2t] (a2l);
                    \vertex[right=1.cm of a2t] (a2r);
                    \vertex[right=5cm of a1] (a3) {\(u\)};

                    \vertex[below=0.75em of a1] (b1) {\(d\)};
                    \vertex[below=0.75em of a3] (b3) {\(d\)};

                    \vertex[below=0.75em of b1] (d1) {\(s\)};
                    \vertex[below=0.75em of b3] (d3) {\(s\)};

                    \vertex[above=of a2r] (c3t);
                    \vertex[left=0.25em of c3t] (c3){\( \gamma \)};
                    %\vertex[right=1em of c1t] (c1){\(\overline d\)};
                    \vertex[above=of a2l] (c1t) ;

                    \vertex[above=.75em of c1t] (c1tu);
                    \vertex[above=.75em of c3t] (c3tu);

                    \diagram* {
                      {[edges=fermion]
                        (a1) -- (a2) -- (a3),
                        (b1) -- (b3),
                        (d1) -- (d3),
                      },

                      (a2) -- [photon] (c3)
                      
                    };

                    \draw [decoration={brace}, decorate] (d1.south west) -- (a1.north west)
                          node [pos=0.5, left] {\(\Sigma^{0}\)};
                    \draw [decoration={brace}, decorate] (a3.north east) -- (d3.south east) 
                          node [pos=0.5, right] {\(\Lambda\)};
                \end{feynman}
            \end{tikzpicture}
        }
    \end{center} 
    \pagebreak
    \item $\pi^+(u\overbar{d})\to \mu^+ \nu_\mu$ \\[0.15in]
    \begin{center}
        \scalebox{1.25}{
            \begin{tikzpicture}
                \begin{feynman}
                   
                    \vertex (a1){\(u\)};
                    \vertex[left=1.5cm of a1] (a0u) {\(u\)};
                    \vertex[below=2.5em of a0u] (a0d) {\(\overline d\)};
                    \vertex[below=2.5em of a1] (a3) {\(\overline d\)};
                    \vertex at ($(a3)!0.5!(a1) + (1.25cm, 0)$) (a2) ;

                    \vertex[right=3em of a2] (b2);

                    \vertex[right=3.5cm of a1] (b1) {\(\mu^+\)};
                    \vertex[right=3.5cm of a3] (b3) {\(\nu_\mu\)};

                    \diagram* {

                      (a0u) -- [fermion] (a1) -- [fermion,bend left] (a2) -- [fermion, bend left] (a3) -- [fermion] (a0d),
                      (a2) -- [boson, edge label =\(W^+\)] (b2), (b1) -- [fermion] (b2) -- [fermion] (b3)

                    };
                    \draw [decoration={brace}, decorate] (a0d.south west) -- (a0u.north west)
                          node [pos=0.5, left] {\(\pi^{+}\)};
                \end{feynman}
            \end{tikzpicture}
        }
    \end{center}
\end{enumerate}
\end{solution}

\noindent\rule{7in}{1.5pt}

%%%%%%%%%%%%%%%%%%%%%%%%%%%%%%%%%%%%%%%%%%%%%%%%%%%%%%%%%%%%%%%%%%%%%%%%%%%%%%%%%%%%%%%%%%%%%%%%%%%%%%%%%%%%%%%%%%%%%%%%%%%%%%%%%%%%%%%%

\begin{problem}{1.5}
Treating the $\pi^0$ as a $uu$ bound state, draw the Feynman diagrams for:
\begin{enumerate}[label=(\alph*)]
    \item $\pi^0\to\gamma\gamma$
    \item $\pi^0\to\gamma e^+e^-$
    \item $\pi^0\to e^+e^-e^+e^-$
    \item $\pi^0\to e^+e^-$
\end{enumerate}
\end{problem}
\begin{solution}
    \begin{enumerate}[label=(\alph*)]
        \item $\pi^0\to\gamma\gamma$
        \begin{center}
            \scalebox{1.5}{
                \begin{tikzpicture}
                    \begin{feynman}
                       
                        \vertex (a1) {\(u\)};
                        \vertex[right=3.75em of a1] (a2u);
                        \vertex[below=2em of a2u] (a2d);
                        \vertex[below=2em of a1] (a3) {\(\overline u\)};
    
                        \vertex[right=3.75em of a2u] (g1t);
                        \vertex[right=3.75em of a2d] (g2t);
                        \vertex[above=1em of g1t] (g1) {\(\gamma\)};
                        \vertex[below=1em of g2t] (g2) {\(\gamma\)};
    
                        \diagram* {   
                          (a1) -- [fermion] (a2u) -- [fermion] (a2d) -- [fermion] (a3),
                          (a2u) -- [photon] (g1) , (a2d) -- [photon] (g2)
                        };

                        \draw [decoration={brace}, decorate] (a3.south west) -- (a1.north west)
                              node [pos=0.5, left] {\(\pi^{0}\)};
                    \end{feynman}
                \end{tikzpicture}
            }
        \end{center}
        \item $\pi^0\to\gamma e^+e^-$
        \begin{center}
            \scalebox{1.5}{
                \begin{tikzpicture}
                    \begin{feynman}
                       
                        \vertex (a1) {\(u\)};
                        \vertex[right=3.75em of a1] (a2u);
                        \vertex[below=2em of a2u] (a2d);
                        \vertex[below=2em of a1] (a3) {\(\overline u\)};
    
                        %\vertex[right=2.25em of a2u] (g1u);
                        \vertex at ($ (a2u) + (5.25em,0.5em) $) (g1u_t) {\(\gamma\)};
                        \vertex at ($ (a2u) + (2.25em,0.75em) $) (g1u);
                        \vertex at ($ (a2d) + (2.25em,-0.75em) $) (g2d);
                        %\vertex[right=2.25em of a2d] (g2d);

                        \vertex at ($ (g1u) + (3em,1.5em) $) (e1);
                        \vertex at ($ (g2d) + (3em,-1.5em) $) (e4) {\(e^+\)};
                        \vertex at ($ (e1)!.33!(e4) $) (e2);
                        \vertex at ($ (e1)!.66!(e4) $) (e3) {\(e^-\)};
    
                        \diagram* {   
                          (a1) -- [fermion] (a2u) -- [fermion] (a2d) -- [fermion] (a3),
                          (a2u) -- [photon] (g1u_t) , (a2d) -- [photon] (g2d) ,
                          (e3) -- [fermion] (g2d) -- [fermion]  (e4)
                        };

                        \draw [decoration={brace}, decorate] (a3.south west) -- (a1.north west)
                              node [pos=0.5, left] {\(\pi^{0}\)};
                    \end{feynman}
                \end{tikzpicture}
            }
        \end{center}
        \item $\pi^0\to e^+e^-e^+e^-$
        \begin{center}
            \scalebox{1.5}{
                \begin{tikzpicture}
                    \begin{feynman}
                       
                        \vertex (a1) {\(u\)};
                        \vertex[right=3.75em of a1] (a2u);
                        \vertex[below=2em of a2u] (a2d);
                        \vertex[below=2em of a1] (a3) {\(\overline u\)};
    
                        %\vertex[right=2.25em of a2u] (g1u);
                        \vertex at ($ (a2u) + (2.25em,0.75em) $) (g1u);
                        \vertex at ($ (a2d) + (2.25em,-0.75em) $) (g2d);
                        %\vertex[right=2.25em of a2d] (g2d);

                        \vertex at ($ (g1u) + (3em,1.5em) $) (e1) {\(e^-\)};
                        \vertex at ($ (g2d) + (3em,-1.5em) $) (e4) {\(e^+\)};
                        \vertex at ($ (e1)!.33!(e4) $) (e2) {\(e^+\)};
                        \vertex at ($ (e1)!.66!(e4) $) (e3) {\(e^-\)};
    
                        \diagram* {   
                          (a1) -- [fermion] (a2u) -- [fermion] (a2d) -- [fermion] (a3),
                          (a2u) -- [photon] (g1u) , (a2d) -- [photon] (g2d) ,
                          (e1) -- [fermion] (g1u) -- [fermion]  (e2) , (e3) -- [fermion] (g2d) -- [fermion]  (e4)
                        };

                        \draw [decoration={brace}, decorate] (a3.south west) -- (a1.north west)
                              node [pos=0.5, left] {\(\pi^{0}\)};
                    \end{feynman}
                \end{tikzpicture}
            }
        \end{center}
        \item $\pi^0\to e^+e^-$
        \begin{center}
            \scalebox{1.25}{
                \begin{tikzpicture}
                    \begin{feynman}
                       
                        \vertex (a1){\(u\)};
                        \vertex[below=2.5em of a1] (a3) {\(\overline u\)};
                        \vertex at ($(a3)!0.5!(a1) + (1.5cm, 0)$) (a2) ;
    
                        \vertex[right=3em of a2] (b2);
    
                        \vertex at ($(a2) + (6em,2.5em)$) (b1) {\(e^-\)};
                        \vertex at ($(a2) + (6em,-2.5em)$) (b3) {\(e^+\)};
                        %\vertex[right=4.5cm of a1] (b1) {\(\mu^+\)};
                        %\vertex[right=4.5cm of a3] (b3) {\(\nu_\mu\)};
    
                        \diagram* {
    
                          (a1) -- [fermion,bend left] (a2) -- [fermion, bend left] (a3) ,
                          (a2) -- [boson, edge label =\(\gamma\)] (b2), (b1) -- [fermion] (b2) -- [fermion] (b3)
    
                        };
                        \draw [decoration={brace}, decorate] (a3.south west) -- (a1.north west)
                              node [pos=0.5, left] {\(\pi^{0}\)};
                    \end{feynman}
                \end{tikzpicture}
            }
        \end{center}
    \end{enumerate}
\end{solution}

\noindent\rule{7in}{1.5pt}

%%%%%%%%%%%%%%%%%%%%%%%%%%%%%%%%%%%%%%%%%%%%%%%%%%%%%%%%%%%%%%%%%%%%%%%%%%%%%%%%%%%%%%%%%%%%%%%%%%%%%%%%%%%%%%%%%%%%%%%%%%%%%%%%%%%%%%%%

\begin{problem}{1.6}
Particle interactions fall into two main categories, scattering processes and annihilation processes, as indicated by the Feynman diagrams below.
\begin{center}
    \scalebox{1.5}{
    \begin{tikzpicture}
        \begin{feynman}
            \vertex (a1);
            \vertex[below=4em of a1] (a3);
            \vertex[right=1.5em of a3] (a2t);
            \vertex[above=2em of a2t] (a2);
    
            \vertex[right=2em of a2] (b2);
            \vertex[right=5em of a1] (b1);
            \vertex[right=5em of a3] (b3);
    
            \diagram* {
                {[edges=fermion]
                  (a1) -- (a2) -- (a3),
                  (b3) -- (b2) -- (b1)
                },
                (a2) -- [boson] (b2)
            };
        \end{feynman}
    \end{tikzpicture} \hspace{0.15in}
    \begin{tikzpicture}
        \begin{feynman}
            \vertex (a1);
            \vertex[right=5em of a1] (a3);
            \vertex[right=2.5em of a1] (a2t);
            \vertex[below=1em of a2t] (a2);
    
            \vertex[below=4em of a1] (b1);
            \vertex[right=5em of b1] (b3);
            \vertex[right=2.5em of b1] (b2t);
            \vertex[above=1em of b2t] (b2);

            \diagram* {
                {[edges=fermion]
                  (a1) -- (a2) -- (a3),
                  (b1) -- (b2) -- (b3)
                },
                (a2) -- [boson] (b2)
            };
            
        \end{feynman}
    \end{tikzpicture}
    }
\end{center}
Draw the lowest-order Feynman diagrams for the scattering and/or annihilation processes:
\begin{enumerate}[label=(\alph*)]
    \item $e^-e^- \to e^-e^-$
    \item $e^+e^- \to \mu^+\mu^-$
    \item $e^+e^- \to e^+e^-$
    \item $e^-\nu_e \to e^-\nu_e$
    \item $e^-\overbar{\nu_e} \to e^-\overbar{\nu_e}$
\end{enumerate}
\end{problem}
\begin{solution}
\begin{enumerate}[label=(\alph*)]
    \item $e^-e^- \to e^-e^-$
    \begin{center}
        \scalebox{1.5}{
            \begin{tikzpicture}
                \begin{feynman}
                    \vertex (a2) ;
                    \vertex at ($(a2) + (-3.5em,1em) $) (a1) {$e^-$};
                    \vertex at ($(a2) + (3.5em,1em) $) (a3) {$e^-$};
            
                    \vertex[below=3.75em of a2] (b2);        
                    \vertex at ($(b2) + (-3.5em,-1em) $) (b1) {$e^-$};
                    \vertex at ($(b2) + (3.5em,-1em) $) (b3) {$e^-$};
            
                    \diagram* {
                        {[edges=fermion]
                          (a1) -- (a2) -- (a3),
                          (b1) -- (b2) -- (b3)
                        },
                        (a2) -- [boson, edge label =\(\gamma\)] (b2)
                    };
                    
                \end{feynman}
            \end{tikzpicture}
        }
    \end{center}
    \item $e^+e^- \to \mu^+\mu^-$
    \begin{center}
        \scalebox{1.25}{
            \begin{tikzpicture}
                \begin{feynman}
                    
                    \vertex (g1);
                    \vertex at ($(g1) + (-3.5em,3.5em) $) (e1) {\(e^-\)};
                    \vertex at ($(g1) + (-3.5em,-3.5em) $) (e2) {\(e^+\)};

                    \vertex at ($(g1) + (3em,0)$) (g2) ;
                    \vertex at ($(g2) + (3.5em,3.5em) $) (m1) {\(\mu^-\)};
                    \vertex at ($(g2) + (3.5em,-3.5em) $) (m2) {\(\mu^+\)};
                    \diagram* {
                        {[edges=fermion]
                          (e1) -- (g1) -- (e2),
                          (m1) -- (g2) -- (m2)
                        },
                        (g1) -- [boson, edge label =\(\gamma\)] (g2)
                    };

                \end{feynman}
            \end{tikzpicture}
        }
    \end{center}
    \item $e^+e^- \to e^+e^-$
    \item $e^-\nu_e \to e^-\nu_e$
    \item $e^-\overbar{\nu_e} \to e^-\overbar{\nu_e}$
\end{enumerate}
\end{solution}

\noindent\rule{7in}{1.5pt}

%%%%%%%%%%%%%%%%%%%%%%%%%%%%%%%%%%%%%%%%%%%%%%%%%%%%%%%%%%%%%%%%%%%%%%%%%%%%%%%%%%%%%%%%%%%%%%%%%%%%%%%%%%%%%%%%%%%%%%%%%%%%%%%%%%%%%%%%

\begin{problem}{1.7}
High-energy muons traversing matter lose energy according to
\begin{align*}
    -\frac{1}{\rho} \frac{\dif E}{\dif x} \approx a + bE 
\end{align*}
where $a$ is due to ionisation energy loss and $b$ is due to the bremsstrahlung and $e^+e^-$ pair-production processes. 
For standard rock, taken to have $A = 22, Z = 11$ and $\rho = 2.65 \unit{\gram\centi\metre^{-3}}$ , the parameters $a$ and $b$ depend only weakly on 
the muon energy and have values $a \approx 2.5 \MeV\unit{\gram^{-1}\centi\metre^2}$ and $b \approx \num{3.5e-6} \unit{\gram^{-1}\centi\metre^2}$.
\begin{enumerate}[label=(\alph*)]
    \item At what muon energy are the ionisation and bremsstrahlung/pair production processes equally important?
    \item Approximately how far does a 100 GeV cosmic-ray muon propagate in rock?
\end{enumerate}
\end{problem}
\begin{solution}
%Using the given approximation of $\dif E / \dif x$, one could solve the differential equation of $E(x)$ as
%\begin{align*}
%    -\frac{1}{\rho} \frac{\dif E}{\dif x} \approx a + bE  &\implies - \frac{\dif E}{\rho \left( a+bE \right)} \approx \dif x  \\[0.15in]
%                                                          &\implies  E(x) \approx \frac{c}{\rho b} e^{\rho b x} - \frac{a}{b}
%\end{align*}
\begin{enumerate}[label=(\alph*)]
    \item One could assume that ionisation and bremsstrahlung/pair production processes become equally important for a certain energy scale $E^\ast$ when $a \simeq bE^\ast$.
    Such $E^\ast \simeq a/b$ can be calculated as $\sim 700 \GeV$.
    \item Using the values given, 

    \begin{align*}
        - \frac{\dif E}{\dif x} &\approx a \rho + b \rho  E \impliedby \left( a\rho \sim \num{6.6}\MeV/\text{cm} \text{ , } b \rho \sim \num{9.275e-6}/\text{cm} \right) \\[0.15in]
                                &\simeq 7.52 \MeV/\text{cm} 
    \end{align*}\\
    which shows that a 100 GeV muon will go through around 132 metres of rock.
\end{enumerate}
\end{solution}

\noindent\rule{7in}{1.5pt}

%%%%%%%%%%%%%%%%%%%%%%%%%%%%%%%%%%%%%%%%%%%%%%%%%%%%%%%%%%%%%%%%%%%%%%%%%%%%%%%%%%%%%%%%%%%%%%%%%%%%%%%%%%%%%%%%%%%%%%%%%%%%%%%%%%%%%%%%

\begin{problem}{1.8}
Tugsten has a radiation length of $X_0=0.35$ cm and a critical energy of $E_c = 7.97$ MeV. Roughly what thickness of tungsten is required to fully contain a 500 GeV electromagnetic shower from an electron?
\end{problem}
\begin{solution}
Getting $x_\text{max}$ for the given situation, one obtains :

\begin{align*}
    x_{\text{max}} = \frac{1}{\ln 2}\ln\left( \frac{E}{E_c}\right) = \frac{1}{\ln 2}\ln\left( \frac{500 \GeV}{7.97 \MeV}\right) \sim 16
\end{align*}\\
Thus, roughly around $x_\text{max}X_0 \simeq 5.6 \unit{\centi\metre}$ of tungsten would be able to contain a 500 GeV electromagnetic shower from an electron.\\
\end{solution} 
\noindent\rule{7in}{1.5pt}

%%%%%%%%%%%%%%%%%%%%%%%%%%%%%%%%%%%%%%%%%%%%%%%%%%%%%%%%%%%%%%%%%%%%%%%%%%%%%%%%%%%%%%%%%%%%%%%%%%%%%%%%%%%%%%%%%%%%%%%%%%%%%%%%%%%%%%%%

\begin{problem}{1.9}
The CPLEAR detector consisted of: tracking detectors in a magnetic field of 0.44 T; and electromagnetic calorimeter;
and Čerenkov detectors with a radiator of refractive index $n=1.25$ used to distinguish $\pi^\pm$ from $K^\pm$.

A charged particle travelling perpendicular to the direction of the magnetic field leaves a track with a
measured radius of curvature of $R=4$m. If it is observed to give a Čerenkov signal, is it 
possible to distinguish between the particle being a pion or kaon? Take $m_\pi \approx 140 \MeV /\text{c}^2$ and $m_K \approx 494 \MeV /\text{c}^2$ 
\end{problem}
\begin{solution}
First, the momentum could be extracted from the fact that the charged particles are travelling perpendicular ($\lambda =0 $) to the 0.44 T magnetic field, 
which eventually gives $p=0.3BR=0.528\GeV$. The threshold mass for Čerenkov radiation in this case would be,

\begin{align*}
    \sqrt{n^2-1} p = 0.75 \times p = 0.396 \GeV
\end{align*}\\
thus in such situation it would be able to identity whether the track corresponds to a pion or kaon, as only $m_\pi$ is smaller than the threshold Čerenkov radiation mass.
\end{solution}

\noindent\rule{7in}{1.5pt}

%%%%%%%%%%%%%%%%%%%%%%%%%%%%%%%%%%%%%%%%%%%%%%%%%%%%%%%%%%%%%%%%%%%%%%%%%%%%%%%%%%%%%%%%%%%%%%%%%%%%%%%%%%%%%%%%%%%%%%%%%%%%%%%%%%%%%%%%

\begin{problem}{1.10}
In a fixed-target pp experiment, what proton energy would be required to achieve the same centre-of-mass energy as the LHC, which will ultimately operate at 14 TeV.
\end{problem}
\begin{solution}
Let the four-momentum of the beam proton and the fixed target proton as $p_1 = (E,0,0,p)$ and $p_2 = (m_p,0,0,0)$. Using the following expression of the centre-of-mass energy $\sqrt{s}$, the proton energy $E$ to satisfy the required situation would be :

\begin{align*}
    \sqrt{s} = (p_1 + p_2) ^2 &= 2m_p^2 + 2 p_1 \cdot p_2 \\[0.15in]
                              &= 2m_p \left( m_p + E \right) = 14 \text{ TeV} \implies \boxed{E \simeq 7.4 \text{ PeV}}
\end{align*}
\end{solution} 

\noindent\rule{7in}{1.5pt}
    
%%%%%%%%%%%%%%%%%%%%%%%%%%%%%%%%%%%%%%%%%%%%%%%%%%%%%%%%%%%%%%%%%%%%%%%%%%%%%%%%%%%%%%%%%%%%%%%%%%%%%%%%%%%%%%%%%%%%%%%%%%%%%%%%%%%%%%%%

\begin{problem}{1.11}
At the LEP $e^+e^-$ collider, which had a circumference of 27 km, the electron and positron beam currents were both 1.0 mA. Each beam consisted of four equally spaced bunches of electrons/positrons. The bunches had an effective area of $\num{1.8e4}\unit{\micro\metre^2}$.
Calculate the instantaneous luminosity on the assumption that the beams collided head-on.
\end{problem}
\begin{solution}
The instantaneous luminosity $\mathcal{L}$ could be computed as,

\begin{align*}
    \mathcal{L} = f\frac{n_1n_2}{4\pi \sigma_x \sigma_y}
\end{align*}\\
As the problem states, the given effective area $\num{1.8e4} \unit{\micro\meter^2}=\num{1.8e-4} \unit{\centi\meter^2}$ of the bunches will correspond to $\sigma_x \sigma_y$. In this case, the bunches are seperated $27/4 = 6.75 \unit{km}$ and as the leptons were accelerated $\sim 0.99c$, the temporal separation between the beams will be $\simeq 5 \unit{\micro\second}$ where relativistic effects are folded in, thus the collision frequency will be $f \simeq 200 \unit{\kilo\hertz}$. Using the beam current $\num{1.0e-3}\unit{C\cdot s^{-1}}$, the number of electrons can be derived using

\begin{align*}
    n_1 = n_2 = \frac{\num{1.0}\unit{\milli\ampere}}{ef}=\frac{\num{1.0e-3}\unit{C\cdot s^{-1}}}{\num{1.62e-19}\unit{C}} \times 5 \unit{\micro\second} \simeq \num{3.5e10}
\end{align*}\\
Then using all the numbers derived, the instantaneous luminosity can be calculated as,

\begin{align*}
    \mathcal{L} = \num{2.0e5} \unit{s^{-1}} \times \frac{\left( \num{3.5e10}\right)^2}{\num{1.8e-4}\unit{\centi\meter^2}} \simeq \num{1.3e30} \unit{cm^{-2}\cdot s^{-1}}
\end{align*}
\end{solution}\\