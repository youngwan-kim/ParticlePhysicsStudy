\noindent\rule{7in}{2.8pt}
\section{Decay Rates and Cross Sections}

\begin{problem}{3.1}
Calculate the energy of the $\mu^-$ produced in the decay at rest $\pi^- \rightarrow \mu\bar{\nu}_{\mu}$.
Assume $m_{\pi} = \mathrm{140 \ MeV}$, $m_{\mu} = \mathrm{106 \ MeV}$ and take $m_{\nu} \sim 0$.
\end{problem}

\begin{solution}
Let the four-momenta of the muon and the neutrino to be $p_1 = (E_1, 0, 0, E_2)$ and $p_2 = (E_2, 0, 0, -E_2)$.
In the pion rest frame, $E_1 + E_2 = m_{\pi}$ and from the muon mass constraint $m_{\mu}^2 = E_1^2 - E_2^2$.
Solving these equation gives 
$$E_1 = {m_{\pi}^2 + m_{\mu}^2 \over 2m_{\pi}} = \mathrm{110.13 \ GeV}$$.

\end{solution} 
\noindent\rule{7in}{1.5pt}

%%%%%%%%%%%%%%%%%%%%%%%%%%%%%%%%%%%%%%%%%%%%%%%%%%%%%%%%%%%%%%%%%%%%%%%%%%%%%%%%%%%%%%%%%%%%%%%%%%%%%%%%%%%%%%%%%%%%%%%%%%%%%%%%%%%%%%%%

\begin{problem}{3.2}
For the decay $a \rightarrow 1 + 2$, show that the momenta of both daughter particles
in the centre-of mass frame $p^*$ are
$$p^* = {1 \over 2m_a}\sqrt{[m_a^2 - (m_1^2 + m_2^2)][m_a^2 - (m_1^2 - m_2^2)]}$$.
\end{problem}

\begin{solution}
Let the four-momenta of the mother particle and the daughter particles to be
$p_a = (m_a, 0, 0, 0)$, $p_1 = (E_1, 0, 0, p^*)$, $p_2 = (E_2, 0, 0, -p^*)$
From the mass constraints, we get $E_1 + E_2 = m_a$, $E_1^2 - p^{*2} = m_1^2$, and 
$E_2^2 - p^{*2} = m_2^2$.

Since we have three unknown variables $E_1, E_2, p^*$ and three equations,
it is possible to get $p^*$ in terms of $m_a, m_1 \ \mathrm{and} \ m_2$, which gives
the desired solution.

\end{solution} 
\noindent\rule{7in}{1.5pt}

%%%%%%%%%%%%%%%%%%%%%%%%%%%%%%%%%%%%%%%%%%%%%%%%%%%%%%%%%%%%%%%%%%%%%%%%%%%%%%%%%%%%%%%%%%%%%%%%%%%%%%%%%%%%%%%%%%%%%%%%%%%%%%%%%%%%%%%%

\begin{problem}{3.3}
Calculate the branching ratio for the decay $K^+\to\pi^+\pi^0$, given the partial decay width $\Gamma(K^+\to\pi^+\pi^0)=\num{1.2e-8} \text{ eV}$
and the mean kaon lifetime $\tau(K^+)=\num{1.2e-8}$ s.
\end{problem}
    
\begin{solution}
Using the given information, 
\begin{align*}
    \text{BR}\left(K^+\to\pi^+\pi^0\right) &= \frac{1}{\Gamma_{K^+}} \times \Gamma\left( K^+ \to \pi^+\pi^0 \right) \\[0.15in]
                                           &= \tau( K^+ )\times \Gamma\left( K^+ \to \pi^+\pi^0 \right) \\[0.15in]
                                           &= \left( \num{1.2e-8}\unit{ s} \right) \times \left( \num{1.2e-8} \text{eV}\right) \\[0.15in]
                                           &= \left( \num{1.2e-8} \right)  \times \left( \frac{1}{6.58} \times \num{e16} \text{ eV}^{-1} \right)  \times \left( \num{1.2e-8} \text{eV}\right) \\[0.15in]
                                           &= \frac{1.2^2}{6.58} \simeq 21\%
\end{align*}
which is as much as expected from the known branching rate.\\
\end{solution} 
\noindent\rule{7in}{1.5pt}
    
%%%%%%%%%%%%%%%%%%%%%%%%%%%%%%%%%%%%%%%%%%%%%%%%%%%%%%%%%%%%%%%%%%%%%%%%%%%%%%%%%%%%%%%%%%%%%%%%%%%%%%%%%%%%%%%%%%%%%%%%%%%%%%%%%%%%%%%%

\begin{problem}{3.4}
At a future $e^+e^-$ linear collider operating as a Higgs factory at a centre-of-mass energy of $\sqrt{s}=250\GeV$, the cross section
for the process $e^+e^-\to HZ$ is 250 fb. If the collider has an instantaneous luminosity of $\num{2e34}\unit{\centi\metre^{-2}\second^{-1}}$
and is operational for $50\%$ of the time, how many Higgs bosons will be produced in five years of running?
\end{problem}
        
\begin{solution}
Let the total number of Higgs bosons that will be produced in 5 years of running in such condition as $N$, then one could calculate $N$ as,
\begin{align*}
    N &= \left( \num{2e34}\unit{\centi\metre^{-2}\second^{-1}} \right) \times \left( 5 \text{ yrs} \right) \times \left( 250 \text{ fb} \right) \times 0.5 \\[0.15in]
      &= \left( \num{2e34}\unit{\centi\metre^{-2}\second^{-1}} \right) \left( \num{1.5768e8} \unit{\second} \right) \times \left( \num{2.5e-42} \unit{\centi\metre^2} \right) \times 0.5 \\[0.15in]
      &= \num{3.942}
\end{align*}
\end{solution} 

\noindent\rule{7in}{1.5pt}

%%%%%%%%%%%%%%%%%%%%%%%%%%%%%%%%%%%%%%%%%%%%%%%%%%%%%%%%%%%%%%%%%%%%%%%%%%%%%%%%%%%%%%%%%%%%%%%%%%%%%%%%%%%%%%%%%%%%%%%%%%%%%%%%%%%%%%%%

\begin{problem}{3.5}
The total $e^+e^-\to\gamma\to\mu^+\mu^-$ annihilation cross section is $\sigma = 4\pi\alpha^2/3s$, where $\alpha \simeq 1/137$. 
Calculate the cross section at $\sqrt{s}=50\GeV$, expressing your answer in both natural units and in barns.
Compare this to the total pp cross section at $\sqrt{s}=50\GeV$ which is approximately 40 mb and comment on the result.
\end{problem}
\begin{solution}
Plugging in all the values we know in natural units ,
\begin{align*}
    \sigma = \frac{4\pi}{3\cdot \left( \num{2.5e3} \GeV^2 \right)\cdot 137^2} = \num{8.9e-8} \GeV^{-2}
\end{align*}
which could be converted into barns using $1\text{ GeV}^{-2} = 0.3894 \text{ mb}$ which gives $\sigma $
\end{solution} 
    
\noindent\rule{7in}{1.5pt}
    
%%%%%%%%%%%%%%%%%%%%%%%%%%%%%%%%%%%%%%%%%%%%%%%%%%%%%%%%%%%%%%%%%%%%%%%%%%%%%%%%%%%%%%%%%%%%%%%%%%%%%%%%%%%%%%%%%%%%%%%%%%%%%%%%%%%%%%%%


\begin{problem}{3.6}
A $1\GeV$ muon neutrino is fired at a $1\unit{\metre}$ thick block of iron with density $\rho=\num{7.874e3}\unit{\kilo\gram\cdot\metre^{-3}}$.
If the average neutrino-nucleon interaction cross section is $\sigma=\num{8e-39}\unit{\square\metre}$, calculate the (small) probability that the neutrino interacts in the block.
\end{problem}
            
\begin{solution}
        
            
\end{solution} 
    
\noindent\rule{7in}{1.5pt}
    
%%%%%%%%%%%%%%%%%%%%%%%%%%%%%%%%%%%%%%%%%%%%%%%%%%%%%%%%%%%%%%%%%%%%%%%%%%%%%%%%%%%%%%%%%%%%%%%%%%%%%%%%%%%%%%%%%%%%%%%%%%%%%%%%%%%%%%%%


\begin{problem}{3.7}
For the process $a+b\to1+2$ the Lorents-invariant flux term installed
\begin{align*}
    F = 4 \left[ \left( p_a \cdot p_b \right)^2 - m_a^2 m_b^2  \right]^{\frac{1}{2}}
\end{align*}
In the non-relativistic limit, $\beta_a,\beta_b \ll 1$, show that
\begin{align*}
    F \approx 4m_am_b \left| \mathbf{v}_a - \mathbf{v}_b \right|
\end{align*}
where $\mathbf{v}_a,\mathbf{v}_b$ are the (non-relativistic) velocities of the two particles.
\end{problem}
\begin{solution}
Let the four-momenta of a,b as $p_a = \left( E_a, \boldp_a\right)$ and $p_b = \left( E_b, \boldp_b\right)$.
Under the non-relativistic limit which implies that $\gamma_a,\gamma_b \simeq 1$, one could write down $F$ as
\begin{align*}
    F &=  4 \left[ \left( p_a \cdot p_b \right)^2 - m_a^2 m_b^2  \right]^{\frac{1}{2}} \\[0.15in]
      &=  4 \left[ \left( E_aE_b - m_am_b \boldsymbol{\beta}_a\cdot\boldsymbol{\beta}_b\right)^2 - m_a^2 m_b^2  \right]^{\frac{1}{2}} \\[0.15in]
      &=  4 \left[ \left( E_aE_b - m_am_b \boldsymbol{\beta}_a\cdot\boldsymbol{\beta}_b\right)^2 - m_a^2 m_b^2  \right]^{\frac{1}{2}} \\[0.15in]
\end{align*}
\end{solution} 
    
\noindent\rule{7in}{1.5pt}
    
%%%%%%%%%%%%%%%%%%%%%%%%%%%%%%%%%%%%%%%%%%%%%%%%%%%%%%%%%%%%%%%%%%%%%%%%%%%%%%%%%%%%%%%%%%%%%%%%%%%%%%%%%%%%%%%%%%%%%%%%%%%%%%%%%%%%%%%%


\begin{problem}{3.8}
The Lorentz-invariant flux term for the process $a+b\to1+2$ in the centre-of-mass frame was shown to be 
$F=4p_i^\ast\sqrt{s}$, where $p_i^\ast$ is the momentum of the initial-state particles. Show that the 
corresponding expression in the frame where $b$ is at rest is 
\begin{align*}
    F = 4m_bp_a.
\end{align*}
\end{problem}
            
\begin{solution}
        
            
\end{solution} 
    
\noindent\rule{7in}{1.5pt}
    
%%%%%%%%%%%%%%%%%%%%%%%%%%%%%%%%%%%%%%%%%%%%%%%%%%%%%%%%%%%%%%%%%%%%%%%%%%%%%%%%%%%%%%%%%%%%%%%%%%%%%%%%%%%%%%%%%%%%%%%%%%%%%%%%%%%%%%%%

\begin{problem}{3.9}
Show that the momentum in the centre-of-mass frame of the initial-state particles in a two-body scattering
process can be expressed as
\begin{align*}
    p_i^{\ast 2} = \frac{1}{4s} \left[ s-\left(m_1+m_2\right)^2 \right]\left[ s-\left(m_1-m_2\right)^2 \right]
\end{align*}
\end{problem}
\begin{solution}

\end{solution}

\noindent\rule{7in}{1.5pt}

%%%%%%%%%%%%%%%%%%%%%%%%%%%%%%%%%%%%%%%%%%%%%%%%%%%%%%%%%%%%%%%%%%%%%%%%%%%%%%%%%%%%%%%%%%%%%%%%%%%%%%%%%%%%%%%%%%%%%%%%%%%%%%%%%%%%%%%%

\begin{problem}{3.10}
Repeat the calculation of Section 3.5.2 for the process $e^-p \to e^-p$ where the mass of the electron is no longer neglected.
\begin{enumerate}[label=(\alph*)]
    \item First show that
    \begin{align*}
        \frac{\dif E}{\dif \left( E \cos\theta \right)} = \frac{p_1p_3^2}{p_3\left(E_1+m_p\right)-E_3p_1\cos\theta}
    \end{align*}
    \item Then show that
    \begin{align*}
        \frac{\dif \sigma}{\dif \Omega} = \frac{1}{64\pi^2} \cdot \frac{p_3^2}{p_1m_p} \cdot \frac{1}{p_3\left(E_1+m_p\right)-E_3p_1\cos\theta} \cdot \left| \mathcal{M}_{fi} \right|^2
    \end{align*}
\end{enumerate}
\end{problem}
\begin{solution}

\end{solution}

\noindent\rule{7in}{1.5pt}

