\noindent\rule{7in}{2.8pt}
\section{Decay Rates and Cross Sections}

\begin{problem}{3.1}
Calculate the energy of the $\mu^-$ produced in the decay at rest $\pi^- \rightarrow \mu\bar{\nu}_{\mu}$.
Assume $m_{\pi} = \mathrm{140 \ MeV}$, $m_{\mu} = \mathrm{106 \ MeV}$ and take $m_{\nu} \sim 0$.
\end{problem}

\begin{solution}
Let the four-momenta of the muon and the neutrino to be $p_1 = (E_1, 0, 0, E_2)$ and $p_2 = (E_2, 0, 0, -E_2)$.
In the pion rest frame, $E_1 + E_2 = m_{\pi}$ and from the muon mass constraint $m_{\mu}^2 = E_1^2 - E_2^2$.
Solving these equation gives 
$$E_1 = {m_{\pi}^2 + m_{\mu}^2 \over 2m_{\pi}} = \mathrm{110.13 \ GeV}$$.

\end{solution} 
\noindent\rule{7in}{1.5pt}

%%%%%%%%%%%%%%%%%%%%%%%%%%%%%%%%%%%%%%%%%%%%%%%%%%%%%%%%%%%%%%%%%%%%%%%%%%%%%%%%%%%%%%%%%%%%%%%%%%%%%%%%%%%%%%%%%%%%%%%%%%%%%%%%%%%%%%%%

\begin{problem}{3.2}
For the decay $a \rightarrow 1 + 2$, show that the momenta of both daughter particles
in the centre-of mass frame $p^*$ are
$$p^* = {1 \over 2m_a}\sqrt{[m_a^2 - (m_1^2 + m_2^2)][m_a^2 - (m_1^2 - m_2^2)]}$$.
\end{problem}

\begin{solution}
Let the four-momenta of the mother particle and the daughter particles to be
$p_a = (m_a, 0, 0, 0)$, $p_1 = (E_1, 0, 0, p^*)$, $p_2 = (E_2, 0, 0, -p^*)$
From the mass constraints, we get $E_1 + E_2 = m_a$, $E_1^2 - p^{*2} = m_1^2$, and 
$E_2^2 - p^{*2} = m_2^2$.

Since we have three unknown variables $E_1, E_2, p^*$ and three equations,
it is possible to get $p^*$ in terms of $m_a, m_1 \ \mathrm{and} \ m_2$, which gives
the desired solution.

\end{solution} 
\noindent\rule{7in}{1.5pt}

%%%%%%%%%%%%%%%%%%%%%%%%%%%%%%%%%%%%%%%%%%%%%%%%%%%%%%%%%%%%%%%%%%%%%%%%%%%%%%%%%%%%%%%%%%%%%%%%%%%%%%%%%%%%%%%%%%%%%%%%%%%%%%%%%%%%%%%%

\begin{problem}{3.3}
Calculate the branching ratio for the decay $K^+\to\pi^+\pi^0$, given the partial decay width $\Gamma(K^+\to\pi^+\pi^0)=\num{1.2e-8} \text{ eV}$
and the mean kaon lifetime $\tau(K^+)=\num{1.2e-8}$ s.
\end{problem}
    
\begin{solution}

    
\end{solution} 
\noindent\rule{7in}{1.5pt}
    
%%%%%%%%%%%%%%%%%%%%%%%%%%%%%%%%%%%%%%%%%%%%%%%%%%%%%%%%%%%%%%%%%%%%%%%%%%%%%%%%%%%%%%%%%%%%%%%%%%%%%%%%%%%%%%%%%%%%%%%%%%%%%%%%%%%%%%%%

\begin{problem}{3.4}
At a future $e^+e^-$ linear collider operating as a Higgs factory at a centre-of-mass energy of $\sqrt{s}=250\GeV$, the cross section
for the process $e^+e^-\to HZ$ is 250 fb. If the collider has an instantaneous luminosity of $\num{2e34}\unit{\centi\metre^{-2}\second^{-1}}$
and is operational for $50\%$ of the time, how many Higgs bosons will be produced in five years of running?
\end{problem}
        
\begin{solution}
    
        
\end{solution} 

\noindent\rule{7in}{1.5pt}

%%%%%%%%%%%%%%%%%%%%%%%%%%%%%%%%%%%%%%%%%%%%%%%%%%%%%%%%%%%%%%%%%%%%%%%%%%%%%%%%%%%%%%%%%%%%%%%%%%%%%%%%%%%%%%%%%%%%%%%%%%%%%%%%%%%%%%%%

\begin{problem}{3.5}
The total $e^+e^-\to\gamma\to\mu^+\mu^-$ annihilation cross section is $\sigma = 4\pi\alpha^2/3s$, where $\alpha \simeq 1/137$. 
Calculate the cross section at $\sqrt{s}=50\GeV$, expressing your answer in both natural units and in barns.
Compare this to the total pp cross section at $\sqrt{s}=50\GeV$ which is approximately 40 mb and comment on the result.
\end{problem}
            
\begin{solution}
        
            
\end{solution} 
    
\noindent\rule{7in}{1.5pt}
    
%%%%%%%%%%%%%%%%%%%%%%%%%%%%%%%%%%%%%%%%%%%%%%%%%%%%%%%%%%%%%%%%%%%%%%%%%%%%%%%%%%%%%%%%%%%%%%%%%%%%%%%%%%%%%%%%%%%%%%%%%%%%%%%%%%%%%%%%


\begin{problem}{3.6}
A $1\GeV$ muon neutrino is fired at a $1\unit{metre}$ thick block of iron with density $\rho=\num{7.874e3}\unit{\kilo\gram\cdot\metre^{-3}}$.
If the average neutrino-nucleon interaction cross section is $\sigma=\num{8e-39}\unit{\square\metre}$, calculate the (small) probability that the neutrino interacts in the block.
\end{problem}
            
\begin{solution}
        
            
\end{solution} 
    
\noindent\rule{7in}{1.5pt}
    
%%%%%%%%%%%%%%%%%%%%%%%%%%%%%%%%%%%%%%%%%%%%%%%%%%%%%%%%%%%%%%%%%%%%%%%%%%%%%%%%%%%%%%%%%%%%%%%%%%%%%%%%%%%%%%%%%%%%%%%%%%%%%%%%%%%%%%%%


\begin{problem}{3.7}

\end{problem}
            
\begin{solution}
        
            
\end{solution} 
    
\noindent\rule{7in}{1.5pt}
    
%%%%%%%%%%%%%%%%%%%%%%%%%%%%%%%%%%%%%%%%%%%%%%%%%%%%%%%%%%%%%%%%%%%%%%%%%%%%%%%%%%%%%%%%%%%%%%%%%%%%%%%%%%%%%%%%%%%%%%%%%%%%%%%%%%%%%%%%


\begin{problem}{3.8}

\end{problem}
            
\begin{solution}
        
            
\end{solution} 
    
\noindent\rule{7in}{1.5pt}
    
%%%%%%%%%%%%%%%%%%%%%%%%%%%%%%%%%%%%%%%%%%%%%%%%%%%%%%%%%%%%%%%%%%%%%%%%%%%%%%%%%%%%%%%%%%%%%%%%%%%%%%%%%%%%%%%%%%%%%%%%%%%%%%%%%%%%%%%%