
\noindent\rule{7in}{2.8pt}
\section{Electron-Positron Annihilation}
    
\begin{problem}{6.1}
Using the properties of the $\gamma$-matrices of (4.33) and (4.34), and the definition of $\gamma^5\equiv\gamma^0\gamma^1\gamma^2\gamma^3$, show that

\begin{align*}
    \left(\gamma^5\right)^2 = 1 \quad \text{,} \quad \gamma^{5\dagger} = i\gamma^5 \quad \text{and} \quad \gamma^5\gamma^\mu = - \gamma^\mu\gamma^5 \\
\end{align*}
\end{problem}
\begin{solution}
    \begin{enumerate}[label=(\alph*)]
        \item $\left(\gamma^5\right)^2 = 1 $
            \begin{align*}
                \left(\gamma^5\right)^2 &= -\gamma^0\gamma^1\gamma^2\gamma^3 \gamma^0\gamma^1\gamma^2\gamma^3 \\[0.12in]
                                        &= \left(-1\right)^4 \gamma^0\gamma^0 \gamma^1\gamma^2\gamma^3 \gamma^1\gamma^2\gamma^3 \\[0.12in]
                                        &= \left(-1\right)^6  \gamma^1  \gamma^1 \gamma^2\gamma^3\gamma^2\gamma^3 \\[0.12in]
                                        &= \left(-1\right)^8  \gamma^2\gamma^2 \gamma^3\gamma^3 = 1 \qed
            \end{align*}
        \item $\gamma^{5\dagger} = \gamma^5$
            \begin{align*}
                \gamma^{5\dagger} = \left(i\gamma^0\gamma^1\gamma^2\gamma^3\right)^\dagger &= -i\gamma^{3\dagger}\gamma^{2\dagger}\gamma^{1\dagger}\gamma^{0\dagger} \\[0.12in]
                &= -i \left(-1\right)^3 \gamma^3 \gamma^2 \gamma^1 \gamma^0 \\[0.12in]
                &= -i \left(-1\right)^3 \left(-1\right)^6 \gamma^0\gamma^1\gamma^2\gamma^3 = \gamma^5 \qed
            \end{align*}
        \item $\gamma^5\gamma^\mu = - \gamma^\mu\gamma^5$
            \begin{align*}
                \gamma^5\gamma^\mu &=  i\gamma^0\gamma^1\gamma^2\gamma^3 \gamma^\mu \\[0.12in]
                                   &= \left(-1\right)^3 \gamma^\mu i\gamma^0\gamma^1\gamma^2\gamma^3 = -\gamma^\mu \gamma^5 \qed \\
            \end{align*}
            Here the fact that $\mu$ will be one of $0,1,2,3$ is used to obtain the $\left(-1\right)^3$ factor, as one of the indices will be identical, thus not giving an additional $-1$ factor when switching the position of $\gamma^\mu$.
    \end{enumerate}
\end{solution}

\noindent\rule{7in}{1.5pt}

%%%%%%%%%%%%%%%%%%%%%%%%%%%%%%%%%%%%%%%%%%%%%%%%%%%%%%%%%%%%%%%%%%%%%%%%%%%%%%%%%%%%%%%%%%%%%%%%%%%%%%%%%%%%%%%%%%%%%%%%%%%%%%%%%%%%%%%%

\begin{problem}{6.2}
    Show that the chiral projection operators 

    \begin{align*}
        P_R = \frac{1}{2}\left(1+\gamma^5\right) \quad \text{and} \quad P_L = \frac{1}{2}\left(1-\gamma^5\right)
    \end{align*}\\
    satisfy 

    \begin{align*}
        P_R + P_L = 1 \comma P_R P_R = P_R \comma P_L P_L = P_L \andtxt P_LP_R = 0 \\
    \end{align*}
\end{problem}
\begin{solution}
    \begin{enumerate}[label=(\alph*)]
        \item $ P_R + P_L = 1$ : Trivial 
        \item $ P_R P_R = P_R$ 
        
            \begin{align*}
                P_R P_R  = \frac{1}{4} \left(1+\gamma^5\right) \left(1+\gamma^5\right) &= \frac{1}{4} \left(1+2\gamma^5+\gamma^5\gamma^5\right) = \frac{1}{4} \left(2+2\gamma^5\right) = P_R \qed \\
            \end{align*}
        \item $ P_L P_L = P_L$ 
        
            \begin{align*}
                P_L P_L  = \frac{1}{4} \left(1-\gamma^5\right) \left(1-\gamma^5\right) &= \frac{1}{4} \left(1-2\gamma^5+\gamma^5\gamma^5\right) = \frac{1}{4} \left(2-2\gamma^5\right) = P_L \qed\\
            \end{align*}
        \item $ P_L P_R = 0$
        
            \begin{align*}
                P_L P_R  = \frac{1}{4} \left(1-\gamma^5\right) \left(1+\gamma^5\right) = \frac{1}{4} \left( 1- \gamma^5\gamma^5\right) = 0 \qed \\
            \end{align*}
    \end{enumerate}
\end{solution}

\noindent\rule{7in}{1.5pt}

%%%%%%%%%%%%%%%%%%%%%%%%%%%%%%%%%%%%%%%%%%%%%%%%%%%%%%%%%%%%%%%%%%%%%%%%%%%%%%%%%%%%%%%%%%%%%%%%%%%%%%%%%%%%%%%%%%%%%%%%%%%%%%%%%%%%%%%%

\begin{problem}{6.3}
    Show that

    \begin{align*}
        \Lambda^+ = \frac{m+\slashed{p}}{2m} \andtxt \Lambda^- = \frac{m-\slashed{p}}{2m}
    \end{align*}\\
    are also projection operators, and show that they respectively project out particle and antiparticle states, i.e.

    \begin{align*}
        \Lambda^+ u = u \comma \Lambda^- v = v \andtxt \Lambda^+ v = \Lambda^- u = 0 \\
    \end{align*}
\end{problem}
\begin{solution}
    \begin{enumerate}[label=(\alph*)]
        \item Show that $\Lambda^\pm$ are projection operators.
        
        \begin{itemize}
            \item $ \Lambda^+ + \Lambda^- = 1$ : Trivial
            
            \item $\Lambda^+\Lambda^+ = \Lambda^+$
            
                \begin{align*}
                    \Lambda^+ \Lambda^+ = \frac{1}{4m^2} \left(m+\pslash\right)\left(m+\pslash\right) &= \frac{1}{4m^2} \left( m^2 + 2m\pslash + \pslash\pslash \right)  \\[0.12in]
                    &= \frac{1}{4m^2} \left( 2m^2 + 2m\pslash \right) = \Lambda^+
                \end{align*}
                where the following identity is used : 

                \begin{align*}
                    \pslash \pslash = \gamma^\mu p_\mu \gamma^\nu p_\nu = \frac{1}{2} \left\{\gamma^\mu,\gamma^\nu \right\} p_\mu p_\nu = p \cdot p = m^2 
                \end{align*}\\
            
            \item  $\Lambda^-\Lambda^- = \Lambda^-$
            
                \begin{align*}
                    \Lambda^- \Lambda^- = \frac{1}{4m^2} \left(m-\pslash\right)\left(m-\pslash\right) &= \frac{1}{4m^2} \left( m^2 - 2m\pslash + \pslash\pslash \right)  \\[0.12in]
                    &= \frac{1}{4m^2} \left( 2m^2 - 2m\pslash \right) = \Lambda^-
                \end{align*}

            \item  $\Lambda^+\Lambda^- = 0$
            
                \begin{align*}
                    \Lambda^+ \Lambda^- = \frac{1}{4m^2} \left(m+\pslash\right)\left(m-\pslash\right) &= \frac{1}{4m^2} \left( m^2 -  \pslash\pslash \right) = 0 \\
                \end{align*}
        \end{itemize}
        \item Using the Dirac equations, $\left(\pslash-m\right)u=0$ and $\left(\pslash+m\right)v=0$ one could easily show the projections : 

        \begin{align*}
            \Lambda^+ u &= \frac{1}{2m} \left(m+\pslash\right) u = \frac{1}{2m} \left(mu+\pslash u\right) = \frac{2mu}{2m} = u \\[0.12in]
            \Lambda^+ v &= \frac{1}{2m} \left(m+\pslash\right) v =0  \\[0.12in]
            \Lambda^- v &= \frac{1}{2m} \left(m-\pslash\right) v = \frac{1}{2m} \left(mv-\pslash v\right) = \frac{2mv}{2m} = v \\[0.12in]
            \Lambda^- u &= \frac{1}{2m} \left(m-\pslash\right) u =0 \qed
        \end{align*}
    \end{enumerate}
\end{solution}

\noindent\rule{7in}{1.5pt}

%%%%%%%%%%%%%%%%%%%%%%%%%%%%%%%%%%%%%%%%%%%%%%%%%%%%%%%%%%%%%%%%%%%%%%%%%%%%%%%%%%%%%%%%%%%%%%%%%%%%%%%%%%%%%%%%%%%%%%%%%%%%%%%%%%%%%%%%

\begin{problem}{6.4}
    Show that the helicity operator can be expressed as

    \begin{align*}
        \hat{h} = - \frac{1}{2} \frac{\gamma^0\gamma^5\boldsymbol{\gamma}\cdot\boldp}{p}\\
    \end{align*}
\end{problem}
\begin{solution}
Using the following form of the helicity operator, 

\begin{align*}
    \hat{h} = \frac{1}{2p} \hat{\Sigma} \cdot \hatp = \frac{1}{2p} \left(1\otimes \boldsymbol{\sigma}\cdot\boldp\right)\\
\end{align*}
Using the properties of Kronecker prodcuts, one could decompose $1\otimes \boldsymbol{\sigma}\cdot\boldp$ as 

\begin{align} \label{P6.4.1}
    1\otimes \boldsymbol{\sigma}\cdot\boldp = \left(i\sigma^2 \otimes 1 \right)  \left(i\sigma^2 \otimes  \boldsymbol{\sigma}\cdot\boldp  \right)  \\ \nonumber
\end{align}
The two terms in the right-hand side of (\ref{P6.4.1}) can be again written as, 

\begin{align}
     i\sigma^2 \otimes 1 &= -\sigma^1\sigma^3 \otimes 1 \nonumber \\  \label{P6.4.2}
                         &= -\left(\sigma^1\otimes 1\right)  \left(\sigma^3\otimes 1\right) = -\gamma^0 \gamma^5  \\[0.12in]  \label{P6.4.3}
    i\sigma^2 \otimes  \boldsymbol{\sigma}\cdot\boldp &=  \left(i\sigma^2 \otimes \sigma^j\right)p_j = \boldsymbol{\gamma} \cdot \boldp \\ \nonumber
\end{align}
where the expression of gamma matrices, $\gamma^0 = \sigma^3 \otimes 1$, $\gamma^j=i\sigma^2\otimes\sigma^j$ and $\gamma^5=\gamma^1\otimes 1$ is used. Plugging in (\ref{P6.4.2}) and (\ref{P6.4.3}) into the original expression of the helicity operator,

\begin{align*}
    \hat{h} =  \frac{1}{2p} \left(1\otimes \boldsymbol{\sigma}\cdot\boldp\right) = -\frac{1}{2p} \gamma^0 \gamma^5 \boldsymbol{\gamma} \cdot \boldp  \qed \\
\end{align*}
\end{solution}

\noindent\rule{7in}{1.5pt}

%%%%%%%%%%%%%%%%%%%%%%%%%%%%%%%%%%%%%%%%%%%%%%%%%%%%%%%%%%%%%%%%%%%%%%%%%%%%%%%%%%%%%%%%%%%%%%%%%%%%%%%%%%%%%%%%%%%%%%%%%%%%%%%%%%%%%%%%

\begin{problem}{6.5}
    In general terms, explain why high-energy electron-positron colliders must also have high instantaneous luminosities.
\end{problem}
\begin{solution}
As seen in the text, the cross section $\sigma$ for electron-positron annihilation decreases as the center-of-mass energy $\sqrt{s}$ increases, from the relation $\sigma \sim s^{-1}$. Thus, such colliders must have high instantaneous luminosities in order to compensate the decreasing effect on the cross section stemming from the high collision energy.
\end{solution}

\noindent\rule{7in}{1.5pt}

%%%%%%%%%%%%%%%%%%%%%%%%%%%%%%%%%%%%%%%%%%%%%%%%%%%%%%%%%%%%%%%%%%%%%%%%%%%%%%%%%%%%%%%%%%%%%%%%%%%%%%%%%%%%%%%%%%%%%%%%%%%%%%%%%%%%%%%%

\begin{problem}{6.6}
    For a spin-1 system, the eigenstate of the operator $\hat{S}_n = \mathbf{n}\cdot\hat{\mathbf{S}}$ with eigenvalue +1 
    corresponds to the spin being in the direction $\hat{\mathbf{n}}$. Writing this state in terms of the eigenstates of $\hat{S}_z$, i.e.

    \begin{align*}
        \ket{1,+1}_\theta = \alpha \ket{1,-1} + \beta \ket{1,0} + \gamma \ket{1,+1}
    \end{align*}\\
    and taking $\mathbf{n} = \left(\sin\theta,0,\cos\theta\right)$ show that

    \begin{align*}
        \ket{1,+1}_\theta = \frac{1}{2}\left(1-\cos\theta\right)\ket{1,-1} + \frac{1}{\sqrt{2}} \sin\theta \ket{1,0} + \frac{1}{2}\left(1+\cos\theta\right) \ket{1,+1}
    \end{align*}\\

    \noindent {\footnotesize Hint: Write $\hat{S}_x$ in terms of the ladder operators.}
\end{problem}
\begin{solution}
Using the given $\mathbf{n}$, the operator $\hat{S}_n$ can be written as $\sin\theta \hat{S}_x + \cos \theta \hat{S}_z$. 

\begin{align*}
    \hat{S}_n = \sin\theta \hat{S}_x + \cos \theta \hat{S}_z = \frac{1}{2}\sin\theta \left(\hat{S}_+ + \hat{S}_-\right) + \cos \theta \hat{S}_z \\
\end{align*}
where $\hat{S}_\pm$ are the ladder operators which follows $\hat{S}_\pm \ket{1,m} = \sqrt{2-m(m\pm 1)}\ket{1,m\pm 1}$. Then using the above expression of $\hat{S}_n$, one could write down 

\begin{align*}
    \hat{S}_n  \ket{1,+1}_\theta &= \frac{1}{2}\sin\theta \left(\hat{S}_+ + \hat{S}_-\right)  \ket{1,+1}_\theta  + \cos \theta \hat{S}_z  \ket{1,+1}_\theta \\[0.12in]
    &= \left(\frac{1}{\sqrt{2}}\sin\theta\beta -\alpha\cos\theta \right) \ket{1,-1} + \frac{1}{\sqrt{2}} \left(\alpha+\gamma\right)\ket{1,0} + \left( \frac{1}{\sqrt{2}}\sin\theta\beta + \gamma\cos\theta \right) \ket{1,+1} \\
\end{align*}
and from the definition of $\hat{S}_n$, it should satisfy $\hat{S}_n \ket{1,+1}_\theta=\ket{1,+1}_\theta$ which gives a set of linear equations of $\alpha,\beta$ and $\gamma$ : 

\begin{align*}
    \alpha &= \frac{1}{\sqrt{2}}\sin\theta\beta -\alpha\cos\theta  \\[0.1in]
    \beta  &=  \frac{1}{\sqrt{2}} \left(\alpha+\gamma\right)\\[0.1in]
    \gamma &=  \frac{1}{\sqrt{2}}\sin\theta\beta + \gamma\cos\theta \\
\end{align*}
which gives the following solution : 

\begin{align*}
    \alpha &=  \frac{1}{2}\left(1-\cos\theta\right)  \\[0.1in]
    \beta  &=  \frac{1}{\sqrt{2}} \sin\theta \\[0.1in]
    \gamma &=  \frac{1}{2}\left(1+\cos\theta\right)  \qed \\
\end{align*}
\end{solution}

\noindent\rule{7in}{1.5pt}

%%%%%%%%%%%%%%%%%%%%%%%%%%%%%%%%%%%%%%%%%%%%%%%%%%%%%%%%%%%%%%%%%%%%%%%%%%%%%%%%%%%%%%%%%%%%%%%%%%%%%%%%%%%%%%%%%%%%%%%%%%%%%%%%%%%%%%%%

\begin{problem}{6.7}
    Using helicity amplitudes, calculate the differential cross section for $e^-\mu^- \to e^-\mu^-$ scattering in the following
    steps :

    \begin{enumerate}[label=(\alph*)]
        \item From the Feynman rules for QED, show that the lowest-order QED matrix element for $e^-\mu^- \to e^-\mu^-$ is 
        
        \begin{align*}
            \mathcal{M}_{fi} = -\frac{e^2}{\left(p_1-p_3\right)^2} g_{\mu\nu} \left[\overbar{u}(p_3)\gamma^\mu u(p_1)\right]\left[\overbar{u}(p_4)\gamma^\nu u(p_2)\right]
        \end{align*}\\
        where $p_1$ and $p_3$ are the four-momenta of the initial and final state $e^-$, and $p_2$ and $p_4$ are the four-momenta of the initial and final state $\mu^-$.

        \item Working in the centre-of-mass frame, and writing the four-momenta of the initial- and final-state $e^-$ as $p_1^\mu = (E_1,0,0,p)$ 
        and $p_3^\mu = (E_1,p\sin\theta,0,p\cos\theta)$ respectively, show that the electron currents for the four possible helicity combinations are

        \begin{align*}
            \overbar{u}_\downarrow(p_3)\gamma^\mu u_\downarrow(p_1) &= 2 (E_1 c , ps,-ips, pc) \\
            \overbar{u}_\uparrow(p_3)\gamma^\mu u_\downarrow(p_1)   &= 2 (ms,0,0,0) \\
            \overbar{u}_\uparrow(p_3)\gamma^\mu u_\uparrow(p_1) &= 2 (E_1 c, ps ,ips,pc) \\
            \overbar{u}_\downarrow(p_3)\gamma^\mu u_\uparrow(p_1) &= -2 (ms,0,0,0)
        \end{align*}\\
        where $m$ is the electron mass, $s=\sin(\theta/2)$ and $c=\cos(\theta/2)$.

        \item Explain why the effect of the parity operator $\hat{\mathsf{P}}=\gamma^0$ is 
        
        \begin{align*}
            \hat{\mathsf{P}} u_\uparrow (p,\theta,\phi) = \hat{\mathsf{P}} u_\downarrow (p,\pi-\theta,\pi+\theta)
        \end{align*}\\
        Hence, or otherwise, show that the muon currents for the four helicity combinations are

        \begin{align*}
            \overbar{u}_\downarrow(p_4)\gamma^\mu u_\downarrow(p_2) &= 2 (E_2 c , -ps,-ips, -pc) \\
            \overbar{u}_\uparrow(p_4)\gamma^\mu u_\downarrow(p_2)   &= 2 (Ms,0,0,0) \\
            \overbar{u}_\uparrow(p_4)\gamma^\mu u_\uparrow(p_2) &= 2 (E_2 c, -ps ,ips,-pc) \\
            \overbar{u}_\downarrow(p_4)\gamma^\mu u_\uparrow(p_2) &= -2 (Ms,0,0,0)
        \end{align*}\\
        where $M$ is the muon mass.

        \item For the relativistic limit where $E\gg M$, , show that the matrix element squared for the case where the incoming $e^-$ and incoming $\mu^-$ are both left-handed is given by
        
        \begin{align*}
            |\mathcal{M}_{LL}|^2 = \frac{4e^2s^2}{(p_1-p_3)^4}.
        \end{align*}\\
        where $s=(p_1+p_2)^2$. Find the corresponding expressions for $|\mathcal{M}_{RL}|^2,|\mathcal{M}_{RR}|^2$ and $|\mathcal{M}_{LR}|^2$.

        \item In this relativistic limit, show that the differential cross section for unpolarised $e^-\mu^- \to e^-\mu^-$ scattering in the centre-of-mass frame is
        
        \begin{align*}
            \frac{\dif\sigma}{\dif\Omega} = \frac{2\alpha^2}{s} \cdot \frac{1+\frac{1}{4}(1+\cos\theta)^2}{(1-\cos\theta)^2}.\\
        \end{align*}
        
    \end{enumerate}
\end{problem}
\begin{solution}
    \begin{enumerate}[label=(\alph*)]
        \item Obtain lowest-order QED matrix element.
            \begin{equation}
                \begin{aligned}
                    \begin{gathered}
                    \scalebox{1.25}{\begin{tikzpicture}
                        \begin{feynman}
                            \vertex (a2) ;
                            \vertex [above = 0.25em of a2] (a2l) {\(\scriptstyle \mu\)};
                            \vertex at ($(a2) + (-4em,1.5em) $) (a1) {$e^-$};
                            \vertex at ($(a2) + (4em,1.5em) $) (a3) {$e^-$};

                            \vertex [below=4.5em of a2] (b2);   
                            \vertex [below = 0.25em of b2] (b2l) {\(\scriptstyle \nu\)};     
                            \vertex at ($(b2) + (-4em,-1.5em) $) (b1) {$\mu^-$};
                            \vertex at ($(b2) + (4em,-1.5em) $) (b3) {$\mu^-$};

                            \diagram* {
                                {[edges=fermion]
                                  (a1) -- [edge label'=\(p_1\), near end] (a2) -- [edge label'=\(p_2\), near start] (a3),
                                  (b1) -- [edge label'=\(p_3\),near start] (b2) -- [edge label'=\(p_4\),near end] (b3) ,
                                }, % [edge label=\(\scriptstyle q\equiv p_3-p_1\)]
                                  (a2) -- [boson,edge label=\(\scriptstyle q\equiv p_3-p_1\)] (b2)
                            };
                        
                        \end{feynman}
                    \end{tikzpicture}
                    }
                    \end{gathered} &= \overbar{u}(p_3) ie\gamma^\mu u(p_1) \frac{-ig^{\mu\nu}}{(p_3-p_1)^2}\overbar{u}(p_4)ie\gamma^\nu u(p_2) \nonumber \\
                    &= -\frac{e^2}{\left(p_1-p_3\right)^2} g_{\mu\nu} \left[\overbar{u}(p_3)\gamma^\mu u(p_1)\right]\left[\overbar{u}(p_4)\gamma^\nu u(p_2)\right] \qed \\ 
                \end{aligned}
            \end{equation}
            
        \item 

        \item 

        \item 

        \item 
    \end{enumerate}
\end{solution}

\noindent\rule{7in}{1.5pt}

%%%%%%%%%%%%%%%%%%%%%%%%%%%%%%%%%%%%%%%%%%%%%%%%%%%%%%%%%%%%%%%%%%%%%%%%%%%%%%%%%%%%%%%%%%%%%%%%%%%%%%%%%%%%%%%%%%%%%%%%%%%%%%%%%%%%%%%%

\begin{problem}{6.8}
    Using $\gamma^\mu\gamma^\nu + \gamma^\nu\gamma^\mu = 2g^{\mu\nu}$, prove that 

    \begin{align*}
        \gamma^\mu\gamma_\mu =4 \comma \gamma^\mu \slashed{a}  \gamma_\mu = - 2 \slashed{a} \andtxt \gamma^\mu \slashed{a} \slashed{b} \gamma_\mu = 4 a\cdot b \\
    \end{align*}
\end{problem}
\begin{solution}

\end{solution}

\noindent\rule{7in}{1.5pt}

%%%%%%%%%%%%%%%%%%%%%%%%%%%%%%%%%%%%%%%%%%%%%%%%%%%%%%%%%%%%%%%%%%%%%%%%%%%%%%%%%%%%%%%%%%%%%%%%%%%%%%%%%%%%%%%%%%%%%%%%%%%%%%%%%%%%%%%%

\begin{problem}{6.9}
Prove the relation $ \left(\overbar{\psi}\gamma^\mu\gamma^5\phi\right)^\dagger=\overbar{\phi}\gamma^\mu\gamma^5\psi$.
\end{problem}
\begin{solution}
One could show that :

\begin{align*}
    \left(\overbar{\psi}\gamma^\mu\gamma^5\phi\right)^\dagger = \left( \psi^\dagger \gamma^0 \gamma^\mu \gamma^5 \phi \right)^\dagger &= -  \left( \psi^\dagger \gamma^0  \gamma^5 \gamma^\mu \phi \right)^\dagger \\[0.12in]
    &= - \phi^\dagger \gamma^{\mu\dagger} \gamma^{5\dagger} \gamma^{0\dagger}  \psi \\[0.12in]
    &=  - \phi^\dagger \left(\gamma^0 \gamma^{\mu} \gamma^0\right) \gamma^5 \gamma^0  \psi \\[0.12in]
    &=  - \phi^\dagger \gamma^0    \gamma^\mu \gamma^0  \gamma^5 \gamma^0    \psi  = \phi^\dagger \gamma^0    \gamma^\mu \gamma^0  \gamma^0  \gamma^5   \psi =\overbar{\phi}\gamma^\mu\gamma^5\psi \qed 
\end{align*}
\end{solution}

\noindent\rule{7in}{1.5pt}

%%%%%%%%%%%%%%%%%%%%%%%%%%%%%%%%%%%%%%%%%%%%%%%%%%%%%%%%%%%%%%%%%%%%%%%%%%%%%%%%%%%%%%%%%%%%%%%%%%%%%%%%%%%%%%%%%%%%%%%%%%%%%%%%%%%%%%%%

\begin{problem}{6.10}
Use the trace formalism to calculate the QED spin-averaged matrix element squared for $e^+e^-\to ff$ including the electron mass term.
\end{problem}
\begin{solution}

\end{solution}

\noindent\rule{7in}{1.5pt}

%%%%%%%%%%%%%%%%%%%%%%%%%%%%%%%%%%%%%%%%%%%%%%%%%%%%%%%%%%%%%%%%%%%%%%%%%%%%%%%%%%%%%%%%%%%%%%%%%%%%%%%%%%%%%%%%%%%%%%%%%%%%%%%%%%%%%%%%

\begin{problem}{6.11}
Neglecting the electron mass term, verify that the matrix element for $e^- f \to e^- f$ given in (6.67) can be obtained from the matrix element for $e^+e^-\to ff$ given in (6.63) using crossing symmetry with the substitutions

\begin{align*}
    p_1 \to p_1 \comma p_2 \to -p_3 \comma p_3 \to p_4 \andtxt p_4 \to -p_2 \\
\end{align*}

\end{problem}
\begin{solution}

\end{solution}

\noindent\rule{7in}{1.5pt}

%%%%%%%%%%%%%%%%%%%%%%%%%%%%%%%%%%%%%%%%%%%%%%%%%%%%%%%%%%%%%%%%%%%%%%%%%%%%%%%%%%%%%%%%%%%%%%%%%%%%%%%%%%%%%%%%%%%%%%%%%%%%%%%%%%%%%%%%

\begin{problem}{6.12}
Write down the matrix elements, $\mathcal{M}_1$ and $\mathcal{M}_2$, for the two Feynman diagrams for the Compton scattering process $e^-\gamma\to e^-\gamma$. From first principles, express the spin-averaged matrix element $\expval{|\mathcal{M}_1+\mathcal{M}_2|^2}$ as a trace. You will need the completeness relation for the photon polarisation states (see Appendix D).
\end{problem}
\begin{solution}

\end{solution}

\noindent\rule{7in}{1.5pt}

%%%%%%%%%%%%%%%%%%%%%%%%%%%%%%%%%%%%%%%%%%%%%%%%%%%%%%%%%%%%%%%%%%%%%%%%%%%%%%%%%%%%%%%%%%%%%%%%%%%%%%%%%%%%%%%%%%%%%%%%%%%%%%%%%%%%%%%%
