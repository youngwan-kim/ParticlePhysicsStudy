
\noindent\rule{7in}{2.8pt}
\section{Electron-Positron Annihilation}
    
\begin{problem}{6.1}
Using the properties of the $\gamma$-matrices of (4.33) and (4.34), and the definition of $\gamma^5\equiv\gamma^0\gamma^1\gamma^2\gamma^3$, show that

\begin{align*}
    \left(\gamma^5\right)^2 = 1 \quad \text{,} \quad \gamma^{5\dagger} = i\gamma^5 \quad \text{and} \quad \gamma^5\gamma^\mu = - \gamma^\mu\gamma^5 \\
\end{align*}
\end{problem}
\begin{solution}
    \begin{enumerate}[label=(\alph*)]
        \item $\left(\gamma^5\right)^2 = 1 $
            \begin{align*}
                \left(\gamma^5\right)^2 &= -\gamma^0\gamma^1\gamma^2\gamma^3 \gamma^0\gamma^1\gamma^2\gamma^3 \\[0.12in]
                                        &= \left(-1\right)^4 \gamma^0\gamma^0 \gamma^1\gamma^2\gamma^3 \gamma^1\gamma^2\gamma^3 \\[0.12in]
                                        &= \left(-1\right)^6  \gamma^1  \gamma^1 \gamma^2\gamma^3\gamma^2\gamma^3 \\[0.12in]
                                        &= \left(-1\right)^8  \gamma^2\gamma^2 \gamma^3\gamma^3 = 1 \qed
            \end{align*}
        \item $\gamma^{5\dagger} = \gamma^5$
            \begin{align*}
                \gamma^{5\dagger} = \left(i\gamma^0\gamma^1\gamma^2\gamma^3\right)^\dagger &= -i\gamma^{3\dagger}\gamma^{2\dagger}\gamma^{1\dagger}\gamma^{0\dagger} \\[0.12in]
                &= -i \left(-1\right)^3 \gamma^3 \gamma^2 \gamma^1 \gamma^0 \\[0.12in]
                &= -i \left(-1\right)^3 \left(-1\right)^6 \gamma^0\gamma^1\gamma^2\gamma^3 = \gamma^5 \qed
            \end{align*}
        \item $\gamma^5\gamma^\mu = - \gamma^\mu\gamma^5$
            \begin{align*}
                \gamma^5\gamma^\mu &=  i\gamma^0\gamma^1\gamma^2\gamma^3 \gamma^\mu \\[0.12in]
                                   &= \left(-1\right)^3 \gamma^\mu i\gamma^0\gamma^1\gamma^2\gamma^3 = -\gamma^\mu \gamma^5 \qed \\
            \end{align*}
            Here the fact that $\mu$ will be one of $0,1,2,3$ is used to obtain the $\left(-1\right)^3$ factor, as one of the indices will be identical, thus not giving an additional $-1$ factor when switching the position of $\gamma^\mu$.
    \end{enumerate}
\end{solution}

\noindent\rule{7in}{1.5pt}

%%%%%%%%%%%%%%%%%%%%%%%%%%%%%%%%%%%%%%%%%%%%%%%%%%%%%%%%%%%%%%%%%%%%%%%%%%%%%%%%%%%%%%%%%%%%%%%%%%%%%%%%%%%%%%%%%%%%%%%%%%%%%%%%%%%%%%%%

\begin{problem}{6.2}
    Show that the chiral projection operators 

    \begin{align*}
        P_R = \frac{1}{2}\left(1+\gamma^5\right) \quad \text{and} \quad P_L = \frac{1}{2}\left(1-\gamma^5\right)
    \end{align*}\\
    satisfy 

    \begin{align*}
        P_R + P_L = 1 \comma P_R P_R = P_R \comma P_L P_L = P_L \andtxt P_LP_R = 0 \\
    \end{align*}
\end{problem}
\begin{solution}
    \begin{enumerate}[label=(\alph*)]
        \item $ P_R + P_L = 1$ : Trivial 
        \item $ P_R P_R = P_R$ 
        
            \begin{align*}
                P_R P_R  = \frac{1}{4} \left(1+\gamma^5\right) \left(1+\gamma^5\right) &= \frac{1}{4} \left(1+2\gamma^5+\gamma^5\gamma^5\right) = \frac{1}{4} \left(2+2\gamma^5\right) = P_R \qed \\
            \end{align*}
        \item $ P_L P_L = P_L$ 
        
            \begin{align*}
                P_L P_L  = \frac{1}{4} \left(1-\gamma^5\right) \left(1-\gamma^5\right) &= \frac{1}{4} \left(1-2\gamma^5+\gamma^5\gamma^5\right) = \frac{1}{4} \left(2-2\gamma^5\right) = P_L \qed\\
            \end{align*}
        \item $ P_L P_R = 0$
        
            \begin{align*}
                P_L P_R  = \frac{1}{4} \left(1-\gamma^5\right) \left(1+\gamma^5\right) = \frac{1}{4} \left( 1- \gamma^5\gamma^5\right) = 0 \qed \\
            \end{align*}
    \end{enumerate}
\end{solution}

\noindent\rule{7in}{1.5pt}

%%%%%%%%%%%%%%%%%%%%%%%%%%%%%%%%%%%%%%%%%%%%%%%%%%%%%%%%%%%%%%%%%%%%%%%%%%%%%%%%%%%%%%%%%%%%%%%%%%%%%%%%%%%%%%%%%%%%%%%%%%%%%%%%%%%%%%%%

\begin{problem}{6.3}
    Show that

    \begin{align*}
        \Lambda^+ = \frac{m+\slashed{p}}{2m} \andtxt \Lambda^- = \frac{m-\slashed{p}}{2m}
    \end{align*}\\
    are also projection operators, and show that they respectively project out particle and antiparticle states, i.e.

    \begin{align*}
        \Lambda^+ u = u \comma \Lambda^- v = v \andtxt \Lambda^+ v = \Lambda^- u = 0 \\
    \end{align*}
\end{problem}
\begin{solution}
    \begin{enumerate}[label=(\alph*)]
        \item Show that $\Lambda^\pm$ are projection operators.
        
        \begin{itemize}
            \item $ \Lambda^+ + \Lambda^- = 1$ : Trivial
            
            \item $\Lambda^+\Lambda^+ = \Lambda^+$
            
                \begin{align*}
                    \Lambda^+ \Lambda^+ = \frac{1}{4m^2} \left(m+\pslash\right)\left(m+\pslash\right) &= \frac{1}{4m^2} \left( m^2 + 2m\pslash + \pslash\pslash \right)  \\[0.12in]
                    &= \frac{1}{4m^2} \left( 2m^2 + 2m\pslash \right) = \Lambda^+
                \end{align*}
                where the following identity is used : 

                \begin{align*}
                    \pslash \pslash = \gamma^\mu p_\mu \gamma^\nu p_\nu = \frac{1}{2} \left\{\gamma^\mu,\gamma^\nu \right\} p_\mu p_\nu = p \cdot p = m^2 
                \end{align*}\\
            
            \item  $\Lambda^-\Lambda^- = \Lambda^-$
            
                \begin{align*}
                    \Lambda^- \Lambda^- = \frac{1}{4m^2} \left(m-\pslash\right)\left(m-\pslash\right) &= \frac{1}{4m^2} \left( m^2 - 2m\pslash + \pslash\pslash \right)  \\[0.12in]
                    &= \frac{1}{4m^2} \left( 2m^2 - 2m\pslash \right) = \Lambda^-
                \end{align*}

            \item  $\Lambda^+\Lambda^- = 0$
            
                \begin{align*}
                    \Lambda^+ \Lambda^- = \frac{1}{4m^2} \left(m+\pslash\right)\left(m-\pslash\right) &= \frac{1}{4m^2} \left( m^2 -  \pslash\pslash \right) = 0 \\
                \end{align*}
        \end{itemize}
        \item Using the Dirac equations, $\left(\pslash-m\right)u=0$ and $\left(\pslash+m\right)v=0$ one could easily show the projections : 

        \begin{align*}
            \Lambda^+ u &= \frac{1}{2m} \left(m+\pslash\right) u = \frac{1}{2m} \left(mu+\pslash u\right) = \frac{2mu}{2m} = u \\[0.12in]
            \Lambda^+ v &= \frac{1}{2m} \left(m+\pslash\right) v =0  \\[0.12in]
            \Lambda^- v &= \frac{1}{2m} \left(m-\pslash\right) v = \frac{1}{2m} \left(mv-\pslash v\right) = \frac{2mv}{2m} = v \\[0.12in]
            \Lambda^- u &= \frac{1}{2m} \left(m-\pslash\right) u =0 \qed
        \end{align*}
    \end{enumerate}
\end{solution}

\noindent\rule{7in}{1.5pt}

%%%%%%%%%%%%%%%%%%%%%%%%%%%%%%%%%%%%%%%%%%%%%%%%%%%%%%%%%%%%%%%%%%%%%%%%%%%%%%%%%%%%%%%%%%%%%%%%%%%%%%%%%%%%%%%%%%%%%%%%%%%%%%%%%%%%%%%%

\begin{problem}{6.4}
    Show that the helicity operator can be expressed as

    \begin{align*}
        \hat{h} = - \frac{1}{2} \frac{\gamma^0\gamma^5\boldsymbol{\gamma}\cdot\boldp}{p}\\
    \end{align*}
\end{problem}
\begin{solution}
Using the following form of the helicity operator, 

\begin{align*}
    \hat{h} = \frac{1}{2p} \hat{\Sigma} \cdot \hatp = \frac{1}{2p} \left(1\otimes \boldsymbol{\sigma}\cdot\boldp\right)\\
\end{align*}
Using the properties of Kronecker prodcuts, one could decompose $1\otimes \boldsymbol{\sigma}\cdot\boldp$ as 

\begin{align} \label{P6.4.1}
    1\otimes \boldsymbol{\sigma}\cdot\boldp = - \left(i\sigma^2 \otimes 1 \right)  \left(i\sigma^2 \otimes  \boldsymbol{\sigma}\cdot\boldp  \right)  \\ \nonumber
\end{align}
The two terms in the right-hand side of (\ref{P6.4.1}) can be again written as, 

\begin{align}
     i\sigma^2 \otimes 1 &= -\sigma^1\sigma^3 \otimes 1 \nonumber \\  \label{P6.4.2}
                         &= -\left(\sigma^1\otimes 1\right)  \left(\sigma^3\otimes 1\right) = -\gamma^0 \gamma^5  \\[0.12in]  \label{P6.4.3}
    i\sigma^2 \otimes  \boldsymbol{\sigma}\cdot\boldp &=  \left(i\sigma^2 \otimes \sigma^j\right)p_j = \boldsymbol{\gamma} \cdot \boldp \\ \nonumber
\end{align}
where the expression of gamma matrices, $\gamma^0 = \sigma^3 \otimes 1$, $\gamma^j=i\sigma^2\otimes\sigma^j$ and $\gamma^5=\sigma^1\otimes 1$ is used. Plugging in (\ref{P6.4.2}) and (\ref{P6.4.3}) into the original expression of the helicity operator,

\begin{align*}
    \hat{h} =  \frac{1}{2p} \left(1\otimes \boldsymbol{\sigma}\cdot\boldp\right) = -\frac{1}{2p} \gamma^5 \gamma^0 \boldsymbol{\gamma} \cdot \boldp  =  \frac{1}{2p} \gamma^0 \gamma^5 \boldsymbol{\gamma} \cdot \boldp  \qed \\
\end{align*}
\end{solution}

\noindent\rule{7in}{1.5pt}

%%%%%%%%%%%%%%%%%%%%%%%%%%%%%%%%%%%%%%%%%%%%%%%%%%%%%%%%%%%%%%%%%%%%%%%%%%%%%%%%%%%%%%%%%%%%%%%%%%%%%%%%%%%%%%%%%%%%%%%%%%%%%%%%%%%%%%%%

\begin{problem}{6.5}
    In general terms, explain why high-energy electron-positron colliders must also have high instantaneous luminosities.
\end{problem}
\begin{solution}
As seen in the text, the cross section $\sigma$ for electron-positron annihilation decreases as the center-of-mass energy $\sqrt{s}$ increases, from the relation $\sigma \sim s^{-1}$. Thus, such colliders must have high instantaneous luminosities in order to compensate the decreasing effect on the cross section stemming from the high collision energy.
\end{solution}

\noindent\rule{7in}{1.5pt}

%%%%%%%%%%%%%%%%%%%%%%%%%%%%%%%%%%%%%%%%%%%%%%%%%%%%%%%%%%%%%%%%%%%%%%%%%%%%%%%%%%%%%%%%%%%%%%%%%%%%%%%%%%%%%%%%%%%%%%%%%%%%%%%%%%%%%%%%

\begin{problem}{6.6}
    For a spin-1 system, the eigenstate of the operator $\hat{S}_n = \mathbf{n}\cdot\hat{\mathbf{S}}$ with eigenvalue +1 
    corresponds to the spin being in the direction $\hat{\mathbf{n}}$. Writing this state in terms of the eigenstates of $\hat{S}_z$, i.e.

    \begin{align*}
        \ket{1,+1}_\theta = \alpha \ket{1,-1} + \beta \ket{1,0} + \gamma \ket{1,+1}
    \end{align*}\\
    and taking $\mathbf{n} = \left(\sin\theta,0,\cos\theta\right)$ show that

    \begin{align*}
        \ket{1,+1}_\theta = \frac{1}{2}\left(1-\cos\theta\right)\ket{1,-1} + \frac{1}{\sqrt{2}} \sin\theta \ket{1,0} + \frac{1}{2}\left(1+\cos\theta\right) \ket{1,+1}
    \end{align*}\\

    \noindent {\footnotesize Hint: Write $\hat{S}_x$ in terms of the ladder operators.}
\end{problem}
\begin{solution}
Using the given $\mathbf{n}$, the operator $\hat{S}_n$ can be written as $\sin\theta \hat{S}_x + \cos \theta \hat{S}_z$. 

\begin{align*}
    \hat{S}_n = \sin\theta \hat{S}_x + \cos \theta \hat{S}_z = \frac{1}{2}\sin\theta \left(\hat{S}_+ + \hat{S}_-\right) + \cos \theta \hat{S}_z \\
\end{align*}
where $\hat{S}_\pm$ are the ladder operators which follows $\hat{S}_\pm \ket{1,m} = \sqrt{2-m(m\pm 1)}\ket{1,m\pm 1}$. Then using the above expression of $\hat{S}_n$, one could write down 

\begin{align*}
    \hat{S}_n  \ket{1,+1}_\theta &= \frac{1}{2}\sin\theta \left(\hat{S}_+ + \hat{S}_-\right)  \ket{1,+1}_\theta  + \cos \theta \hat{S}_z  \ket{1,+1}_\theta \\[0.12in]
    &= \left(\frac{1}{\sqrt{2}}\sin\theta\beta -\alpha\cos\theta \right) \ket{1,-1} + \frac{1}{\sqrt{2}} \left(\alpha+\gamma\right)\ket{1,0} + \left( \frac{1}{\sqrt{2}}\sin\theta\beta + \gamma\cos\theta \right) \ket{1,+1} \\
\end{align*}
and from the definition of $\hat{S}_n$, it should satisfy $\hat{S}_n \ket{1,+1}_\theta=\ket{1,+1}_\theta$ which gives a set of linear equations of $\alpha,\beta$ and $\gamma$ : 

\begin{align*}
    \alpha &= \frac{1}{\sqrt{2}}\sin\theta\beta -\alpha\cos\theta  \\[0.1in]
    \beta  &=  \frac{1}{\sqrt{2}} \left(\alpha+\gamma\right)\\[0.1in]
    \gamma &=  \frac{1}{\sqrt{2}}\sin\theta\beta + \gamma\cos\theta \\
\end{align*}
which gives the following solution : 

\begin{align*}
    \alpha &=  \frac{1}{2}\left(1-\cos\theta\right)  \\[0.1in]
    \beta  &=  \frac{1}{\sqrt{2}} \sin\theta \\[0.1in]
    \gamma &=  \frac{1}{2}\left(1+\cos\theta\right)  \qed \\
\end{align*}
\end{solution}

\noindent\rule{7in}{1.5pt}

%%%%%%%%%%%%%%%%%%%%%%%%%%%%%%%%%%%%%%%%%%%%%%%%%%%%%%%%%%%%%%%%%%%%%%%%%%%%%%%%%%%%%%%%%%%%%%%%%%%%%%%%%%%%%%%%%%%%%%%%%%%%%%%%%%%%%%%%

\begin{problem}{6.7}
    Using helicity amplitudes, calculate the differential cross section for $e^-\mu^- \to e^-\mu^-$ scattering in the following
    steps :

    \begin{enumerate}[label=(\alph*)]
        \item From the Feynman rules for QED, show that the lowest-order QED matrix element for $e^-\mu^- \to e^-\mu^-$ is 
        
        \begin{align*}
            \mathcal{M}_{fi} = -\frac{e^2}{\left(p_1-p_3\right)^2} g_{\mu\nu} \left[\overbar{u}(p_3)\gamma^\mu u(p_1)\right]\left[\overbar{u}(p_4)\gamma^\nu u(p_2)\right]
        \end{align*}\\
        where $p_1$ and $p_3$ are the four-momenta of the initial and final state $e^-$, and $p_2$ and $p_4$ are the four-momenta of the initial and final state $\mu^-$.

        \item Working in the centre-of-mass frame, and writing the four-momenta of the initial- and final-state $e^-$ as $p_1^\mu = (E_1,0,0,p)$ 
        and $p_3^\mu = (E_1,p\sin\theta,0,p\cos\theta)$ respectively, show that the electron currents for the four possible helicity combinations are

        \begin{align*}
            \overbar{u}_\downarrow(p_3)\gamma^\mu u_\downarrow(p_1) &= 2 (E_1 c , ps,-ips, pc) \\
            \overbar{u}_\uparrow(p_3)\gamma^\mu u_\downarrow(p_1)   &= 2 (ms,0,0,0) \\
            \overbar{u}_\uparrow(p_3)\gamma^\mu u_\uparrow(p_1) &= 2 (E_1 c, ps ,ips,pc) \\
            \overbar{u}_\downarrow(p_3)\gamma^\mu u_\uparrow(p_1) &= -2 (ms,0,0,0)
        \end{align*}\\
        where $m$ is the electron mass, $s=\sin(\theta/2)$ and $c=\cos(\theta/2)$.

        \item Explain why the effect of the parity operator $\hat{\mathsf{P}}=\gamma^0$ is 
        
        \begin{align*}
            \hat{\mathsf{P}} u_\uparrow (p,\theta,\phi) =   u_\downarrow (p,\pi-\theta,\pi+\theta)
        \end{align*}\\
        Hence, or otherwise, show that the muon currents for the four helicity combinations are

        \begin{align*}
            \overbar{u}_\downarrow(p_4)\gamma^\mu u_\downarrow(p_2) &= 2 (E_2 c , -ps,-ips, -pc) \\
            \overbar{u}_\uparrow(p_4)\gamma^\mu u_\downarrow(p_2)   &= 2 (Ms,0,0,0) \\
            \overbar{u}_\uparrow(p_4)\gamma^\mu u_\uparrow(p_2) &= 2 (E_2 c, -ps ,ips,-pc) \\
            \overbar{u}_\downarrow(p_4)\gamma^\mu u_\uparrow(p_2) &= -2 (Ms,0,0,0)
        \end{align*}\\
        where $M$ is the muon mass.

        \item For the relativistic limit where $E\gg M$, , show that the matrix element squared for the case where the incoming $e^-$ and incoming $\mu^-$ are both left-handed is given by
        
        \begin{align*}
            |\mathcal{M}_{LL}|^2 = \frac{4e^2s^2}{(p_1-p_3)^4}.
        \end{align*}\\
        where $s=(p_1+p_2)^2$. Find the corresponding expressions for $|\mathcal{M}_{RL}|^2,|\mathcal{M}_{RR}|^2$ and $|\mathcal{M}_{LR}|^2$.

        \item In this relativistic limit, show that the differential cross section for unpolarised $e^-\mu^- \to e^-\mu^-$ scattering in the centre-of-mass frame is
        
        \begin{align*}
            \frac{\dif\sigma}{\dif\Omega} = \frac{2\alpha^2}{s} \cdot \frac{1+\frac{1}{4}(1+\cos\theta)^2}{(1-\cos\theta)^2}.\\
        \end{align*}
        
    \end{enumerate}
\end{problem}
\begin{solution}
    \begin{enumerate}[label=(\alph*)]
        \item Obtain lowest-order QED matrix element.
            \begin{equation}
                \begin{aligned}
                    \begin{gathered}
                    \scalebox{1.25}{\begin{tikzpicture}
                        \begin{feynman}
                            \vertex (a2) ;
                            \vertex [above = 0.25em of a2] (a2l) {\(\scriptstyle \mu\)};
                            \vertex at ($(a2) + (-4em,1.5em) $) (a1) {$e^-$};
                            \vertex at ($(a2) + (4em,1.5em) $) (a3) {$e^-$};

                            \vertex [below=4.5em of a2] (b2);   
                            \vertex [below = 0.25em of b2] (b2l) {\(\scriptstyle \nu\)};     
                            \vertex at ($(b2) + (-4em,-1.5em) $) (b1) {$\mu^-$};
                            \vertex at ($(b2) + (4em,-1.5em) $) (b3) {$\mu^-$};

                            \diagram* {
                                {[edges=fermion]
                                  (a1) -- [edge label'=\(p_1\), near end] (a2) -- [edge label'=\(p_3\), near start] (a3),
                                  (b1) -- [edge label'=\(p_2\),near start] (b2) -- [edge label'=\(p_4\),near end] (b3) ,
                                }, % [edge label=\(\scriptstyle q\equiv p_3-p_1\)]
                                  (a2) -- [boson,edge label=\(\scriptstyle q\equiv p_3-p_1\)] (b2)
                            };
                        
                        \end{feynman}
                    \end{tikzpicture}
                    }
                    \end{gathered}  = -i\mathcal{M} &= \overbar{u}(p_3) ie\gamma^\mu u(p_1) \frac{-ig^{\mu\nu}}{(p_3-p_1)^2}\overbar{u}(p_4)ie\gamma^\nu u(p_2) \nonumber \\[-0.5in]
                    &= \frac{ie^2}{\left(p_1-p_3\right)^2} g_{\mu\nu} \left[\overbar{u}(p_3)\gamma^\mu u(p_1)\right]\left[\overbar{u}(p_4)\gamma^\nu u(p_2)\right] \qed \\[0.35in]
                \end{aligned}
            \end{equation}
            
        \item Letting the incoming and outgoing $e^-$ with $\left(\theta',\phi'\right)=(0,0)$ and $(\theta,0)$ respectively, one could write down the corresponding spinors as ,
        
        \begin{align*}
            u_\uparrow (p_1) = \sqrt{E_1+m} \spinor{1.2}{1}{0}{\frac{p}{E_1+m}}{0} &\comma u_\downarrow (p_1) = \sqrt{E_1+m} \spinor{1.2}{0}{1}{0}{\frac{-p}{E_1+m}} \\[0.12in]
            u_\uparrow (p_3) = \sqrt{E_1+m} \spinor{1.2}{c}{s}{\frac{p}{E_1+m}c}{\frac{p}{E_1+m}s} &\comma u_\downarrow (p_3) = \sqrt{E_1+m} \spinor{1.2}{-s}{c}{\frac{p}{E_1+m}s}{\frac{-p}{E_1+m}c} \\
        \end{align*}
        Using such expression of the spinors, one could directly calculate the possible electron currents as : 

        \begin{align*}
            \overbar{u}_\downarrow(p_3)\gamma^\mu u_\downarrow(p_1) &= 2 (E_1 c , ps,-ips, pc) \\
            \overbar{u}_\uparrow(p_3)\gamma^\mu u_\downarrow(p_1)   &= 2 (ms,0,0,0) \\
            \overbar{u}_\uparrow(p_3)\gamma^\mu u_\uparrow(p_1) &= 2 (E_1 c, ps ,ips,pc) \\
            \overbar{u}_\downarrow(p_3)\gamma^\mu u_\uparrow(p_1) &= -2 (ms,0,0,0) \\
        \end{align*}
        (Trust me I really did all the calculations lol)

        \item The effect of $\hat{\mathsf{P}}$ on helicity states was once discussed in Problem (\ref{P4.13}), that acting such operator will actually flip the spatial part of $p$ also leading to an opposite helicity state, hence $u_\downarrow$. Also, considering how the four-momenta are set from the electron currents, one could notice that by substituting $E_1 \leftrightarrow E_2$ and $m \leftrightarrow M$ while having all the electron current helicity transformed under $\hat{\mathsf{P}}$ will get us the full set of the muon current helicity combinations. 

        \item Denoting the corresponding helicity configuration with momentum $p_i$ as $h_i$, $\mathcal{M}_{LL}$ can be expressed as, 

            \begin{align}
                \mathcal{M}_{LL} &= -\frac{e^2}{(p_1-p_3)^2} \sum_{h_3,h_4 \in \{L,R\}} j^e_{Lh_3} \cdot j^\mu_{Lh_4} \nonumber \\[0.125in]  \label{P.6.7.d1}
                &=  -\frac{e^2}{(p_1-p_3)^2} \big[ j^e_{LL}\cdot j^\mu_{LL}+ j^e_{LL}\cdot j^\mu_{LR} + j^e_{LR}\cdot j^\mu_{LL} + j^e_{LR}\cdot j^\mu_{LR} \big] \\ \nonumber
            \end{align}
        Noting that under the relativistic limit of $E\gg M$, the only term that survives in (\ref{P.6.7.d1}) is $j^e_{LL}\cdot j^\mu_{LL}$ which can be calculated as,

            \begin{align*}
                j^e_{LL}\cdot j^\mu_{LL} &= 4 \left(E_1 \cos\halftheta , p \sin\halftheta,-ip\sin\halftheta , p\cos\halftheta \right) \cdot \left( E_2 \cos\halftheta , -p \sin\halftheta , -ip\sin\halftheta , -p\cos\halftheta \right) \\[0.12in]
                &= 4\left(E_1E_2 \cos^2\halftheta +p^2\sin^2\halftheta + p^2 \sin^2\halftheta + p^2 \cos^2 \halftheta \right)  \impliedby p^2 \sim E_1^2 \sim E_2^2 (E\gg M)\\[0.12in]
                &= 4E_1E_2 = 2 \left(p_1+p_2\right)^2 \equiv 2s  \\ 
            \end{align*}
        Plugging this back into (\ref{P.6.7.d1}) and squaring the matrix element gives,

            \begin{align*}
                \left| \mathcal{M}_{LL} \right|^2 = \frac{e^4}{(p_1-p_3)^4} \left(j^e_{LL}\cdot j^\mu_{LL}\right)^2 =  \frac{4e^4s^2}{(p_1-p_3)^4} \qed \\
            \end{align*}
        For the other helicity combinations, the same could be done. 

            \begin{align*}
                \left|\mathcal{M}_{RL}\right|^2 &= \frac{e^4}{(p_1-p_3)^4}\left[ \sum_{h_3,h_4 \in \{L,R\}} j^e_{Rh_3} \cdot j^\mu_{Lh_4}\right]^2  \\[0.125in] 
                &=  \frac{e^4}{(p_1-p_3)^4} \big[\cancelto{0}{j^e_{RL}\cdot j^\mu_{LL}}+ \cancelto{0}{j^e_{RL}\cdot j^\mu_{LR} }+ j^e_{RR}\cdot j^\mu_{LL} + \cancelto{0}{j^e_{RR}\cdot j^\mu_{LR}} \big]^2 \\[0.125in]
                &= \frac{16e^4}{(p_1-p_3)^4} \left( E_1E_2\cos^2\halftheta + p^2 \sin^2\halftheta - p^2 \sin^2\halftheta + p^2 \cos^2\halftheta \right)^2 \\[0.125in]
                &= \frac{4e^4s^2}{(p_1-p_3)^4}  \cos^4\halftheta 
            \end{align*}

        Again, 

            \begin{align*}
                \left|\mathcal{M}_{RR}\right|^2 &= \frac{e^4}{(p_1-p_3)^4} \left(j^e_{RR}\cdot j^\mu_{RR}\right)^2  \\[0.125in] 
                &= \frac{16e^4}{(p_1-p_3)^4} \left( E_1E_2\cos^2\halftheta + p^2 \sin^2\halftheta + p^2 \sin^2\halftheta + p^2 \cos^2\halftheta \right)^2 =  \frac{4e^4s^2}{(p_1-p_3)^4} 
            \end{align*}

        And finally, 

            \begin{align*}
                \left|\mathcal{M}_{LR}\right|^2 &= \frac{e^4}{(p_1-p_3)^4} \left(j^e_{LL}\cdot j^\mu_{RR}\right)^2  \\[0.125in] 
                &= \frac{16e^4}{(p_1-p_3)^4} \left( E_1E_2\cos^2\halftheta + p^2 \sin^2\halftheta - p^2 \sin^2\halftheta + p^2 \cos^2\halftheta \right)^2 = \frac{4e^4s^2}{(p_1-p_3)^4}  \cos^4\halftheta \\
            \end{align*}

        \item Averaging out all the helicity dependent amplitudes from (d), one would obtain 
        
            \begin{align*}
                \expval{\left|\mathcal{M}_{fi}\right|^2} &= \frac{1}{4} \left\{ \left|\mathcal{M}_{LR}\right|^2 + \left|\mathcal{M}_{RR}\right|^2  + \left|\mathcal{M}_{RL}\right|^2  + \left|\mathcal{M}_{LL}\right|^2  \right\} \\[0.12in]
                &= \frac{2e^4s^2}{(p_1-p_3)^4} \left( 1+ \cos^4\halftheta \right) \\[0.12in]
                &= \frac{2e^4s^2}{(p_1-p_3)^4} \left[ 1+ \frac{1}{4}\left(1+\cos\theta\right)^2 \right] \\[0.12in]
                &= \frac{e^4s^2}{2E_1^4 \left( 1-\cos\theta\right)^2} \left[ 1+ \frac{1}{4}\left(1+\cos\theta\right)^2 \right] \impliedby \left(p_1-p_3\right)^2 \simeq -2p_1\cdot p_3 = -2 E_1^2 \left(1-\cos\theta \right)\\[0.12in]
                &\simeq \frac{8e^4 }{  \left( 1-\cos\theta\right)^2} \left[ 1+ \frac{1}{4}\left(1+\cos\theta\right)^2 \right]  
            \end{align*}\\
            Then, the differential cross section can be written as,

            \begin{align*}
                \frac{\dif\sigma}{\dif\Omega} &= \frac{1}{64\pi^2 s}  \expval{\left|\mathcal{M}_{fi}\right|^2} = \frac{1}{8\pi^2 s} \cdot \frac{e^4 }{  \left( 1-\cos\theta\right)^2} \left[ 1+ \frac{1}{4}\left(1+\cos\theta\right)^2 \right]  \impliedby e^4 = 16\pi^2\alpha^2\\[0.12in]
                &= \frac{2\alpha^2}{s} \cdot\frac{1 }{  \left( 1-\cos\theta\right)^2} \left[ 1+ \frac{1}{4}\left(1+\cos\theta\right)^2 \right] \qed
            \end{align*}
    \end{enumerate}
\end{solution}

\noindent\rule{7in}{1.5pt}

%%%%%%%%%%%%%%%%%%%%%%%%%%%%%%%%%%%%%%%%%%%%%%%%%%%%%%%%%%%%%%%%%%%%%%%%%%%%%%%%%%%%%%%%%%%%%%%%%%%%%%%%%%%%%%%%%%%%%%%%%%%%%%%%%%%%%%%%

\begin{problem}{6.8}
    Using $\gamma^\mu\gamma^\nu + \gamma^\nu\gamma^\mu = 2g^{\mu\nu}$, prove that 

    \begin{align*}
        \gamma^\mu\gamma_\mu =4 \comma \gamma^\mu \slashed{a}  \gamma_\mu = - 2 \slashed{a} \andtxt \gamma^\mu \slashed{a} \slashed{b} \gamma_\mu = 4 a\cdot b \\
    \end{align*}
\end{problem}
\begin{solution}
    \begin{enumerate}[label=(\alph*)]
        \item $\gamma^\mu\gamma_\mu =4$
        
            \begin{align*}
                \gamma^\mu\gamma_\mu = \frac{1}{2}g_{\nu\mu} \left(\gamma^\mu\gamma^\nu + \gamma^\nu\gamma^\mu\right) = g^{\mu\nu} g_{\mu\nu} = 4 \qed
            \end{align*}

        \item $ \gamma^\mu \slashed{a}  \gamma_\mu = - 2 \slashed{a}$
        
            \begin{align*}
                \gamma^\mu \slashed{a}  \gamma_\mu = a_\nu \gamma^\mu \gamma^\nu  \gamma_\mu &= a_\nu \left( 2g^{\mu\nu} - \gamma^\nu\gamma^\mu \right)\gamma_\mu  \\[0.12in]
                &= 2a^\mu\gamma_\mu - \slashed{a} \gamma^\mu\gamma_\mu = 2\slashed{a} - 4\slashed{a} = -2 \slashed{a} \qed 
            \end{align*}

        \item $\gamma^\mu \slashed{a} \slashed{b} \gamma_\mu = 4 a\cdot b$
        
            \begin{align*}
                \gamma^\mu \slashed{a} \slashed{b} \gamma_\mu = a_\nu b_\rho \gamma^\mu \gamma^\nu \gamma^\rho \gamma_\mu &= \frac{1}{2} a_\nu b_\rho \gamma^\mu \left\{ \gamma^\nu ,\gamma^\rho \right\} \gamma_\mu \\[0.12in]
                &= \frac{1}{2} a_\nu b_\rho 2 g^{\nu\rho} \gamma^\mu\gamma_\mu = 4 a\cdot b \qed 
            \end{align*}
    \end{enumerate}
\end{solution}

\noindent\rule{7in}{1.5pt}

%%%%%%%%%%%%%%%%%%%%%%%%%%%%%%%%%%%%%%%%%%%%%%%%%%%%%%%%%%%%%%%%%%%%%%%%%%%%%%%%%%%%%%%%%%%%%%%%%%%%%%%%%%%%%%%%%%%%%%%%%%%%%%%%%%%%%%%%

\begin{problem}{6.9}
Prove the relation $ \left(\overbar{\psi}\gamma^\mu\gamma^5\phi\right)^\dagger=\overbar{\phi}\gamma^\mu\gamma^5\psi$.
\end{problem}
\begin{solution}
One could show that :

\begin{align*}
    \left(\overbar{\psi}\gamma^\mu\gamma^5\phi\right)^\dagger = \left( \psi^\dagger \gamma^0 \gamma^\mu \gamma^5 \phi \right)^\dagger &= -  \left( \psi^\dagger \gamma^0  \gamma^5 \gamma^\mu \phi \right)^\dagger \\[0.12in]
    &= - \phi^\dagger \gamma^{\mu\dagger} \gamma^{5\dagger} \gamma^{0\dagger}  \psi \\[0.12in]
    &=  - \phi^\dagger \left(\gamma^0 \gamma^{\mu} \gamma^0\right) \gamma^5 \gamma^0  \psi \\[0.12in]
    &=  - \phi^\dagger \gamma^0    \gamma^\mu \gamma^0  \gamma^5 \gamma^0    \psi  = \phi^\dagger \gamma^0    \gamma^\mu \gamma^0  \gamma^0  \gamma^5   \psi =\overbar{\phi}\gamma^\mu\gamma^5\psi \qed 
\end{align*}
\end{solution}

\noindent\rule{7in}{1.5pt}

%%%%%%%%%%%%%%%%%%%%%%%%%%%%%%%%%%%%%%%%%%%%%%%%%%%%%%%%%%%%%%%%%%%%%%%%%%%%%%%%%%%%%%%%%%%%%%%%%%%%%%%%%%%%%%%%%%%%%%%%%%%%%%%%%%%%%%%%

\begin{problem}{6.10}
Use the trace formalism to calculate the QED spin-averaged matrix element squared for $e^+e^-\to ff$ including the electron mass term.
\end{problem}
\begin{solution}
The spin-averaged amplitude $\expval{\mathcal{M}_{fi}^2}$ for the $e^+e^-\to ff$ without neglecting both mass terms, can be written as 

\begin{align*}
    \expval{\mathcal{M}_{fi}^2} = \frac{1}{4} \sum_{\text{spins}} \left| \mathcal{M}_{fi}\right|^2 = \frac{Q_f^2 e^4}{4q^4} \tr\left[ (\slashed{p}_2 - m_e) \gamma^\mu (\slashed{p}_1 + m_e) \gamma^\nu \right] \times \tr\left[ (\slashed{p}_3 + m_f) \gamma_\mu (\slashed{p}_4 - m_f) \gamma_\nu \right]
\end{align*}\\
The first trace term can be calculated as,

\begin{align*}
    \tr\left[ (\slashed{p}_2 - m_e) \gamma^\mu (\slashed{p}_1 + m_e) \gamma^\nu \right] &= \tr \left[  \pslash_2\gamma^\mu\pslash_1\gamma^\nu + m_e\cancelto{0}{ \pslash_2 \gamma^\mu \gamma^\nu}  - m_e \cancelto{0}{\gamma^\mu \pslash_1 \gamma^\nu}    -m_e^2 \gamma^\mu \gamma^\nu \right] \\[0.12in]
    &= \tr \left[  p_{1\sigma}p_{2\rho}\gamma^\rho \gamma^\mu \gamma^\sigma \gamma^\nu -m_e^2 \gamma^\mu \gamma^\nu \right] \\[0.12in]
    &= p_{1\sigma}p_{2\rho} \tr \left[  \gamma^\rho \gamma^\mu \gamma^\sigma \gamma^\nu \right] -m_e^2  \tr \left[ \gamma^\mu \gamma^\nu \right] \\[0.12in]
    &= 4 p_{1\sigma}p_{2\rho} \left( g^{\rho\mu} g^{\sigma\nu} - g^{\rho\sigma}g^{\mu\nu} + g^{\rho\nu}g^{\mu\sigma} \right) -4 m_e^2 g^{\mu\nu} \\[0.12in]
    &= 4\left[ p_{1\nu}p_{2\mu} + p_{1\mu}p_{2\nu} -\left( p_1\cdot p_2 + m_e^2 \right) g^{\mu\nu}\right]
\end{align*}\\
Observing the second trace term, one could realize that switching $p_1\leftrightarrow p_3$, $p_2\leftrightarrow p_4$, $\mu \leftrightarrow \nu$ and $m_e \leftrightarrow m_f$ in the first trace term will be identical to the second one due to the fact that traces remain the same under cyclic permutations. Thus,

\begin{align*}
    \tr\left[ (\slashed{p}_3 + m_f) \gamma_\mu (\slashed{p}_4 - m_f) \gamma_\nu \right] = 4\left[ p_3^\mu p_4^\nu + p_3^\nu p_4^\mu -\left( p_3\cdot p_4 + m_f^2 \right) g_{\nu\mu}\right]
\end{align*}\\
Using these two trace expressions, the spin-averaged amplitude can be fully written down as 

\begin{align*}
    \expval{\mathcal{M}_{fi}^2}  &= 4Q_f^2  \frac{  e^4}{q^4}  \left[ p_{1\nu}p_{2\mu} + p_{1\mu}p_{2\nu} -\left( p_1\cdot p_2 + m_e^2 \right) g^{\mu\nu}\right] \times \left[ p_3^\mu p_4^\nu + p_3^\nu p_4^\mu -\left( p_3\cdot p_4 + m_f^2 \right) g_{\nu\mu}\right] \\[0.12in]
    &= 8Q_f^2  \frac{e^4}{q^4} \Big[  \left(p_1\cdot p_4\right)\left(p_2\cdot p_3\right) + \left(p_1\cdot p_3\right)\left(p_2\cdot p_4\right) + m_e^2 \left( p_3\cdot p_4 \right)  + m_f^2 \left( p_1\cdot p_2 \right) + 2 m_e^2  m_f^2   \Big] \\[0.12in]
    &=  \frac{ 2Q_f^2 e^4}{(m_e^2 + p_1 \cdot p_2)^2} \Big[  \left(p_1\cdot p_4\right)\left(p_2\cdot p_3\right) + \left(p_1\cdot p_3\right)\left(p_2\cdot p_4\right) + m_e^2 \left( p_3\cdot p_4 \right)  + m_f^2 \left( p_1\cdot p_2 \right) + 2 m_e^2  m_f^2   \Big] 
\end{align*}\\
Under the limit of $m_e \sim 0 $ the above coincides with equation (6.63) and going further by neglecting the fermion mass reduces to (6.25), which is as expected.
\end{solution}

\noindent\rule{7in}{1.5pt}

%%%%%%%%%%%%%%%%%%%%%%%%%%%%%%%%%%%%%%%%%%%%%%%%%%%%%%%%%%%%%%%%%%%%%%%%%%%%%%%%%%%%%%%%%%%%%%%%%%%%%%%%%%%%%%%%%%%%%%%%%%%%%%%%%%%%%%%%

\begin{problem}{6.11}
Neglecting the electron mass term, verify that the matrix element for $e^- f \to e^- f$ given in (6.67) can be obtained from the matrix element for $e^+e^-\to ff$ given in (6.63) using crossing symmetry with the substitutions

\begin{align*}
    p_1 \to p_1 \comma p_2 \to -p_3 \comma p_3 \to p_4 \andtxt p_4 \to -p_2 \\
\end{align*}
\end{problem}
\begin{solution}
    One could directly verify by substituting four-vectors as the problem required,

    \begin{align*}
        \expval{\mathcal{M}_{e^+e^-\to ff}^2} &= 2 \frac{Q_f^2 e^4}{\left(p_1\cdot p_2\right)^2} \left[ \left(p_1 \cdot p_3\right) \left( p_2 \cdot p_4 \right) + \left(p_1 \cdot p_4\right)\left( p_2 \cdot p_3 \right) + m_f^2 \left( p_1 \cdot p_2\right) \right] \\[0.12in]
        &\longrightarrow   2 \frac{Q_f^2 e^4}{\left(p_1\cdot p_3\right)^2} \left[ \left(p_1 \cdot p_4\right) \left(- p_3 \cdot- p_2 \right) + \left(p_1 \cdot -p_2\right)\left( -p_3 \cdot p_4 \right) + m_f^2 \left( p_1 \cdot - p_3\right) \right] \\[0.12in]
        &= 2 \frac{Q_f^2 e^4}{\left(p_1\cdot p_3\right)^2} \left[ \left(p_1 \cdot p_4\right) \left( p_2 \cdot  p_3 \right) + \left(p_1 \cdot p_2\right)\left( p_3 \cdot p_4 \right) - m_f^2 \left( p_1 \cdot   p_3\right) \right] =   \expval{\mathcal{M}_{e^-f\to e^-f}^2} \qed
    \end{align*}
\end{solution}

\noindent\rule{7in}{1.5pt}

%%%%%%%%%%%%%%%%%%%%%%%%%%%%%%%%%%%%%%%%%%%%%%%%%%%%%%%%%%%%%%%%%%%%%%%%%%%%%%%%%%%%%%%%%%%%%%%%%%%%%%%%%%%%%%%%%%%%%%%%%%%%%%%%%%%%%%%%

\begin{problem}{6.12}
Write down the matrix elements, $\mathcal{M}_1$ and $\mathcal{M}_2$, for the two Feynman diagrams for the Compton scattering process $e^-\gamma\to e^-\gamma$. From first principles, express the spin-averaged matrix element $\expval{|\mathcal{M}_1+\mathcal{M}_2|^2}$ as a trace. You will need the completeness relation for the photon polarisation states (see Appendix D).
\end{problem}
\begin{solution}
Starting from the two leading order Feynman diagrams that contribute to the Compton scattering process, using QED Feynman rules one could write down the corresponding matrix elements as, \\

\begin{equation}
    \begin{aligned}
        \begin{gathered}
        \scalebox{1.25}{\begin{tikzpicture}
            \begin{feynman}
                \vertex (a2) ;
                \vertex [above = 0.25em of a2] (a2l) {\(\scriptstyle \mu\)};
                \vertex at ($(a2) + (-4em,1.5em) $) (a1) {$e^-$};
                \vertex at ($(a2) + (4em,1.5em) $) (a3) {$\gamma$};
                
                \vertex [below=4.5em of a2] (b2);   
                \vertex [below = 0.25em of b2] (b2l) {\(\scriptstyle \nu\)};     
                \vertex at ($(b2) + (-4em,-1.5em) $) (b1) {$\gamma$};
                \vertex at ($(b2) + (4em,-1.5em) $) (b3) {$e^-$};
                
                \diagram* {
                    {[edges=fermion]
                      (a1) -- [edge label'=\(p_1\), near end] (a2) -- [edge label=\(\scriptstyle q\equiv p_3-p_1\)]  (b2) -- [edge label'=\(p_4\),near end]  (b3) ,
                    },
                      (b1) -- [boson, edge label'=\(p_2\), near start] (b2) , (a2) -- [boson,edge label'=\(p_3\), near start] (a3)
                };
            
            \end{feynman}
        \end{tikzpicture}
        }
        \end{gathered}  :  -i\mathcal{M}_1   &= e^2   \overbar{u}(p_4) \slashed{\epsilon} (p_2) \left[  \frac{i\left(\slashed{q}+m\right)}{q^2-m^2}\right]  \slashed{\epsilon}^\ast(p_3) u(p_1)  \nonumber \\[-0.5in]
        &= i e^2 \overbar{u}(p_4) \slashed{\epsilon} (p_2)   \left[  \frac{ \pslash_3-\pslash_1+m }{(p_3-p_1)^2-m^2}\right] \slashed{\epsilon}^\ast(p_3) u(p_1)  \equiv  i e^2 \overbar{u}(p_4) \Gamma^1 u(p_1)   \\[0.55in]
    \end{aligned}
\end{equation}
And for the other diagram, \\

\begin{equation}
    \begin{aligned}
        \begin{gathered}
        \scalebox{1.25}{\begin{tikzpicture}
            \begin{feynman}
                \vertex (g1);
                \vertex [left = 0.25em of g1] (g1l) {\(\scriptstyle \mu\)};
                \vertex at ($(g1) + (-2.5em,3.5em) $) (a) {\(\gamma\)};
                \vertex at ($(g1) + (-2.5em,-3.5em) $) (b) {\(e^-\)};

                \vertex at ($(g1) + (4em,0)$) (g2) ;
                \vertex [right = 0.25em of g2] (g2l) {\(\scriptstyle \nu\)};
                \vertex at ($(g2) + (2.5em,3.5em) $) (c) {\(\gamma\)};
                \vertex at ($(g2) + (2.5em,-3.5em) $) (d) {\(e^-\)};
                \diagram* {
                    {[edges=boson]
                    (a) -- [edge label=\(p_3\), near start]  (g1) ,  (g2) -- [edge label=\(p_2\), near end] (c)
                    },
                    {[edges=fermion]
                    (b) -- [edge label'=\(p_1\)] (g1) -- [edge label=\(\scriptstyle q\equiv p_1+p_3\)]  (g2) -- [edge label'=\(p_4\)] (d)
                    }
                };
            
            \end{feynman}
        \end{tikzpicture}
        }
        \end{gathered}  :  -i\mathcal{M}_2   &= e^2 \overbar{u}(p_4) \slashed{\epsilon}^\ast (p_2)   \left[  \frac{i\left(\slashed{q}+m\right)}{q^2-m^2}\right] \slashed{\epsilon}(p_3) u(p_1) \nonumber \\[-0.5in]
        &= i e^2 \overbar{u}(p_4) \slashed{\epsilon}^\ast (p_2)   \left[  \frac{ \pslash_1+\pslash_3+m }{(p_1+p_3)^2-m^2}\right]  \slashed{\epsilon}(p_3) u(p_1) \equiv  i e^2 \overbar{u}(p_4) \Gamma^2 u(p_1)     \\[0.55in]
    \end{aligned}
\end{equation}
In total, there will be 4 terms in the spin-averaged matrix element,

\begin{align*}
    \expval{|\mathcal{M}_1+\mathcal{M}_2|^2} = \expval{\mathcal{M}_1\mathcal{M}_1^\dagger} + \expval{\mathcal{M}_1\mathcal{M}_2^\dagger} + \expval{\mathcal{M}_2\mathcal{M}_1^\dagger} + \expval{\mathcal{M}_2\mathcal{M}_2^\dagger}
\end{align*}\\
which could be calculated term-by-term. 

\begin{enumerate}[label=(\alph*)]
    \item $\expval{\mathcal{M}_1\mathcal{M}_1^\dagger} $
    
    \begin{align}
        \expval{\mathcal{M}_1\mathcal{M}_1^\dagger} &= \frac{e^4}{4}  \sum_{\lambda,\lambda'} \sum_{s,s'} \left[  \overbar{u}^s_a(p_4) \Gamma^1_{ab}(\lambda,\lambda')u^{s'}_b(p_1)\right] \left[  \overbar{u}^{s'}_c(p_1)\overbar{\Gamma}^1_{cd}(\lambda,\lambda')u^s_d(p_4)\right] \nonumber \\[0.12in]
        &=  \frac{e^4}{4} \sum_{\lambda,\lambda'} \left[ \sum_{s} u^s_d(p_4) \overbar{u}^s_a(p_4) \right]  \left[ \sum_{s'}u^{s'}_b(p_1)   \overbar{u}^{s'}_c(p_1) \right] \Gamma^1_{ab}(\lambda,\lambda') \overbar{\Gamma}^1_{cd}(\lambda,\lambda') \nonumber \\[0.15in]
        &= \frac{e^4}{4}  (\pslash_4+m)_{da}  (\pslash_1+m)_{bc}  \sum_{\lambda,\lambda'} \Gamma^1_{ab}(\lambda,\lambda') \overbar{\Gamma}^1_{cd}(\lambda,\lambda') \label{P6.12.1}
    \end{align}\\
    Using the following expression for $\Gamma^1$,

    \begin{align*}
        \Gamma^1_{ab}(\lambda,\lambda') \equiv \epsilon^\lambda_\mu \epsilon^{\lambda' \ast}_\nu (\gamma^\mu T \gamma^\nu)_{ab} \quad \text{where} \quad T \equiv \frac{\pslash_3-\pslash_1+m}{(p_3-p_1)^2-m^2}
    \end{align*}\\
    which gives the adjoint as,

    \begin{align*}
        \overbar{\Gamma}^1 &\equiv \gamma^0 \Gamma^{1\dagger} \gamma^0  = \epsilon^{\lambda\ast}_{\mu'} \epsilon^{\lambda'}_{\nu'} \gamma^0 \gamma^{\nu\dagger} T^\dagger \gamma^{\mu\dagger} \gamma^0 =  \epsilon^{\lambda\ast}_{\mu'} \epsilon^{\lambda'}_{\nu'} \gamma^{\nu'} T \gamma^{\mu'} \\[0.15in]
        &\implies \overbar{\Gamma}^1_{cd}(\lambda,\lambda') = \epsilon^{\lambda\ast}_{\mu'} \epsilon^{\lambda'}_{\nu'} (\gamma^{\nu'} T \gamma^{\mu'})_{cd}
    \end{align*}\\
    Plugging in the above expressions back into equation (\ref{P6.12.1}) yields,

    \begin{align*}
        \expval{\mathcal{M}_1\mathcal{M}_1^\dagger} &= \frac{e^4}{4}  (\pslash_4+m)_{da}  (\pslash_1+m)_{bc}  \sum_{\lambda,\lambda'} \Gamma^1_{ab}(\lambda,\lambda') \overbar{\Gamma}^1_{cd}(\lambda,\lambda') \\[0.12in]
        &= \frac{e^4}{4}  (\pslash_4+m)_{da}  (\pslash_1+m)_{bc} \left[ \sum_{\lambda}  \epsilon^{\lambda\ast}_{\mu'} \epsilon^\lambda_\mu \right] \left[ \sum_{\lambda'}  \epsilon^{\lambda' \ast}_\nu \epsilon^{\lambda'}_{\nu'} \right]  (\gamma^\mu T \gamma^\nu)_{ab} (\gamma^{\nu'} T \gamma^{\mu'})_{cd} \\[0.12in]
        &= \frac{e^4}{4}  (\pslash_4+m)_{da}  (\pslash_1+m)_{bc} \left[ g_{\mu\mu'} g_{\nu\nu'} \right](\gamma^\mu T \gamma^\nu)_{ab} (\gamma^{\nu'} T \gamma^{\mu'})_{cd}  \\[0.12in]
        &= \frac{e^4}{4}  (\pslash_4+m)_{da}  (\pslash_1+m)_{bc} (\gamma^\mu T \gamma^\nu)_{ab} (\gamma_{\nu} T \gamma_{\mu})_{cd}  \\[0.12in]
        &= \frac{e^4}{4} \tr \left[ \gamma^\mu T \gamma^\nu (\pslash_1+m) \gamma_{\nu} T \gamma_{\mu} (\pslash_4+m )\right] 
    \end{align*}\\
    Fully expanding $T$, one could finally obtain 

    \begin{align}
        \expval{\mathcal{M}_1\mathcal{M}_1^\dagger} = \frac{e^4}{4\left[(p_3-p_1)^2-m^2\right]^2} \tr \left[ \gamma^\mu (\pslash_3-\pslash_1 +m) \gamma^\nu (\pslash_1 + m ) \gamma_\nu  (\pslash_3-\pslash_1 +m) \gamma_\mu (\pslash_4 + m ) \right] \label{P6.12.a.1}
    \end{align}
    \item $\expval{\mathcal{M}_1\mathcal{M}_2^\dagger} $
    
    \begin{align}
        \expval{\mathcal{M}_1\mathcal{M}_2^\dagger} &= \frac{e^4}{4}  \sum_{\lambda,\lambda'} \sum_{s,s'} \left[  \overbar{u}^s_a(p_4) \Gamma^1_{ab}(\lambda,\lambda')u^{s'}_b(p_1)\right] \left[  \overbar{u}^{s'}_c(p_1)\overbar{\Gamma}^2_{cd}(\lambda,\lambda')u^{s}_d(p_4)\right] \nonumber \\[0.12in]
        &=  \frac{e^4}{4} \sum_{\lambda,\lambda'} \left[ \sum_{s} u^s_d(p_4) \overbar{u}^s_a(p_4) \right]  \left[ \sum_{s'}u^{s'}_b(p_1)   \overbar{u}^{s'}_c(p_1) \right] \Gamma^1_{ab}(\lambda,\lambda') \overbar{\Gamma}^1_{cd}(\lambda,\lambda') \nonumber \\[0.15in]
        &= \frac{e^4}{4}  (\pslash_4+m)_{da}  (\pslash_1+m)_{bc}  \sum_{\lambda,\lambda'} \Gamma^1_{ab}(\lambda,\lambda') \overbar{\Gamma}^2_{cd}(\lambda,\lambda')  \label{P6.12.b}
    \end{align}\\
    Again, using the similar calculation used before, one could see that
    
    \begin{align}
        \Gamma^2_{cd} = \epsilon^{\lambda'\ast}_\mu \epsilon^{\lambda}_{\nu} \left( \gamma^\mu T' \gamma^\nu \right)_{cd} \andtxt \overbar{\Gamma}^2_{cd} = \epsilon^{\lambda'}_\mu \epsilon^{\lambda\ast}_{\nu} ( \gamma^{\nu} T' \gamma^{\mu} )_{cd} \quad \text{where} \quad T' \equiv \frac{\pslash_1+\pslash_3+m}{(p_1+p_3)^2 - m^2}
    \end{align}\\
    Then (\ref{P6.12.b}) becomes,

    \begin{align}
        \expval{\mathcal{M}_1\mathcal{M}_2^\dagger} &= \frac{e^4}{4} (\pslash_4+m)_{da}  (\pslash_1+m)_{bc} \left( \gamma^\mu T \gamma^\nu \right)_{ab} \left( \gamma_\nu T' \gamma_\mu \right)_{cd} \nonumber \\[0.12in]
        &=  \frac{e^4}{4} \Tr{\gamma^\mu T \gamma^\nu (\pslash_1+m) \gamma_\nu T' \gamma_\mu (\pslash_4+m)} \nonumber \\[0.12in]
        &=  \frac{e^4}{4\left[(p_1-p_3)^2-m^2\right]\left[(p_1+p_3)^2-m^2\right]} \Tr{\gamma^\mu (\pslash_3-\pslash_1+m) \gamma^\nu (\pslash_1+m) \gamma_\nu (\pslash_1+\pslash_3+m) \gamma_\mu (\pslash_4+m)} \label{P6.12.b.1}
    \end{align}

    \item $\expval{\mathcal{M}_2\mathcal{M}_1^\dagger} $
    
    \begin{align}
        \expval{\mathcal{M}_2\mathcal{M}_1^\dagger} &= \frac{e^4}{4}  \sum_{\lambda,\lambda'} \sum_{s,s'} \left[  \overbar{u}^s_a(p_4) \Gamma^2_{ab}(\lambda,\lambda')u^{s'}_b(p_1)\right] \left[  \overbar{u}^{s'}_c(p_1)\overbar{\Gamma}^1_{cd}(\lambda,\lambda')u^{s}_d(p_4)\right] \nonumber \\[0.12in]
        &=  \frac{e^4}{4} \sum_{\lambda,\lambda'} \left[ \sum_{s} u^s_d(p_4) \overbar{u}^s_a(p_4) \right]  \left[ \sum_{s'}u^{s'}_b(p_1)   \overbar{u}^{s'}_c(p_1) \right] \Gamma^2_{ab}(\lambda,\lambda') \overbar{\Gamma}^1_{cd}(\lambda,\lambda') \nonumber \\[0.15in]
        &= \frac{e^4}{4}  (\pslash_4+m)_{da}  (\pslash_1+m)_{bc}  \sum_{\lambda,\lambda'} \Gamma^2_{ab}(\lambda,\lambda') \overbar{\Gamma}^1_{cd}(\lambda,\lambda')  \label{P6.12.c}
    \end{align}\\
    Using expressions of $\Gamma^1,\Gamma^2$, one could further write down (\ref{P6.12.c}) as, 

    \begin{align}
        \expval{\mathcal{M}_2\mathcal{M}_1^\dagger} &=  \frac{e^4}{4}  (\pslash_4+m)_{da}  (\pslash_1+m)_{bc} \left( \gamma^\mu T' \gamma^\nu \right)_{ab}  \left( \gamma_\nu T \gamma_\mu \right)_{cd} \nonumber \\[0.12in]
        &=  \frac{e^4}{4} \Tr{    \gamma^\mu T' \gamma^\nu   (\pslash_1+m)   \gamma_\nu T \gamma_\mu   (\pslash_4+m)  } \nonumber \\[0.12in]
        &=  \frac{e^4}{4\left[(p_1-p_3)^2-m^2\right]\left[(p_1+p_3)^2-m^2\right]} \Tr{    \gamma^\mu  (\pslash_1+\pslash_3+m)  \gamma^\nu   (\pslash_1+m)   \gamma_\nu (\pslash_3-\pslash_1+m) \gamma_\mu   (\pslash_4+m)  }  \label{P6.12.c.1}
    \end{align}

    \item $\expval{\mathcal{M}_2\mathcal{M}_2^\dagger} $
    
    \begin{align}
        \expval{\mathcal{M}_2\mathcal{M}_2^\dagger} &= \frac{e^4}{4}  \sum_{\lambda,\lambda'} \sum_{s,s'} \left[  \overbar{u}^{s'}_a(p_4) \Gamma^2_{ab}(\lambda,\lambda')u^{s}_b(p_1)\right] \left[  \overbar{u}^{s}_c(p_1)\overbar{\Gamma}^2_{cd}(\lambda,\lambda')u^{s'}_d(p_4)\right] \nonumber \\[0.12in]
        &=  \frac{e^4}{4} \sum_{\lambda,\lambda'} \left[ \sum_{s} u^s_b(p_1) \overbar{u}^s_c(p_1) \right]  \left[ \sum_{s'} u^{s'}_d(p_4)   \overbar{u}^{s'}_a(p_4) \right] \Gamma^2_{ab}(\lambda,\lambda') \overbar{\Gamma}^2_{cd}(\lambda,\lambda') \nonumber \\[0.15in]
        &= \frac{e^4}{4}  (\pslash_1+m)_{bc}  (\pslash_4+m)_{da}  \sum_{\lambda,\lambda'} \Gamma^2_{ab}(\lambda,\lambda') \overbar{\Gamma}^2_{cd}(\lambda,\lambda') \nonumber \\[0.12in]
        &= \frac{e^4}{4}  (\pslash_1+m)_{bc}  (\pslash_4+m)_{da} \left( \gamma^\mu T' \gamma^\nu \right)_{ab} \left( \gamma_\nu T' \gamma_\mu \right)_{cd} \nonumber \\[0.12in]
        &= \frac{e^4}{4\left[(p_1+p_3)^2-m^2\right]^2} \tr \left[ \gamma^\mu (\pslash_1+\pslash_3 +m) \gamma^\nu (\pslash_1 + m ) \gamma_\nu  (\pslash_1+\pslash_3 +m) \gamma_\mu (\pslash_4 + m ) \right] \label{P6.12.d.1}
    \end{align}\\
\end{enumerate}

Then, adding up (\ref{P6.12.a.1}),(\ref{P6.12.b.1}),(\ref{P6.12.c.1}) and (\ref{P6.12.d.1}) one could get the total spin-averaged matrix element as, 

\begin{align*}
    \expval{|\mathcal{M}_1+\mathcal{M}_2|^2} &= \frac{e^4}{4\left[(p_1-p_3)^2-m^2\right]^2} \tr \left[ \gamma^\mu (\pslash_3-\pslash_1 +m) \gamma^\nu (\pslash_1 + m ) \gamma_\nu  (\pslash_3-\pslash_1 +m) \gamma_\mu (\pslash_4 + m ) \right] \\[0.12in]
    &+  \frac{e^4}{4\left[(p_1-p_3)^2-m^2\right]\left[(p_1+p_3)^2-m^2\right]} \left\{ \Tr{\gamma^\mu (\pslash_3-\pslash_1+m) \gamma^\nu (\pslash_1+m) \gamma_\nu (\pslash_1+\pslash_3+m) \gamma_\mu (\pslash_4+m)}  \right. \\[0.08in]
    &\hspace{2.in} + \left. \Tr{    \gamma^\mu  (\pslash_1+\pslash_3+m)  \gamma^\nu   (\pslash_1+m)   \gamma_\nu (\pslash_3-\pslash_1+m) \gamma_\mu   (\pslash_4+m)  }  \right\}  \\[0.12in]
    &+ \frac{e^4}{4\left[(p_1+p_3)^2-m^2\right]^2} \tr \left[ \gamma^\mu (\pslash_1+\pslash_3 +m) \gamma^\nu (\pslash_1 + m ) \gamma_\nu  (\pslash_1+\pslash_3 +m) \gamma_\mu (\pslash_4 + m ) \right] 
\end{align*}
\end{solution}



%%%%%%%%%%%%%%%%%%%%%%%%%%%%%%%%%%%%%%%%%%%%%%%%%%%%%%%%%%%%%%%%%%%%%%%%%%%%%%%%%%%%%%%%%%%%%%%%%%%%%%%%%%%%%%%%%%%%%%%%%%%%%%%%%%%%%%%%
