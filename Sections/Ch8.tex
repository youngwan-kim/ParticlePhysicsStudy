
\noindent\rule{7in}{2.8pt}
\section{Deep Inelastic Scattering}
    
\begin{problem}{8.1}
Use the data in Figure 8.2 to estimate the lifetime of the $\Delta^+$ baryon.
\end{problem}
\begin{solution}

\end{solution}

\noindent\rule{7in}{1.5pt}

%%%%%%%%%%%%%%%%%%%%%%%%%%%%%%%%%%%%%%%%%%%%%%%%%%%%%%%%%%%%%%%%%%%%%%%%%%%%%%%%%%%%%%%%%%%%%%%%%%%%%%%%%%%%%%%%%%%%%%%%%%%%%%%%%%%%%%%%

\begin{problem}{8.2}
In fixed-target electron-proton elastic scattering

\begin{align*}
    Q^2 = 2m_p \left(E_1-E_2\right) = 2m_p E_{1y} \andtxt Q^2 = 4E_1E_3 \sin^2 \frac{\theta}{2}
\end{align*}

\begin{enumerate}[label=(\alph*)]
    \item Use these relations to show that 
    
    \begin{align*}
        \sin^2\frac{\theta}{2} = \frac{E_1}{E_3} \frac{m_p^2}{Q^2} y^2 \quad \text{and hence} \quad \frac{E_3}{E_1}\cos^2\frac{\theta}{2} = 1-y - \frac{m_p^2 y^2}{Q^2}.
    \end{align*}

    \item Assuming azimuthal symmetry and using Equations (7.31) and (7.32), show that
    
    \begin{align*}
        \frac{\dif \sigma}{\dif Q^2} = \left|  \frac{\dif \Omega}{\dif Q^2} \right| \frac{\dif \sigma }{\dif \Omega} = \frac{\pi}{E_3^2} \frac{\dif\sigma}{\dif\Omega}
    \end{align*}

    \item Using the results of (a) and (b) show that the Rosenbluth equation,
    
    \begin{align*}
        \frac{\dif\sigma}{\dif\Omega} = \frac{\alpha^2}{4E_1^2 \sin^4\halftheta} \frac{E_3}{E_1} \left( \frac{G_E^2+\tau G_M^2}{1+\tau} \cos^2 \halftheta + 2\tau G_M^2 \sin^2\halftheta \right),
    \end{align*}\\
    can be written in the Lorentz-invariant form

    \begin{align*}
        \frac{\dif\sigma}{\dif\Omega} = \frac{4\pi\alpha^2}{Q^4} \left[ \frac{G_E^2 + \tau G_M^2}{1+\tau} \left( 1 - y - \frac{m_p^2 y^2}{Q^2} \right) + \frac{1}{2}y^2 G_M^2 \right]
    \end{align*}
\end{enumerate}
\end{problem}
\begin{solution}

\end{solution}

\noindent\rule{7in}{1.5pt}

%%%%%%%%%%%%%%%%%%%%%%%%%%%%%%%%%%%%%%%%%%%%%%%%%%%%%%%%%%%%%%%%%%%%%%%%%%%%%%%%%%%%%%%%%%%%%%%%%%%%%%%%%%%%%%%%%%%%%%%%%%%%%%%%%%%%%%%%

\begin{problem}{8.3}
    In fixed-target electron–proton inelastic scattering:
    \begin{enumerate}[label=(\alph*)]
        \item Show that the laboratory frame differential cross section for deep-inelastic scattering is related to the Lorentz-invariant differential cross section of Equation (8.11) by
        
        \begin{align*}
            \frac{\dif^2 \sigma}{\dif E_3 \dif \Omega} = \frac{E_1E_3}{\pi} \frac{\dif^2 \sigma}{\dif E_3 \dif Q^2} =  \frac{E_1E_3}{\pi} \frac{2m_px^2}{Q^2} \frac{\dif^2 \sigma}{\dif x \dif Q^2},
        \end{align*}\\
        where $E_1,E_3$ are the energies of the incoming and outgoing electron. 

        \item Show that
        
        \begin{align*}
            \frac{2m_p x^2}{Q^2} \cdot \frac{y^2}{2} = \frac{1}{m_p} \frac{E_3}{E_1} \sin^2 \halftheta \andtxt 1-y-\frac{m_p^2x^2y^2}{Q^2} = \frac{E_3}{E_1} \cos^2 \halftheta.
        \end{align*}

        \item Hence, show that the Lorentz-invariant cross section of Equation (8.11) becomes
        
        \begin{align*}
            \frac{\dif^2 \sigma}{\dif E_3 \dif \Omega} = \frac{\alpha^2}{4E_1^2 \sin^4 \halftheta} \left[ \frac{F_2}{\nu}\cos^2\halftheta + \frac{2F_1}{m_p^2} \sin^2 \halftheta \right].
        \end{align*}

        \item A fixed-target ep scattering experiment consists of an electron beam of maximum energy $20 \GeV$ and a variable angle spectrometer that can detect scattered electrons with energies greater than $2 \GeV$. Find the range of values of $\theta$ over which deep inelastic scattering events can be studied at $x = 0.2$ and $Q^2 = 2 \GeV^2$
    \end{enumerate}
\end{problem}
\begin{solution}

\end{solution}

\noindent\rule{7in}{1.5pt}

%%%%%%%%%%%%%%%%%%%%%%%%%%%%%%%%%%%%%%%%%%%%%%%%%%%%%%%%%%%%%%%%%%%%%%%%%%%%%%%%%%%%%%%%%%%%%%%%%%%%%%%%%%%%%%%%%%%%%%%%%%%%%%%%%%%%%%%%

\begin{problem}{8.4}
    If quarks were spin-0 particles, why would $F^{ep}_1(x)/F^{ep}_2(x)$ be zero?
\end{problem}
\begin{solution}

\end{solution}

\noindent\rule{7in}{1.5pt}

%%%%%%%%%%%%%%%%%%%%%%%%%%%%%%%%%%%%%%%%%%%%%%%%%%%%%%%%%%%%%%%%%%%%%%%%%%%%%%%%%%%%%%%%%%%%%%%%%%%%%%%%%%%%%%%%%%%%%%%%%%%%%%%%%%%%%%%%

\begin{problem}{8.5}
    What is the expected value of $\int_0^1 u(x) - \overbar{u}(x) \dif x$ for the proton?
\end{problem}
\begin{solution}

\end{solution}

\noindent\rule{7in}{1.5pt}

%%%%%%%%%%%%%%%%%%%%%%%%%%%%%%%%%%%%%%%%%%%%%%%%%%%%%%%%%%%%%%%%%%%%%%%%%%%%%%%%%%%%%%%%%%%%%%%%%%%%%%%%%%%%%%%%%%%%%%%%%%%%%%%%%%%%%%%%

\begin{problem}{8.6}
Figure 8.18 shows the raw measurements of the structure function $F_2(x)$ in low-energy electron-deuterium scattering. When combined with the measurements of Figure 8.11, it is found that

\begin{align*}
    \frac{\int_0^1 F^{eD}_2(x)}{\int_0^1 F^{ep}_2(x)} \simeq 0.84
\end{align*}\\
Write down the quark–parton model prediction for this ratio and determine the relative fraction of the momen- tum of proton carried by down-/anti-down-quarks compared to that carried by the up-/anti-up-quarks, $f_d/f_u$
\end{problem}
\begin{solution}

\end{solution}

\noindent\rule{7in}{1.5pt}

%%%%%%%%%%%%%%%%%%%%%%%%%%%%%%%%%%%%%%%%%%%%%%%%%%%%%%%%%%%%%%%%%%%%%%%%%%%%%%%%%%%%%%%%%%%%%%%%%%%%%%%%%%%%%%%%%%%%%%%%%%%%%%%%%%%%%%%%

\begin{problem}{8.7}
    Including the contribution from strange quarks:

    \begin{enumerate}[label=(\alph*)]
        \item Show that $F_2^{ep}(x)$ can be written 
        
        \begin{align*}
            F_2^{\text{ep}}(x) = \frac{4}{9} x \left[ u(x) + \bar{u}(x) \right] + \frac{1}{9} x \left[ d(x) + \bar{d}(x) + s(x) + \bar{s}(x) \right],
        \end{align*}\\
        where $s(x)$ and $\bar{s}(x)$ are the strange quark-parton distribution functions of the proton.
        \item Find the corresponding expression for $F_2^{en}(x)$ and show that
        
        \begin{align*}
            \int_0^1 \frac{F^{\text{ep}}_2(x)-F^{\text{en}}_2(x)}{x} \dif x \approx \frac{1}{3} + \frac{2}{3} \int_0^1 \left[ \bar{u}(x) - \bar{d}(x) \right] \dif x
        \end{align*}\\
        and interpret the measured value of $0.24\pm 0.03$
    \end{enumerate}
\end{problem}
\begin{solution}

\end{solution}

\noindent\rule{7in}{1.5pt}

%%%%%%%%%%%%%%%%%%%%%%%%%%%%%%%%%%%%%%%%%%%%%%%%%%%%%%%%%%%%%%%%%%%%%%%%%%%%%%%%%%%%%%%%%%%%%%%%%%%%%%%%%%%%%%%%%%%%%%%%%%%%%%%%%%%%%%%%

\begin{problem}{8.8}
At the HERA collider, electrons of energy $E_1 = 27.5 \GeV$ collided with protons of energy $E_2 = 820 \GeV$. In deep inelastic scattering events at HERA, show that the Bjorken x is given by

\begin{align*}
    x = \frac{E_3}{E_2} \left[ \frac{1-\cos\theta}{2-\frac{E_3}{E_1}\left(1+\cos\theta\right)} \right]
\end{align*}\\
where $\theta$ is the angle through which the electron has scattered and $E_3$ is the energy of the scattered electron. Estimate $x$ and $Q^2$ for the event shown in Figure 8.13 assuming that the energy of the scattered electron is $250 \GeV$.
\end{problem}
\begin{solution}

\end{solution}

\noindent\rule{7in}{1.5pt}

%%%%%%%%%%%%%%%%%%%%%%%%%%%%%%%%%%%%%%%%%%%%%%%%%%%%%%%%%%%%%%%%%%%%%%%%%%%%%%%%%%%%%%%%%%%%%%%%%%%%%%%%%%%%%%%%%%%%%%%%%%%%%%%%%%%%%%%%
