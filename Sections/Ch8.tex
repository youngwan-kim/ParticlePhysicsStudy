
\noindent\rule{7in}{2.8pt}
\section{Deep Inelastic Scattering}
    
\begin{problem}{8.1}
Use the data in Figure 8.2 to estimate the lifetime of the $\Delta^+$ baryon.
\end{problem}
\begin{solution}
One could easily notice that the corresponding FWHM $\Gamma_{\Delta^+}$ for the $\Delta^+$ baryon which resonates around $W=1.232\GeV$ or $E_3 \approx 4.2 \GeV$ is around $\Gamma_{\Delta^+} \simeq 0.1 \GeV$. Thus the lifetime $\tau$ can be simply estimated by inverting $\Gamma_{\Delta^+}$, 

\begin{align*}
    \tau_{\Delta^+} = \Gamma_{\Delta^+}^{-1} \simeq 10 \GeV^{-1} = 10 \cdot \num{6.58e-25} \unit{s} \simeq \boxed{ \num{0.6e-23} \unit{s}} 
\end{align*}\\
which is a good estimation compared to the value stated in the \href{https://pdg.lbl.gov/2023/tables/rpp2023-tab-baryons-Delta.pdf}{\texttt{PDG summary table}}.
\end{solution}

\noindent\rule{7in}{1.5pt}

%%%%%%%%%%%%%%%%%%%%%%%%%%%%%%%%%%%%%%%%%%%%%%%%%%%%%%%%%%%%%%%%%%%%%%%%%%%%%%%%%%%%%%%%%%%%%%%%%%%%%%%%%%%%%%%%%%%%%%%%%%%%%%%%%%%%%%%%

\begin{problem}{8.2} \customlabel{P8.2}{8.2}
In fixed-target electron-proton elastic scattering
\begin{align*}
    Q^2 = 2m_p \left(E_1-E_2\right) = 2m_p E_{1}y \andtxt Q^2 = 4E_1E_3 \sin^2 \frac{\theta}{2}
\end{align*}

\begin{enumerate}[label=(\alph*)]
    \item Use these relations to show that 

    \begin{align*}
        \sin^2\frac{\theta}{2} = \frac{E_1}{E_3} \frac{m_p^2}{Q^2} y^2 \quad \text{and hence} \quad \frac{E_3}{E_1}\cos^2\frac{\theta}{2} = 1-y - \frac{m_p^2 y^2}{Q^2}.
    \end{align*}

    \item Assuming azimuthal symmetry and using Equations (7.31) and (7.32), show that
    
    \begin{align*}
        \frac{\dif \sigma}{\dif Q^2} = \left|  \frac{\dif \Omega}{\dif Q^2} \right| \frac{\dif \sigma }{\dif \Omega} = \frac{\pi}{E_3^2} \frac{\dif\sigma}{\dif\Omega}
    \end{align*}

    \item Using the results of (a) and (b) show that the Rosenbluth equation,
    
    \begin{align*}
        \frac{\dif\sigma}{\dif\Omega} = \frac{\alpha^2}{4E_1^2 \sin^4\halftheta} \frac{E_3}{E_1} \left( \frac{G_E^2+\tau G_M^2}{1+\tau} \cos^2 \halftheta + 2\tau G_M^2 \sin^2\halftheta \right),
    \end{align*}\\
    can be written in the Lorentz-invariant form

    \begin{align*}
        \frac{\dif\sigma}{\dif Q^2} = \frac{4\pi\alpha^2}{Q^4} \left[ \frac{G_E^2 + \tau G_M^2}{1+\tau} \left( 1 - y - \frac{m_p^2 y^2}{Q^2} \right) + \frac{1}{2}y^2 G_M^2 \right]
    \end{align*}\\
\end{enumerate}
\end{problem}
\begin{solution}
    \begin{enumerate}[label=(\alph*)]
        \item It is straightforward to see that, 
            
            \begin{align*}
                Q^2 = 4E_1E_3\sin^2\frac{\theta}{2} \implies \sin^2 \frac{\theta}{2} &= \frac{Q^2}{4E_1E_3}  = \frac{2m_p E_{1}y}{2 E_3} \\[0.12in]
                &= \frac{m_p y}{2 E_3 } \times \frac{2E_1 y}{2 E_1 y}\\[0.12in]
                &= \frac{m_p y}{2 E_3} \times \frac{2E_1 y}{Q^2 m_p^{-1}} = \frac{E_1}{E_3} \frac{m_p^2}{Q^2}y^2 \qed
            \end{align*}\\
            and also,

            \begin{align*}
                \frac{E_3}{E_1}\cos^2\frac{\theta}{2} &= \frac{E_3}{E_1} \left(1 - \sin^2 \frac{\theta}{2}\right) \\[0.12in]
                &= \frac{E_3}{E_1} \left(  1 - \frac{E_1}{E_3} \frac{m_p^2}{Q^2}y^2  \right) \\[0.12in]
                &= \frac{E_3}{E_1} - \frac{m_p^2}{Q^2}y^2 = 1 - y - \frac{m_p^2}{Q^2}y^2 \qed
            \end{align*}\\
            \item From azimuthal symmetry, one could let $\dif \Omega = \dif\left(  \cos \theta \right) \dif \phi = 2\pi \dif\left(  \cos \theta \right)$ which leads to : 
            
                \begin{align*}
                    \left|\frac{\dif Q^2}{ \dif \Omega}\right| &= \frac{1}{2\pi} \left| \frac{\dif Q^2}{ \dif \left( \cos \theta \right)} \right| \\[0.12in]
                    &= \frac{1}{2\pi} \left| \frac{\dif}{\dif \left( \cos\theta \right)} \left[ \frac{2m_p E_1^2 \left( 1-\cos\theta \right)}{ m_p + E_1 \left( 1-\cos\theta \right) }  \right]\right| \\[0.12in]
                    &= \frac{1}{2\pi} \left| \frac{-2m_p^2 E_1^2}{\left[  m_p + E_1 \left( 1-\cos\theta \right) \right]^2} \right| = \frac{1}{\pi} E_3^2 
                \end{align*}\\
                Thus it could be written as, 

                \begin{align}
                    \frac{\dif \sigma}{\dif Q^2} = \left|  \frac{\dif Q^2}{\dif \Omega} \right|^{-1} \frac{\dif \sigma }{\dif \Omega} = \frac{\pi}{E_3^2} \frac{\dif\sigma}{\dif\Omega} \qed \label{P8.2b}
                \end{align}\\
                \item Starting from the relation (\ref{P8.2b}), 
                
                    \begin{align*}
                        \frac{\dif \sigma}{\dif Q^2} = \frac{\pi}{E_3^2} \frac{\dif \sigma}{\dif \Omega} &= \frac{\pi}{\color{pinegreen} E_3^2} \cdot  \frac{\alpha^2}{\color{pinegreen} 4E_1^2 \sin^4\halftheta} \frac{E_3}{E_1} \left( \frac{G_E^2+\tau G_M^2}{1+\tau} \cos^2 \halftheta + 2\tau G_M^2 \sin^2\halftheta \right) \\[0.12in]
                        &=  \alpha^2 \pi {\color{pinegreen} \frac{4}{Q^4} }\left[ \left( \frac{G_E^2+\tau G_M^2}{1+\tau} \right) {\color{red} \frac{E_3}{E_1}  \cos^2 \halftheta }+ 2\tau G_M^2 {\color{blue} \frac{E_3}{E_1}  \sin^2\halftheta} \right] \\[0.12in]
                        &=   {  \frac{4 \alpha^2 \pi }{Q^4} }\left[   \frac{G_E^2+\tau G_M^2}{1+\tau}  {\color{red} \left( 1-y-\frac{m_p^2}{Q^2}y^2 \right) }+ 2{\color{magenta} \tau} G_M^2 {\color{blue} \frac{m_p^2}{Q^2}y^2 } \right] \\[0.12in]
                        &=  {  \frac{4 \alpha^2 \pi }{Q^4} }\left[   \frac{G_E^2+\tau G_M^2}{1+\tau}  { \left( 1-y-\frac{m_p^2}{Q^2}y^2 \right) }+ 2{\color{magenta} \frac{Q^2}{4m_p^2}} G_M^2 {  \frac{m_p^2}{Q^2}y^2 } \right] \\[0.12in]
                        &=  \frac{4\pi\alpha^2}{Q^4} \left[ \frac{G_E^2 + \tau G_M^2}{1+\tau} \left( 1 - y - \frac{m_p^2 y^2}{Q^2} \right) + \frac{1}{2}y^2 G_M^2 \right] \qed
                    \end{align*}\\
    \end{enumerate}
\end{solution}

\noindent\rule{7in}{1.5pt}

%%%%%%%%%%%%%%%%%%%%%%%%%%%%%%%%%%%%%%%%%%%%%%%%%%%%%%%%%%%%%%%%%%%%%%%%%%%%%%%%%%%%%%%%%%%%%%%%%%%%%%%%%%%%%%%%%%%%%%%%%%%%%%%%%%%%%%%%

\begin{problem}{8.3}
    In fixed-target electron-proton inelastic scattering:
    \begin{enumerate}[label=(\alph*)]
        \item Show that the laboratory frame differential cross section for deep-inelastic scattering is related to the Lorentz-invariant differential cross section of Equation (8.11) by
        
        \begin{align*}
            \frac{\dif^2 \sigma}{\dif E_3 \dif \Omega} = \frac{E_1E_3}{\pi} \frac{\dif^2 \sigma}{\dif E_3 \dif Q^2} =  \frac{E_1E_3}{\pi} \frac{2m_px^2}{Q^2} \frac{\dif^2 \sigma}{\dif x \dif Q^2},
        \end{align*}\\
        where $E_1,E_3$ are the energies of the incoming and outgoing electron. 

        \item Show that
        
        \begin{align*}
            \frac{2m_p x^2}{Q^2} \cdot \frac{y^2}{2} = \frac{1}{m_p} \frac{E_3}{E_1} \sin^2 \halftheta \andtxt 1-y-\frac{m_p^2x^2y^2}{Q^2} = \frac{E_3}{E_1} \cos^2 \halftheta.
        \end{align*}

        \item Hence, show that the Lorentz-invariant cross section of Equation (8.11) becomes
        
        \begin{align*}
            \frac{\dif^2 \sigma}{\dif E_3 \dif \Omega} = \frac{\alpha^2}{4E_1^2 \sin^4 \halftheta} \left[ \frac{F_2}{\nu}\cos^2\halftheta + \frac{2F_1}{m_p} \sin^2 \halftheta \right].
        \end{align*}

        \item A fixed-target ep scattering experiment consists of an electron beam of maximum energy $20 \GeV$ and a variable angle spectrometer that can detect scattered electrons with energies greater than $2 \GeV$. Find the range of values of $\theta$ over which deep inelastic scattering events can be studied at $x = 0.2$ and $Q^2 = 2 \GeV^2$
    \end{enumerate}
\end{problem}
\begin{solution}
    \begin{enumerate}[label=(\alph*)]
        \item From the chain rule of derivations, the following holds 
        
            \begin{align*}
                \frac{\dif^2 \sigma}{\dif E_3 \dif \Omega} &= \left| \frac{\dif E_3}{ \dif x}\right|^{-1} \cdot \left| \frac{\dif Q^2}{\dif \Omega} \right|^{-1} \frac{\dif^2 \sigma}{\dif x \dif \Omega}  \\[0.12in]
                &= \left| \frac{\dif E_3}{ \dif x}\right|^{-1} \cdot \left| \frac{1}{2\pi}\frac{\dif }{\dif \left( \cos\theta \right)} \left( p_1 - p_3 \right)^2 \right|^{-1} \frac{\dif^2 \sigma}{\dif x \dif \Omega}  \\[0.12in]
                &=  \left| \frac{\dif E_3}{ \dif x}\right|^{-1} \cdot \pi \left|  \frac{\dif }{\dif \left( \cos\theta \right)} \left[ m_e^2 - E_1 E_3 \left( 1-\cos\theta  \right) \right] \right|^{-1} \frac{\dif^2 \sigma}{\dif x \dif \Omega} \\[0.12in]
                &= \left| \frac{\dif E_3}{ \dif x}\right|^{-1} \cdot  \frac{E_1E_3}{\pi} \frac{\dif^2 \sigma}{\dif x \dif \Omega} \\[0.12in]
                &=  \left| \frac{\dif }{ \dif x} \left( E_1 - \frac{Q^2}{2m_p x} \right) \right|^{-1} \cdot  \frac{E_1E_3}{\pi} \frac{\dif^2 \sigma}{\dif x \dif \Omega}  \\[0.12in]
                &= \frac{2m_p x^2}{Q^2} \frac{E_1E_3}{\pi}  \frac{\dif^2 \sigma}{\dif x \dif Q^2} \qed
            \end{align*}\\
        \item  Starting from the relation of $Q^2 = - \left( p_1 - p_3 \right)^2 \simeq 2E_1 E_3 \left( 1-\cos\theta \right) $, one could write down
        
            \begin{align}
                Q^2 = 2E_1 E_3 \left( 1-\cos\theta \right) = 4 E_1 E_3 \sin^2 \halftheta = 2m_p \nu x &\implies \sin^2\halftheta = \frac{ m_p \nu x}{2E_1E_3} = \frac{ m_p  xy}{2 E_3} \nonumber \\[0.12in]
                &\implies \frac{1}{m_p} \frac{E_3}{E_1} \sin^2\halftheta =  \frac{1}{m_p} \frac{E_3}{E_1}\frac{ m_p  xy}{2 E_3} \label{P8.3.b}
            \end{align}\\
        Also from the relation between $Q^2$ and $E_1$

            \begin{align*}
                Q^2 \simeq \left( s-m_p^2 \right) xy = 2E_1 m_p xy \implies E_1 = \frac{Q^2}{2m_p xy}
            \end{align*}\\
        Plugging this into (\ref{P8.3.b}) yields,

            \begin{align*}
                \frac{1}{m_p} \frac{E_3}{E_1} \sin^2\halftheta = \frac{1}{m_p} \frac{E_3}{E_1}\frac{ m_p  xy}{2 E_3} = \frac{m_p }{Q^2}x^2 y^2 \qed
            \end{align*}\\
        Then using the trigonometric relation, 

            \begin{align*}
                \frac{E_3}{E_1}\cos^2 \halftheta  &=  \frac{E_3}{E_1} -  \frac{E_3}{E_1}\sin^2 \halftheta  \\[0.12in]
                &= \frac{E_3}{E_1} -  \frac{m_p^2 }{Q^2}x^2 y^2  = 1-y-\frac{m_p^2 }{Q^2}x^2 y^2 \qed
            \end{align*}\\
        \item From the above results,
        
            \begin{align*}
                \frac{\dif^2 \sigma}{\dif E_3 \dif \Omega} &=  \frac{2m_p x^2}{Q^2} \frac{E_1E_3}{\pi} \frac{\dif^2 \sigma}{\dif x \dif Q^2} \\[0.12in]
                &= \frac{2m_p x^2}{Q^2} \frac{E_1E_3}{\pi} \frac{4\pi\alpha^2}{\color{pinegreen} Q^4} \left[  { \color{red} \left( 1-y-\frac{m_p^2 y^2}{Q^2} \right)} \frac{F_2}{x} + {\color{blue} y^2} F_1 \right] \\[0.12in]
                &=  \frac{2m_p x^2}{Q^2} \frac{E_1E_3}{\pi} \frac{4\pi\alpha^2}{\color{pinegreen} 16 E_1^2 E_3^2 \sin^4 \halftheta} \left[  {\color{red} \frac{E_3}{E_1} \cos^2 \halftheta} \frac{F_2}{x} + {\color{blue} \frac{1}{x^2 m_p^2} \frac{E_3}{E_1} \sin^2\halftheta  Q^2 }F_1 \right] \\[0.12in]
                &= \frac{\alpha^2}{4E_1^2 \sin^4 \halftheta} \left( \frac{2m_p x^2}{Q^2} \cancelto{}{\frac{E_3}{E_1}} \right)\left[   \cancelto{}{\frac{E_3}{E_1}} \cos^2 \halftheta  \frac{F_2}{x} +   \frac{1}{x^2 m_p^2} \cancelto{}{\frac{E_3}{E_1}} \sin^2\halftheta  Q^2 F_1 \right] \\[0.12in]
                &= \frac{\alpha^2}{4E_1^2 \sin^4 \halftheta}  \left[     \cos^2 \halftheta  F_2 \frac{2 m_p x}{ \color{magenta} Q^2} +     \sin^2\halftheta    F_1 \frac{2}{m_p} \right] \\[0.12in]
                &= \frac{\alpha^2}{4E_1^2 \sin^4 \halftheta}  \left[     \cos^2 \halftheta  F_2 \frac{2 m_p x}{ \color{magenta} 2 m_p x \nu} +     \sin^2\halftheta    F_1 \frac{2}{m_p} \right] \\[0.12in]
                &=  \frac{\alpha^2}{4E_1^2 \sin^4 \halftheta} \left[ \frac{F_2}{\nu}\cos^2\halftheta + \frac{2F_1}{m_p} \sin^2 \halftheta \right] \qed
            \end{align*}\\
        \item From the given values, one could find that 
        
            \begin{align*}
                \nu = \frac{Q^2}{2m_p x} \simeq \frac{2 \GeV^2}{2 \cdot 0.938 \GeV \cdot 0.2} \simeq 5.3 \GeV \implies E_3 = E_1 - 5.3 \GeV \in \left[ 2,14.7 \right] \GeV
            \end{align*}\\
            where the minimum value of 2 GeV is from the detector resolution mentioned in the problem. Using the expression of $\sin^2 \theta/2$ derived in (b), one could obtain 

            \begin{align*}
                \sin^2 \halftheta = \frac{E_1}{E_3} \frac{m_p^2}{Q^2} x^2 y^2 = \frac{1}{E_1E_3} {\color{red}\frac{m_p^2}{Q^2} x^2 \nu^2} \simeq \frac{\color{red} 2.5 \GeV^2}{E_3 \left( E_3 + 5.3 \GeV \right)}  \quad \text{for} \quad E_3 \in  \left[ 2,14.7 \right] \GeV
            \end{align*}\\
            which converts into an expression of angle as,

            \begin{align*}
                \theta \simeq 2 \arcsin \sqrt{\frac{2.5 \GeV^2}{E_3 \left( E_3 + 5.3 \GeV \right)}} \quad \text{for} \quad E_3 \in  \left[ 2,14.7 \right] \GeV
            \end{align*}\\
            it can be checked that in the considered region $\theta(E_3)$ decreases monotonically, thus the range for $\theta$ can be derived as $\boxed{  10.3^\circ \lesssim  \theta \lesssim  48.8^\circ }$.
    \end{enumerate}
\end{solution}

\noindent\rule{7in}{1.5pt}

%%%%%%%%%%%%%%%%%%%%%%%%%%%%%%%%%%%%%%%%%%%%%%%%%%%%%%%%%%%%%%%%%%%%%%%%%%%%%%%%%%%%%%%%%%%%%%%%%%%%%%%%%%%%%%%%%%%%%%%%%%%%%%%%%%%%%%%%

\begin{problem}{8.4}
    If quarks were spin-0 particles, why would $F^{ep}_1(x)/F^{ep}_2(x)$ be zero?
\end{problem}
\begin{solution}
One should remind that the angular dependence of electron-proton cross sections arise from the helicity conservation of electrons and quarks, which is due to the half-spin structure of themselves. Starting from the differential cross section of an electron-quark collision,

    \begin{align*}
        \frac{\dif \sigma}{ \dif \Omega^\ast} &= \frac{Q_q^2 e^4}{8\pi^2 s}  \frac{1}{\left( 1-\cos\theta^\ast \right)^2} \left[ 1 + \cancelto{}{\frac{1}{4}\left( 1+ \cos\theta^\ast \right)^2 }\right]  \\[0.12in]
        &= \frac{Q_q^2 e^4}{8\pi^2 s}   \frac{1}{\left( 1-\cos\theta^\ast \right)^2} = \frac{Q_q^2 }{8\pi^2 s} \cdot 16\pi^2 \alpha^2 \cdot \frac{E^4}{q^4} = \frac{2Q_q^2 \alpha^2}{q^4} \frac{E^4}{s} = \frac{Q_q^2 \alpha^2s}{8 q^4} 
    \end{align*}\\
where the second angular dependent term is neglected as spin-0 is assumed for quarks. The corresponding Lorentz-invariant expression can be then achieved using chain rules, 

    \begin{align*}
        \frac{\dif \sigma}{\dif q^2} = \frac{\dif \sigma}{\dif \Omega^\ast} \left| \frac{\dif q^2}{\dif \Omega^\ast} \right|^{-1} &=  \frac{\dif \sigma}{\dif \Omega^\ast} \left| \frac{1}{2\pi} \frac{\dif}{\dif \left( \cos \theta^\ast \right)} E^2 \left[ \left( 1-\cos\theta^\ast \right) \right] \right|^{-1} \\[0.12in]
        &= \frac{Q_q^2 \alpha^2s}{8 q^4}  \cdot \frac{2\pi}{E^2} = \frac{\pi \alpha^2  Q_q^2 }{ q^4}\\
    \end{align*}    
Now considering the quark-parton model, one could interpret the above differential cross section in terms of proton structure functions :

    \begin{align*}
        \frac{\dif \sigma}{\dif Q^2} = \frac{\pi \alpha^2 Q_q^2}{Q^4} \left[ 1+0\cdot (1-y)^2 \right] \implies \frac{\dif^2 \sigma}{\dif x \dif Q^2} = \frac{\pi \alpha^2 Q_q^2}{Q^4} \sum_i Q_i^2 q_i^p(x)
    \end{align*}\\
    which implies that as the coefficient for $y^2$ has to be $0$, $F_1^{\text{ep}}(x)$ has  to be 0. 
\end{solution}

\noindent\rule{7in}{1.5pt}

%%%%%%%%%%%%%%%%%%%%%%%%%%%%%%%%%%%%%%%%%%%%%%%%%%%%%%%%%%%%%%%%%%%%%%%%%%%%%%%%%%%%%%%%%%%%%%%%%%%%%%%%%%%%%%%%%%%%%%%%%%%%%%%%%%%%%%%%

\begin{problem}{8.5}
    What is the expected value of $\int_0^1 u(x) - \overbar{u}(x) \dif x$ for the proton?
\end{problem}
\begin{solution}
Splitting the up quark PDF into valence and sea contribution, one obtains

    \begin{align*}
        \int_0^1 u(x) - \overbar{u}(x) \dif x =  \int_0^1 \left[ \left[ u_V(x) + u_S(x) \right] - u_S(x) \right] \dif x = \int_0^1 u_V(x) \dif x = \boxed{2} 
    \end{align*}\\
where the last identity holds as the proton is being considered here.
\end{solution}

\noindent\rule{7in}{1.5pt}

%%%%%%%%%%%%%%%%%%%%%%%%%%%%%%%%%%%%%%%%%%%%%%%%%%%%%%%%%%%%%%%%%%%%%%%%%%%%%%%%%%%%%%%%%%%%%%%%%%%%%%%%%%%%%%%%%%%%%%%%%%%%%%%%%%%%%%%%

\begin{problem}{8.6}
Figure 8.18 shows the raw measurements of the structure function $F_2(x)$ in low-energy electron-deuterium scattering. When combined with the measurements of Figure 8.11, it is found that

\begin{align*}
    \frac{\int_0^1 F^{eD}_2(x) \dif x}{\int_0^1 F^{ep}_2(x) \dif x} \simeq 0.84
\end{align*}\\
Write down the quark-parton model prediction for this ratio and determine the relative fraction of the momen- tum of proton carried by down-/anti-down-quarks compared to that carried by the up-/anti-up-quarks, $f_d/f_u$
\end{problem}
\begin{solution}
Starting from writing down $F_2^{\text{eD}}(x)$ in terms of quark PDFs in the scattered deuterium, 

    \begin{align}
        F_2^{\text{eD}}(x) = x \sum_i Q_i^2  q_i^\text{D}(x) = x \left[ \frac{4}{9} u^\text{D}(x) + \frac{1}{9} d^\text{D}(x) +  \frac{4}{9} \bar{u}^\text{D}(x) + \frac{1}{9} \bar{d}^\text{D}(x)  \right] \label{P8.6.1}
    \end{align}\\
    and upon the fact that deuterium is consisted of one proton and neutron each, the following expressions for $q_i^\text{D}(x)$ holds :

    \begin{align*}
        u^\text{D}(x) = \frac{1}{2}\left[ u^p(x) + u^n(x) \right] =  \frac{1}{2} \left[ u(x) + d(x) \right]  &\andtxt  d^\text{D}(x) =  \frac{1}{2} \left[ d^p(x) + d^n(x) \right] =  \frac{1}{2} \left[ d(x) + u(x) \right] \\[0.12in]
        \bar{u}^\text{D}(x) =  \frac{1}{2}\left[  \bar{u}^p(x) + \bar{u}^n(x) \right] =   \frac{1}{2} \left[ \bar{u}(x) + \bar{d}(x)  \right] &\andtxt  \bar{d}^\text{D}(x) =  \frac{1}{2}\left[  \bar{d}^p(x) + \bar{d}^n(x) \right] =  \frac{1}{2} \left[ \bar{d}(x) + \bar{u}(x)  \right]
    \end{align*}\\
    Inserting these relations into (\ref{P8.6.1}) gives, 

    \begin{align*}
        F_2^{\text{eD}}(x) &=  x \frac{1}{2} \left[ \frac{4}{9} u^\text{D}(x) + \frac{1}{9} d^\text{D}(x) +  \frac{4}{9} \bar{u}^\text{D}(x) + \frac{1}{9} \bar{d}^\text{D}(x)  \right] \\[0.12in]
        &= x  \frac{1}{2}  \left[ \frac{4}{9}  \left\{ u(x) + d(x)  \right\}  + \frac{1}{9}  \left\{ u(x) + d(x)  \right\}  +  \frac{4}{9} \left\{  \bar{u}(x) + \bar{d}(x) \right\} + \frac{1}{9} \left\{  \bar{u}(x) + \bar{d}(x) \right\}\right] \\[0.12in]
        &= x \frac{5}{18} \left[  u(x) + d(x) +  \bar{u}(x) + \bar{d}(x) \right]
    \end{align*}\\
    Then, the ratio between the integrated structure functions can be expressed as the following upon the quark-parton model as,

    \begin{align*}
        \frac{\int_0^1 F_2^\text{eD}(x)}{\int_0^1 F_2^\text{ep}(x)} = \frac{\frac{5}{18}f_u + \frac{5}{18}f_d}{\frac{4}{9}f_u + \frac{1}{9}f_d} = \frac{5(1+r)}{8+2r} \simeq 0.84 \quad \text{where} \quad r \equiv \frac{f_d}{f_u}
    \end{align*}\\
    solving the above in terms of $r$ gives $\boxed{r\equiv f_d/f_u \simeq 0.51}$ which is as expected.
\end{solution}

\noindent\rule{7in}{1.5pt}

%%%%%%%%%%%%%%%%%%%%%%%%%%%%%%%%%%%%%%%%%%%%%%%%%%%%%%%%%%%%%%%%%%%%%%%%%%%%%%%%%%%%%%%%%%%%%%%%%%%%%%%%%%%%%%%%%%%%%%%%%%%%%%%%%%%%%%%%

\begin{problem}{8.7}
    Including the contribution from strange quarks:

    \begin{enumerate}[label=(\alph*)]
        \item Show that $F_2^{ep}(x)$ can be written 
        
        \begin{align*}
            F_2^{\text{ep}}(x) = \frac{4}{9} x \left[ u(x) + \bar{u}(x) \right] + \frac{1}{9} x \left[ d(x) + \bar{d}(x) + s(x) + \bar{s}(x) \right],
        \end{align*}\\
        where $s(x)$ and $\bar{s}(x)$ are the strange quark-parton distribution functions of the proton.
        \item Find the corresponding expression for $F_2^{en}(x)$ and show that
        
        \begin{align*}
            \int_0^1 \frac{F^{\text{ep}}_2(x)-F^{\text{en}}_2(x)}{x} \dif x \approx \frac{1}{3} + \frac{2}{3} \int_0^1 \left[ \bar{u}(x) - \bar{d}(x) \right] \dif x
        \end{align*}\\
        and interpret the measured value of $0.24\pm 0.03$
    \end{enumerate}
\end{problem}
\begin{solution}

\end{solution}

\noindent\rule{7in}{1.5pt}

%%%%%%%%%%%%%%%%%%%%%%%%%%%%%%%%%%%%%%%%%%%%%%%%%%%%%%%%%%%%%%%%%%%%%%%%%%%%%%%%%%%%%%%%%%%%%%%%%%%%%%%%%%%%%%%%%%%%%%%%%%%%%%%%%%%%%%%%

\begin{problem}{8.8}
At the HERA collider, electrons of energy $E_1 = 27.5 \GeV$ collided with protons of energy $E_2 = 820 \GeV$. In deep inelastic scattering events at HERA, show that the Bjorken x is given by

\begin{align*}
    x = \frac{E_3}{E_2} \left[ \frac{1-\cos\theta}{2-\frac{E_3}{E_1}\left(1+\cos\theta\right)} \right]
\end{align*}\\
where $\theta$ is the angle through which the electron has scattered and $E_3$ is the energy of the scattered electron. Estimate $x$ and $Q^2$ for the event shown in Figure 8.13 assuming that the energy of the scattered electron is $250 \GeV$.
\end{problem}
\begin{solution}

\end{solution}

\noindent\rule{7in}{1.5pt}

%%%%%%%%%%%%%%%%%%%%%%%%%%%%%%%%%%%%%%%%%%%%%%%%%%%%%%%%%%%%%%%%%%%%%%%%%%%%%%%%%%%%%%%%%%%%%%%%%%%%%%%%%%%%%%%%%%%%%%%%%%%%%%%%%%%%%%%%
